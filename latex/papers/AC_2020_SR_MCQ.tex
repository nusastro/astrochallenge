\documentclass[a4paper,11pt]{exam}

%%%%%%%%%%%%%%%%%%%%%%%%%%%%%%%%%%%%%%%%%%%%%%%%%%%%%%%%%%%%%%%%%%%%%%%%%%%%%%%%%%%%
%
% PACKAGE IMPORTS %%%%%
%
%%%%%%%%%%%%%%%%%%%%%%%%%%%%%%%%%%%%%%%%%%%%%%%%%%%%%%%%%%%%%%%%%%%%%%%%%%%%%%%%%%%%

%%%%% SOME MISC IMPORTS TO DEFINE STUFF FIRST %%%%%
\usepackage{etoolbox}
\usepackage{pgfkeys}

%%%%% FONTS & SYMBOLS %%%%%
\usepackage[utf8]{inputenc}
\usepackage[T1]{fontenc}
\pgfkeys{
	/ac/.is family, /ac/.cd,
	font/.is choice,
	font/lm/.code={
			\usepackage{lmodern}
			\usepackage{amsfonts,amssymb}
			\providecommand{\tablining}{}
			\providecommand{\propold}{}
			% \renewcommand{\textssc}[1]{\textsc{#1}}
			% \newcommand{\boldsmallcaps}[1]{{\fontfamily{cmr}\textsc{\textbf{#1}}}}
			\usepackage{inconsolata}
		},
	font/mpro/.code={
			\usepackage{MnSymbol}
			\usepackage[
				minionint,
				lf,
				mathtabular,
				loosequotes,
				swash,
				opticals,
				footnotefigures]{MinionPro}
			\providecommand{\tablining}{\figureversion{tab}}
			\providecommand{\propold}{\figureversion{osf}}
			\newcommand{\boldsmallcaps}[1]{\textssc{\textbf{#1}}}
			\usepackage{inconsolata}
		},
	font/lmodern/.code=\pgfkeysalso{font/lm},
	font/minionpro/.code=\pgfkeysalso{font/mpro},
}
\providerobustcmd*{\setfont}[1]{\pgfqkeys{/ac}{#1}}
\usepackage{amsmath}
\usepackage{wasysym}
\usepackage{microtype}

%%%%% GEOMETRY, PAGE SETUP, SPACING, PARAGRAPHING %%%%%
\usepackage[margin=2.5cm,a4paper]{geometry}
\usepackage{titlesec}
\usepackage{multicol}
\usepackage{multirow}
\usepackage{parskip}
\usepackage{tabto}
\usepackage{pdflscape}
\usepackage{enumitem}
\usepackage{adjustbox}
\usepackage[super]{nth}

%%%%% SCIENCE FORMATTING %%%%%
\usepackage{physics}
\usepackage[
	arc-separator = \,,
	retain-explicit-plus,
	%inter-unit-product =\cdot,
	detect-weight=true,
	detect-family=true,
	range-phrase=--,
	range-units=single
]{siunitx}
\usepackage[version=4]{mhchem}
\usepackage[makeroom]{cancel}
\renewcommand{\CancelColor}{\color{red}}

%%%%% GRAPHICS, CAPTIONS, TABLES %%%%%
\usepackage[table,dvipsnames]{xcolor}
\usepackage{graphicx}
\usepackage{float}
\usepackage{tikz,tikz-3dplot}
\usepackage{pgfplots}
\usepackage{pdfpages}
\usepackage[
	justification=centering,
	labelfont={small,bf},
	font={small}
]{caption}
\usepackage{subcaption}
\usepackage{array}
\usepackage{tabularx}
\usepackage{booktabs}
\usetikzlibrary{
	calc,
	arrows,
	arrows.meta,
	positioning,
	decorations.pathreplacing,
	decorations.markings,
	decorations.text,
	calligraphy,
	pgfplots.dateplot
}
\pgfplotsset{compat=1.17}
\usepackage[outline]{contour}
\contourlength{1pt}
\newcommand*\circled[1]{
	\begin{tikzpicture}[baseline=(char.base)]
		\node[shape=circle, draw, minimum size=1.5em, inner sep=0pt, thick] (char) {#1};
	\end{tikzpicture}
}
\tikzset{
	mid arrow/.style={postaction={decorate,decoration={
							markings,
							mark=at position .15 with {\arrow[#1]{stealth}}
						}}},
	>=stealth
}

%%%%% REFERENCES AND LINKS %%%%%
\usepackage{hyperref}
\usepackage[noabbrev]{cleveref}

%%%%% MISCELLANEOUS %%%%%
\usepackage[useregional,calc]{datetime2}
\DTMlangsetup[en-GB]{ord=raise}

%%%%%%%%%%%%%%%%%%%%%%%%%%%%%%%%%%%%%%%%%%%%%%%%%%%%%%%%%%%%%%%%%%%%%%%%%%%%%%%%%%%%
%
% MATHS MACROS %%%%%
%
%%%%%%%%%%%%%%%%%%%%%%%%%%%%%%%%%%%%%%%%%%%%%%%%%%%%%%%%%%%%%%%%%%%%%%%%%%%%%%%%%%%%

\newcommand{\tomb}{\quad\blacksquare{}}

%%%%%%%%%%%%%%%%%%%%%%%%%%%%%%%%%%%%%%%%%%%%%%%%%%%%%%%%%%%%%%%%%%%%%%%%%%%%%%%%%%%%
%
% CREF AND HYPERREF SETUP %%%%%
%
%%%%%%%%%%%%%%%%%%%%%%%%%%%%%%%%%%%%%%%%%%%%%%%%%%%%%%%%%%%%%%%%%%%%%%%%%%%%%%%%%%%%

\crefdefaultlabelformat{#2\textbf{#1}#3}

\creflabelformat{equation}{#2\textbf{(#1)}#3}
\creflabelformat{figure}{#2\textbf{#1}#3}

\crefname{equation}{\textbf{equation}}{\textbf{equations}}
\Crefname{equation}{\textbf{Equation}}{\textbf{Equations}}
\crefname{figure}{\textbf{Figure}}{\textbf{Figures}}
\Crefname{figure}{\textbf{Figure}}{\textbf{Figures}}
\crefname{table}{\textbf{Table}}{\textbf{Tables}}
\Crefname{table}{\textbf{Table}}{\textbf{Tables}}
\crefname{appendix}{\textbf{Appendix}}{\textbf{Appendices}}
\Crefname{appendix}{\textbf{Appendix}}{\textbf{Appendices}}
\crefname{section}{\textbf{\S}}{\textbf{\S}}
\Crefname{section}{\textbf{\S}}{\textbf{\S}}
\crefname{algorithm}{\textbf{Algorithm}}{\textbf{Algorithms}}
\Crefname{algorithm}{\textbf{Algorithm}}{\textbf{Algorithms}}

%%%%% SPECIFIC TO EXAM CLASS%%%%%
\creflabelformat{question}{#2\textbf{#1}.#3}
\creflabelformat{partno}{(#2\textbf{#1}#3)}
\creflabelformat{subpart}{(#2\textbf{#1}#3)}

\crefname{question}{question}{questions}
\Crefname{question}{Question}{Questions}
\crefname{partno}{}{}
\Crefname{partno}{}{}
\crefname{subpart}{}{}
\Crefname{subpart}{}{}

\hypersetup{
	colorlinks   = true,            %Colours links instead of ugly boxes
	urlcolor     = NavyBlue,        %Colour for external hyperlinks
	linkcolor    = Magenta,         %Colour of internal links
	citecolor    = Aquamarine       %Colour of citations
}

%%%%%%%%%%%%%%%%%%%%%%%%%%%%%%%%%%%%%%%%%%%%%%%%%%%%%%%%%%%%%%%%%%%%%%%%%%%%%%%%%%%%
%
%%%%% EXAM CLASS SETUP %%%%%
%
%%%%%%%%%%%%%%%%%%%%%%%%%%%%%%%%%%%%%%%%%%%%%%%%%%%%%%%%%%%%%%%%%%%%%%%%%%%%%%%%%%%%

%%%%% CHOICES ON ONE PAGE %%%%%
\BeforeBeginEnvironment{choices}{\par\nopagebreak\minipage{\linewidth}}
\AfterEndEnvironment{choices}{\endminipage}

%%%%% QUESTION/CHOICE LABELS %%%%%
\renewcommand{\questionlabel}{\thequestion.\hfill}
\renewcommand{\subpartlabel}{(\thesubpart)}
\renewcommand{\choicelabel}{\circled{\thechoice}}

%%%%% POINTS FORMATTING %%%%%
\renewcommand{\questionshook}{
	\setlength{\rightpointsmargin}{1.75cm}
	\setlength{\itemsep}{30pt}
}

%%%%% QUESTION/PART/SUBPART INDENTATION %%%%%
\renewcommand{\partshook}{
	\renewcommand\makelabel[1]{\rlap{##1}\hss}
	% \setlength{\itemsep}{6pt}
}

\renewcommand{\subpartshook}{
	\renewcommand\makelabel[1]{\rlap{##1}\hss}
	% \setlength{\itemsep}{6pt}
}

\renewcommand{\choiceshook}{
	\setlength{\labelsep}{10pt}
	\settowidth{\leftmargin}{\circled{W}.\hspace{5pt}\hspace{0em}}
	\setlength{\itemsep}{10pt}
}

\renewcommand{\solutiontitle}{
	\noindent\textbf{Solution:}\par\noindent
}

%%%%% ONEPAR CHOICES SPREAD %%%%%
\patchcmd{\oneparchoices}{\penalty -50\hskip 1em plus 1em\relax}{\hfill}{}{}
\patchcmd{\oneparchoices}{\penalty -50\hskip 1em plus 1em\relax}{\hfill}{}{}

%%%%% SOLUTION ENVIRONMENT %%%%%
\SolutionEmphasis{\color{NavyBlue}}
\correctchoiceemphasis{\color{NavyBlue}\bfseries\boldmath}
\marksnotpoints{}
\pointsinrightmargin{}
\pointsdroppedatright{}
\pointformat{\bfseries\textbf[\themarginpoints]}

%%%%% REDEFINE COVER PAGINATION AS ARABIC %%%%%
\makeatletter
\renewenvironment{coverpages}{%
	\ifnum \value{numquestions}>0\relax
		\ClassError{exam}{%
			Coverpages cannot be used after questions have begun.\MessageBreak
		}{%
			All question, part, subpart, and subsubpart environments
			\MessageBreak
			must begin after the cover pages are complete.\MessageBreak
		}%
	\fi
	\@coverpagestrue
	% \pagenumbering{arabic}%
	\adj@hdht@ftht
	\thispagestyle{headandfoot}
}{%
	\clearpage
	% \setcounter{num@coverpages}{\value{page}}%
	% \addtocounter{num@coverpages}{-1}%
	% \pagenumbering{arabic}%
	% Bugfix, Version 2.307\beta, 2009/06/11:
	% We have to say \@coverpagesfalse before \adj@hdht@ftht
	% because we're still inside the group created by the
	% coverpages environment and we want to set the
	% extraheadheight and extrafootheight to the values correct
	% for the first non-cover page:
	\@coverpagesfalse
	\adj@hdht@ftht
}
\makeatother

%%%%% SHOW 'SOLUTIONS' IN TITLEPAGE IF ANSWERS PRINTED %%%%%
\providerobustcmd*{\printsolntitle}{
	\ifprintanswers{
		\vspace{20pt}
		\fbox{\fbox{\Huge{\textssc{\textbf{\textcolor{red}{solutions}}}}}}
		\vspace{20pt}
	}
	\fi{}
}
\providerobustcmd*{\envupspace}{\ifanswers{\vspace{20pt}}}

%%%%% REMOVE SPACES FROM EnvFullwidth command
\BeforeBeginEnvironment{EnvFullwidth}{\vspace{-10pt}}
\AfterEndEnvironment{EnvFullwidth}{\vspace{-35pt}}

%%%%%%%%%%%%%%%%%%%%%%%%%%%%%%%%%%%%%%%%%%%%%%%%%%%%%%%%%%%%%%%%%%%%%%%%%%%%%%%%%%%%
%
%%%%% SIUNITX SETUP %%%%%
%
%%%%%%%%%%%%%%%%%%%%%%%%%%%%%%%%%%%%%%%%%%%%%%%%%%%%%%%%%%%%%%%%%%%%%%%%%%%%%%%%%%%%

\DeclareSIUnit{\year}{y}
\DeclareSIUnit{\AU}{AU}
\DeclareSIUnit{\parsec}{pc}
\DeclareSIUnit{\lightyear}{ly}
\DeclareSIUnit{\earthmass}{\textit{M}_{\earth}}
\DeclareSIUnit{\jupitermass}{\textit{M}_{J}}
\DeclareSIUnit{\solarmass}{\textit{M}_{\astrosun}}
\DeclareSIUnit{\atm}{atm}

%%%%%%%%%%%%%%%%%%%%%%%%%%%%%%%%%%%%%%%%%%%%%%%%%%%%%%%%%%%%%%%%%%%%%%%%%%%%%%%%%%%%
%
%%%%% MISCELLANEOUS %%%%%
%
%%%%%%%%%%%%%%%%%%%%%%%%%%%%%%%%%%%%%%%%%%%%%%%%%%%%%%%%%%%%%%%%%%%%%%%%%%%%%%%%%%%%

%%%%% SET NUMBERED AND BULLETED LIST MARGIN
\setlist[itemize, 1]{left=0pt}
\setlist[enumerate, 1]{left=0pt,label=\arabic*.}

%%%%% PURPOSE FORGOTTEN %%%%%
\AtBeginEnvironment{tabularx}{
	\tablining
	\sisetup{text-rm={\tablining}}
}

\renewcommand\tabularxcolumn[1]{m{#1}}

\titlelabel{\vspace{-4cm}\thetitle\hspace{10pt}}
\setlength{\jot}{8pt}

%%%%% COPY-PASTABLE PDF %%%%%
\input{glyphtounicode}
\pdfgentounicode=1

%%%%% MICROTYPE SETUP FOR HYPHENATION %%%%%
\pretolerance=2500
\tolerance=4500
\emergencystretch=0pt
\righthyphenmin=4
\lefthyphenmin=4

%%%%% DEFAULT CENTRED FIGURES %%%%%
\AtBeginEnvironment{figure}{\centering}

%%%%% RADIO BUTTONS %%%%%
\makeatletter
\providerobustcmd*{\radiobutton}{%
	\@ifstar{\@radiobutton0}{\@radiobutton1}%
}
\providerobustcmd*{\@radiobutton}[1]{%
	\begin{tikzpicture}
		\pgfmathsetlengthmacro\radius{height("X")/2}
		\draw[radius=\radius] circle;
		\ifcase#1 \fill[radius=.6*\radius] circle;\fi
	\end{tikzpicture}
}
\makeatother

%%%%%%%%%%%%%%%%%%%%%%%%%%%%%%%%%%%%%%%%%%%%%%%%%%%%%%%%%%%%%%%%%%%%%%%%%%%%%%%%%%%%
%
% ASTROCHALLENGE SETUP AND MACROS %%%%%
%
%%%%%%%%%%%%%%%%%%%%%%%%%%%%%%%%%%%%%%%%%%%%%%%%%%%%%%%%%%%%%%%%%%%%%%%%%%%%%%%%%%%%

%%%%% ASTROCHALLENGE PAPER DATE %%%%%
\providerobustcmd*{\setacpaperdate}[1]{\DTMsavedate{theacpaperdate}{#1}}
\providerobustcmd*{\acpaperdate}{\DTMusedate{theacpaperdate}}

%%%%% ASTROCHALLENGE SMALL CAPS %%%%%
\providecommand{\astrochallenge}{\textbf{\textssc{AstroChallenge \propold\DTMfetchyear{theacpaperdate}}}}

%%%%% AUTOMATIC CATEGORY AND ROUND WITH KEY-VALUE SYSTEM %%%%%
\providebool{ismcq}
\providebool{isteam}
\providebool{isobs}
\providebool{isdaq}

\pgfkeys{
	/ac/.is family, /ac/.cd,
	category/.is choice,
	round/.is choice,
	% DEFINE ACTIONS FOR CATEGORY
	/ac/category/.cd,
	jnr/.code={\edef\category{Junior}},
	snr/.code={\edef\category{Senior}},
	junior/.code=\pgfkeysalso{category/jnr},
	senior/.code=\pgfkeysalso{category/snr},
	% DEFINE ACTIONS FOR CHOICES
	/ac/round/.cd,
	mcq/.code={\edef\round{MCQ}\booltrue{ismcq}},
	team/.code={\edef\round{Team}\booltrue{isteam}},
	obs/.code={\edef\round{Observation}\booltrue{isobs}},
	daq/.code={\edef\round{Data Analysis}\booltrue{isdaq}},
	% ALTERNATIVE NAMES
	MCQ/.code=\pgfkeysalso{round/mcq},
	Team/.code=\pgfkeysalso{round/team},
	Obs/.code=\pgfkeysalso{round/obs},
	DAQ/.code=\pgfkeysalso{round/daq},
}
\providecommand*{\setcatround}[1]{\pgfqkeys{/ac}{#1}}
\providecommand*{\catround}{\textbf{\textssc{\category{} \round{} Round}}}

%%%%% ASTROCHALLENGE TITLING %%%%%
\title{{\Huge\astrochallenge{}}}
\author{\textcopyright \ National University of Singapore Astronomical Society \\
	\textcopyright \ Nanyang Technological University Astronomical Society \\
}

%%%%% ACCESS TITLE COMMANDS %%%%%
\makeatletter
\let\newtitle\@title
\let\newauthor\@author
\makeatother

%%%%% EXAM HEADER/FOOTER SETUP %%%%%
\coverheader{\small{\astrochallenge{}}}{}{\small{\catround{}}}
\header{\small{\astrochallenge{}}}{}{\small{\catround{}}}
\headrule{}

%%%%% PAGE NUMBERING: 'Page X of total' %%%%%
\coverfooter{}{Page \thepage{} of \totalnumpages{}}{\oddeven{\textbf{[Turn over]}}}
\footer{}{Page \thepage{} of \totalnumpages{}}{
	\oddeven{
		\iflastpage{\textbf{[End of paper]}}{\textbf{[Turn over]}}
	}
}

%%%%% ASTROCHALLENGE INSTRUCTIONS: MCQ %%%%%
\providerobustcmd*{\acmcqinst}{
	\begin{enumerate}[itemsep=8pt]
		\item This paper consists of \textbf{\totalnumpages} printed pages, including this cover page.
		\item Do \textbf{NOT} turn over this page until instructed to do so.
		\item You have \textbf{2 hours} to attempt all questions in this paper. If you think there is more than one correct answer, choose the most correct answer.
		\item At the end of the paper, submit this booklet together with your answer script.
		\item Your answer script should clearly indicate your name, school, and team.
		\item It is \textit{your} responsibility to ensure that your answer script has been submitted.
	\end{enumerate}
}

%%%%% ASTROCHALLENGE INSTRUCTIONS: TEAM %%%%%
\providerobustcmd*{\acteaminst}{
	\begin{enumerate}[itemsep=8pt]
		\item This paper consists of \textbf{\totalnumpages} printed pages, including this cover page.
		\item Do \textbf{NOT} turn over this page until instructed to do so.
		\item You have \textbf{2 hours} to attempt all questions in this paper.
		\item At the end of the paper, submit this booklet together with your answer script.
		\item Your answer script should clearly indicate your name, school, and team.
		\item It is your responsibility to ensure that your answer script has been submitted.
		\item The marks for each question are given in brackets in the right margin, like such: \textbf{[2]}.
		\item The \textbf{alphabetical} parts (i) and (l) have been intentionally skipped, to avoid confusion with the Roman numbering of (i).
	\end{enumerate}
}

%%%%% ASTROCHALLENGE INSTRUCTIONS: DAQ %%%%%
\providerobustcmd*{\acdaqinst}{
	In this part of \astrochallenge{}, you will work with a moderately large (approx. \num{4000} points) data set. You will process this data set, analyse it, observe trends, and draw conclusions. \textbf{There are no right or wrong answers}; you will be marked solely on the quality of your analysis, even if your statistical methods are incorrect.\\[8pt]
	We \textbf{strongly} recommend you use industry-standard tools like \texttt{Microsoft Excel}\texttrademark, \texttt{RStudio} or various \texttt{Python} libraries to process the data.\par
}

%%%%% ASTROCHALLENGE INSTRUCTIONS: OBS %%%%%
\providerobustcmd*{\acobsinst}{
	\begin{enumerate}[itemsep=8pt]
		\item This paper consists of \textbf{\totalnumpages} printed pages, including this cover page.
		\item Do \textbf{NOT} turn over this page until instructed to do so.
		\item You have \textbf{1 hour and 30 minutes} to attempt all questions in this paper.
		\item At the end of the paper, submit this booklet together with your answer script.
		\item Your answer script should clearly indicate your name, school, and team.
		\item It is your responsibility to ensure that your answer script has been submitted.
	\end{enumerate}
}

%%%%% ASTROCHALLENGE INSTRUCTIONS %%%%%
\providerobustcmd*{\acinstructions}{
	\ifbool{ismcq}{\acmcqinst}{
		\ifbool{isteam}{\acteaminst}{
			\ifbool{isdaq}{\acdaqinst}{
				\ifbool{isobs}{\acobsinst}{}
			}
		}
	}
}

%%%%% ASTROCHALLENGE INSTRUCTION BOX %%%%%
\providerobustcmd*{\acinstbox}{
	\vspace*{20pt}
	\fbox{
		\fbox{
			\parbox{0.8\textwidth}{
				\vspace*{5pt}
				\begin{center}
					{\large{\textbf{\textssc{please read these instructions carefully.}}}}
				\end{center}
				\vspace*{10pt}

				\acinstructions{}

				\vspace*{5pt}
			}
		}
	}
}

%%%%% ASTROCHALLENGE COVER PAGE %%%%%
\providerobustcmd*{\accoverpage}{
	\begin{coverpages}
		\begin{center}
			\includegraphics[width=0.8\linewidth]{../graphics/misc/logo.jpg}

			\newtitle{}

			{\Huge{\catround{}}}

			\printsolntitle{}

			\acpaperdate{}

			\acinstbox{}

			\vspace*{20pt}

			\newauthor{}
		\end{center}
	\end{coverpages}
}


\coverfirstpageheader{\footnotesize{\textsc{\textbf{AstroChallenge \figureversion{osf}2019}}}}{}{\footnotesize{\textsc{\textbf{Senior MCQ Round}}}}
\firstpageheader{\footnotesize{\textsc{\textbf{\figureversion{osf}AstroChallenge 2019}}}}{}{\footnotesize{\textsc{\textbf{Senior MCQ Round}}}}
\runningheader{\footnotesize{\textsc{\textbf{AstroChallenge \figureversion{osf}2019}}}}{}{\footnotesize{\textsc{\textbf{Senior MCQ Round}}}}

\begin{document}
\begin{coverpages}
	\begin{center}
		\includegraphics[width=0.8\linewidth]{Graphics/Misc/logo.jpg}

		\newtitle

		{\Huge{\textssc{\textbf{Senior MCQ Round}}}}

		%\printsolns

		{\Large{Monday, \nth{3} June 2019}}

		\vspace*{20pt}
		\fbox{
			\fbox{
				\parbox{0.9\textwidth}{
					\vspace*{5pt}
					\begin{center}
						{\Large{\textsc{\textbf{Please read these instructions carefully.}}}}
					\end{center}
					\vspace*{10pt}

					\begin{enumerate}[itemsep=8pt]
						\item This paper consists of \textbf{\totalnumpages} printed pages, including this cover page.
						\item Do \textbf{NOT} turn over this page until instructed to do so.
						\item You have \textbf{2 hours} to attempt all questions in this paper. If you think there is more than
						one correct answer, choose the most correct answer.
						\item At the end of the paper, submit this booklet together with your answer script.
						\item Your answer script should clearly indicate your name, school, and team.
						\item It is your responsibility to ensure that your answer script has been submitted.
					\end{enumerate}
					\vspace*{5pt}
				}
			}
		}
	\end{center}
\pagenumbering{arabic}
\end{coverpages}
\newpage

\begin{questions}
\question
	Which of the following is the best approximate ratio of the brightness of a magnitude 1 star, to that of a magnitude 3 star?

	\begin{oneparchoices}
		\choice $ 2 $
		\choice $ 4 $
		\correctchoice $\displaystyle \frac{631}{100} $
		\choice $\displaystyle \frac{100}{631} $
		\choice $\displaystyle \frac{89}{5} $
	\end{oneparchoices}
	\begin{solution}
		Repurpose the Luminosity-Absolute Magnitude relationship in the Formula Book, and it follows that the magnitude 1 star is $ 10^{\frac{3-1}{2.5}} \approx 6.31 $ times as bright as the magnitude 3 star. Hence the nearest ratio is $  \frac{631}{100}$.
	\end{solution}

\filbreak
\question
	A rule of thumb is that it is best to stargaze during a new moon, rather than during a full moon. Why?
	\begin{choices}
		\choice The Moon will pass in front of several objects in the sky.
		\choice Celestial objects only align with their charted positions, under a new moon.
		\choice Telescopes \textbf{only} function under a new moon, and \textbf{never} under a full moon.
		\correctchoice The glare of the Moon washes out the fainter objects in the sky.
		\choice \textbf{All} celestial objects are only visible in the sky under a new moon.
	\end{choices}
	\begin{solution}
		The Moon is the brightest object in the night sky. The glare of the Moon will wash out any dimmer deep sky objects, magnified or otherwise.
	\end{solution}

\filbreak
\question
	During an astronomy outreach event, a member of the public asked, `How would you visually tell a star apart from a planet?' Which of the following would be the best response?
	\begin{choices}
		\choice Planets always appear \textbf{brighter} than stars.
		\correctchoice Stars `twinkle', but planets don't.
		\choice Planets appear significantly \textbf{bigger} than stars to the naked eye.
		\choice Stars are of a different colour compared to the planets.
		\choice Planets are only visible through a telescope.
	\end{choices}
	\begin{solution}
		Although stars are enormous, they are several orders of magnitude further away from the Earth, than the planets are. Through a telescope, it is clear that the planets are actually \textit{discs}, while stars are \textit{still} apparent \textit{points} of light.

		This means that the refractive and turbulent effect of the atmosphere causes the star's image to change slightly in brightness and position, and hence the `twinkle'. The \textit{apparent} size of the planets in the sky means that this turbulent effect is not pronounced, hence presenting a relatively stable image to the eye.
	\end{solution}

\filbreak
\question
	Which of the following is an approximate ratio of a man’s weight on Saturn to his weight on Earth?

	\begin{oneparchoices}
		\correctchoice $1$
		\choice $ \displaystyle \frac{1}{2} $
		\choice $ \displaystyle \frac{1}{3} $
		\choice $ \displaystyle \frac{1}{5} $
		\choice $ \displaystyle \frac{1}{6} $
	\end{oneparchoices}
	\begin{solution}
		Compute the Newtonian gravitational acceleration at the given radii of both planets.
	\end{solution}

\filbreak
\question
	Which of the following stellar spectral classes has the highest surface temperature?

	\begin{oneparchoices}
		\choice A
		\correctchoice	B
		\choice	F
		\choice G
		\choice K
	\end{oneparchoices}
	\begin{solution}
		In order from hottest to coolest: O, B, A, F, G, K, M.
	\end{solution}

\filbreak
\question
	While Clarence was reading a book on telescopes, he came across a rather interesting-looking equation, which suggested that for a night sky object with a declination of $ \delta $, if an object was to be kept within the field of view (FOV) for a certain amount of time $ t $ seconds, the required FOV (in degrees) would be given by the equation:
	\[ \text{FOV} = kt\cos{\delta}\]
	where $ k $ is some constant to be determined. This is provided that we position the object to one side of the eyepiece (along the edge of the field of view) and turn off the telescope drive.

	Which of the following could be a possible value of the constant $ k $?

	\begin{oneparchoices}
		\choice \num{1.04d-3}
		\choice \num{2.09d-3}
		\correctchoice \num{4.18d-3}
		\choice \num{6.25d-3}
		\choice \num{8.33d-3}
	\end{oneparchoices}
	\begin{solution}
		Since the equation must work for all declination values, we can easily set it to be at $ \delta= \ang{0} $. Hence, these objects will tend to drift across the diameter of the field of view. Thus, to determine the FOV, i.e. the angular displacement, we know that the angle travelled (ie. “FOV”) of $ \ang{360} $ is done through 24 hours (because that is the total angle rotated by the Earth about its own axis).

		Hence, $\text{FOV}=\ang{360}, \, t = \SI{86164}{\second}, \,\delta=\ang{0} $ gives $ k = \num{4.18d-3} $.

		Note that it does not matter if we use a sidereal day or a solar day. The answers will round off to be approximately this value, either way.
	\end{solution}

\filbreak
\question
	Table \ref{q7} lists some information about three stars: Vega, Aldebaran, and 10 Lacertae.
	\begin{figure}[H]
		\centering
		\begin{tabularx}{0.75\textwidth}{@{}Xllr@{}}
			\toprule
			\textbf{Star} & \textbf{Vega} & \textbf{Aldebaran} & \textbf{10 Lacertae} \\ \midrule
			Spectral type & A0V & K5III & O9V \\
			Declination & \ang{-38;47;01} & \ang{+16;30;33} & \ang{-39;03;} \\
			Apparent magnitude & \num{+0.026} & \num{+0.86} & \num{+4.88} \\
			Distance from Earth & \SI{25.04}{\lightyear} & \SI{65.3}{\lightyear} & \SI{2330.9}{\lightyear} \\
			Colour index (B–V) & +0.00 & +1.44 & -0.21 \\ \bottomrule
		\end{tabularx}
		\renewcommand{\figurename}{Table}
		\caption{Spectral data of three stars}
		\label{q7}
	\end{figure}
	Which of the following statements is \textbf{incorrect}?
	\begin{choices}
		\choice 10 Lacertae is the hottest star.
		\choice All three stars are in the Milky Way.
		\choice There are exactly \textbf{two} main-sequence stars.
		\correctchoice Aldebaran cannot be seen in the southern hemisphere.
		\choice Vega is the dimmest in terms of absolute magnitude.
	\end{choices}
	\begin{solution}
		While Aldebaran lies in the northern half of the celestial sphere, that does not imply that its invisible to \textbf{\textit{all}} observers in the southern hemisphere.
	\end{solution}

\filbreak
\question
	Contrary to what one might expect, the latest sunrise of the year does not occur during the
	respective hemispheres' winter solstices. Why is this so?

	You may assume this location does not lie within equatorial latitudes.

	\begin{choices}
		\choice	The \textbf{Earth} is significantly oblate. This leads to variations in the time of sunrise due to horizon effects.
		\choice	The \textbf{Sun} is significantly oblate. This leads to variations in the time of sunrise as our perspective of the Sun’s shape changes across time.
		\correctchoice	As the Earth’s orbit is elliptical, the Sun appears to drift across the night sky at different rates, over the year.
		\choice	Milankovitch cycles affect the Earth’s axial tilt, causing a large constant drift in sunrise timing.
		\choice	The question statement is false. The latest sunrise of the year \textit{does} in fact occur during the respective hemispheres' winter solstices.
	\end{choices}
	\begin{solution}
		\circled{A} and \circled{B} are clearly false, and while Milankovitch cycles do affect Earth’s axial tilt, it takes place over long timescales and thus is too small to be easily noticeable.
	\end{solution}

\filbreak
\question
	What is meant by the statement, `an object has a redshift of $ z = 1.2 $'?
	\begin{choices}
		\choice	The object has a luminosity distance of \SI{1.2}{\mega\parsec}.
		\choice	The object has a co-moving distance of \SI{1.2}{\mega\parsec}.
		\choice	If $x\,\%$  of the total emission of our Sun is red light at \SI{600}{\nano\metre}, then $(1.2x) \,\%$ of the total emission of the object is red light at \SI{600}{\nano\metre}.
		\correctchoice	The wavelengths of the object’s emitted light are all increased by \num{1.2}$\times$ compared to the original wavelength.
		\choice	The object’s observed velocity is 1.2$\times$ greater than the velocity generated by the Hubble flow at that distance.
	\end{choices}
	\begin{solution}
		The formula for $ z $ in the Formula Book is given: simply translate the maths.
	\end{solution}

\filbreak
\question
	Figure \ref{q11} depicts the light curve of a hypothetical star system.
	\begin{figure}[H]
		\centering
		\includegraphics{Graphics/Questions/LightCurve/lightcurve.pdf}
		\caption{The observed light curve of a hypothetical star system}
		\label{q11}
	\end{figure}

	Which of the following are plausible explanations for this observation?

	\begin{enumerate}[align=left,labelsep=0pt,leftmargin=0pt,labelsep=3pt,itemindent=23pt,parsep=6pt,label=\textbf{\Roman{*}}.]
		\item The star is part of a binary-star system, and this curve is a result of eclipses.
		\item The star is likely a small red dwarf with at least two massive exoplanets.
		\item The star is a classical Cepheid variable.
		\item The star is a RR Lyrae variable.
	\end{enumerate}
	\begin{choices}
		\correctchoice \textbf{I} and \textbf{II}
		\choice \textbf{II} only
		\choice \textbf{III} and \textbf{IV}
		\choice \textbf{III }only
		\choice \textbf{I}, \textbf{II}, \textbf{III}, and \textbf{IV}
	\end{choices}
	\begin{solution}
		\textbf{I} and \textbf{II} are correct. Otherwise, Cepheids and RR Lyrae have very different light curves.
	\end{solution}

\filbreak
\question
	Brian noticed two satellites passing directly overhead in opposite directions. He decided to label the satellite moving from West to East \textbf{A}, and the satellite moving from East to West \textbf{C}. He then noticed the same two satellites passing directly over his head again, exactly 8 hours later.

	Suppose that the satellites are in circular orbits, and that Brian is on the equator. What is the ratio of the orbital radius of satellite \textbf{A} to that of satellite \textbf{C}?

	\begin{oneparchoices}
		\choice	$ 0.481 $
		\correctchoice	$ 0.630 $
		\choice	$ 1 $
		\choice	$ 1.587 $
		\choice	$ 2.080 $
	\end{oneparchoices}
	\begin{solution}
		Note that the Earth rotates prograde---when viewed from above the North pole, it rotates anticlockwise, i.e. from West to East.

		Notice that for satellite A, since it is orbiting in the same direction as the direction of rotation of the Earth, he must have covered an angle equivalent to one entire round, and the angle rotated by the Earth during these 8 hours. Hence, $ \theta_A=\ang{360}+\ang{120}=\ang{480} $ (8 hours amount to $ \ang{120} $)

		A similar logic can be applied to satellite C. However, since it is orbiting in the opposite direction, the total angle covered is one entire round, but with the angle rotated by the Earth during these 8 hours being subtracted off.

		$ \therefore \, \theta_B=\ang{360}-\ang{120}=\ang{240}$.
		Now, these angles are covered in the same amount of time, 8 hours. Hence, we can deduce that the angular velocity of A is twice of C, i.e, $\displaystyle \frac{\omega_C}{\omega_A}=\frac{1}{2} $. This is equivalent to saying that the orbital period of A is half that of C.

		From Kepler’s \nth{3} Law of Planetary Motion, we can obtain the following equation: $ \omega^2=\frac{GM}{r^3} $.

		Hence, $\displaystyle \frac{r_A}{r_C}=\left(\frac{\omega_C}{\omega_A}\right)^\slantfrac{2}{3} $
		Therefore, the required ratio is $\displaystyle (0.5)^\slantfrac{2}{3}=0.630 $.
	\end{solution}

\filbreak
\question
	Mercury's orbit has a semi-major axis of \SI{0.387}{\AU}. At aphelion, Mercury is \SI{0.467}{\AU} from the Sun; at perihelion, it is \SI{0.307}{\AU} from the Sun. It has an orbital period of \SI{0.241}{\year}. What is the \textbf{ratio} of its \textbf{angular momentum} at its aphelion to its perihelion?
	\begin{choices}

		\choice $ 0.5 $
		\choice $ 1.52 $
		\choice $ 1.21 $
		\choice $ 0.657 $
		\correctchoice None of the above.
	\end{choices}
	\begin{solution}
		Trick question. Angular momentum is \textit{always} conserved, so the ratio should be \textbf{1}. Hence none of the options are correct except \circled{E}. Information is given as a red herring.
	\end{solution}

\filbreak
\question
	Which of the following statements about the celestial sphere is \textbf{false}?

	\begin{choices}
		\choice The angle between the ecliptic and the celestial equator is the axial tilt of the Earth.
		\correctchoice An observer stationed at a \textit{fixed} location would observe \textit{exactly} the same night sky every year at the same date and time.
		\choice The rotational axis of the Earth always points at the celestial North and South poles.
		\choice Most of the planets of the solar system lie close to the ecliptic.
		\choice The equatorial coordinate system uses declination and right ascension, the latter being the `longitude' of the object from the March equinox.
	\end{choices}
	\begin{solution}
		The visible planets all have synodic periods that are significantly different from 1 year.

		Furthermore, due to axial precession, the rotational axis of the Earth changes its orientation, so the night sky changes slowly with a period of \num{26000} years. The rest of the options are self-explanatory.
	\end{solution}

\filbreak
\question
	If the escape velocity of an object of one solar mass exceeded the speed of light, what would be its radius?
	\begin{choices}
		\choice \SI{4.43e11}{\metre}
		\choice \SI{8.85e11}{\metre}
		\correctchoice \SI{2950}{\metre}
		\choice \SI{4.43e13}{\metre}
		\choice \SI{1476}{\metre}
	\end{choices}
	\begin{solution}
		This question is essentially asking what would be the radius of the Sun, if it were to become a black hole (i.e. its Schwarzschild radius). Hence options \circled{A}, \circled{B} and \circled{D} can be eliminated immediately, as these values are far too large.

		Thence, it is a straightforward matter of plugging numbers into the formula:
		\begin{flalign*}
		r_s &= \frac{2GM_{\astrosun}}{c^2} &\\
		&= \frac{2\left(\SI{6.67384e-11}{\newton\metre\squared\per\kg\squared}\right)\left(\SI{1.989e30}{\kg}\right)}{\left(\SI{2.99792458e8}{\metre\per\second}\right)^2} &\\
		&= \textbf{\SI{2950}{\metre}}
		\end{flalign*}
	\end{solution}

\filbreak
\question
	Ceres has an eccentricity of \num{0.0758}. What is the ratio of its \textbf{angular velocity} at its perihelion to that at its aphelion?

	\begin{oneparchoices}
		\choice \num{0.28}
		\choice \num{1}
		\choice \num{1.16}
		\correctchoice \num{1.35}
		\choice \num{3.63}
	\end{oneparchoices}
	\begin{solution}
		Kepler's \nth{2} Law states that the \textbf{linear} velocity ratio at perihelion to aphelion is inversely proportional to distance---in other words, angular momentum is conserved.

		On the other hand, \textbf{angular} velocity is \textit{inversely} proportional to $ r^2 $. The ratio of distance is $ \frac{1+e}{1-e} $. \circled{B} is a red herring for those who confuse angular momentum with angular velocity.
	\end{solution}

\filbreak
\question
	Figure \ref{q16} shows an empirically derived relationship between mass and radius for a particular subset of stars, given a certain surface temperature.
	\begin{figure}[H]
		\centering
		\includegraphics[width=0.75\linewidth]{Graphics/Questions/MassRadius/massradiuswhitedwarfs}
		\renewcommand{\figurename}{Figure}
		\caption{Mass-radius relationship of a certain subset of stars}
		\label{q16}
	\end{figure}
	Which of the following stars might belong in the subset that Figure \ref{q16} depicts?

	\hspace{21pt} \textbf{Star} \tabto{80pt}{\textbf{Stellar classification}} \tabto{200 pt}{\textbf{Type}}
	\begin{choices}
		\correctchoice Sirius B     \tabto{95pt}{A1} \tabto{178pt}{white dwarf}
		\choice $ \beta $ Tauri     \tabto{95pt}{B7} \tabto{178pt}{blue giant}
		\choice Rigel               \tabto{95pt}{B8} \tabto{178pt}{blue supergiant}
		\choice $ \alpha $ Herculis \tabto{95pt}{M5} \tabto{178pt}{red giant}
		\choice $ \sigma $ Orionis  \tabto{95pt}{O9} \tabto{178pt}{main-sequence}
	\end{choices}
	\begin{solution}
		The major challenge in this question is to identify that the subset is that of white dwarves. There are 3 clues to point to that:
		\begin{enumerate}[nosep]
			\item The occurrence of the stars sharply falls at the Chandrasekhar limit
			\item The radius range is comparable to that of terrestrial planets
			\item Have extremely high surface temperatures
		\end{enumerate}
		Furthermore, Sirius B is the nearest white dwarf to Earth and participants are expected to know this.
	\end{solution}

\vspace{-20pt}
\filbreak
\question
	Below are five solar system objects and their respective deities according to ancient Vedic astrology, dated to 700 BCE. Which of the five object-deity pairs is \textbf{incorrect}?

	\tabto{23pt}{\textbf{Object}} \tabto{103pt}{\textbf{Hindu deity}}
	\begin{choices}
		\choice Sun     \tabto{80pt}{Surya}
		\choice Mercury \tabto{80pt}{Budha}
		\choice Mars    \tabto{80pt}{Ma\.{n}gala}
		\choice Saturn  \tabto{80pt}{Sani}
		\correctchoice Neptune \tabto{80pt}{Rahu}
	\end{choices}
	\begin{solution}
		As Neptune is not visible to the naked eye, the ancient Indians did not have any known means to observe Neptune and thus did not know of its existence.
	\end{solution}

\filbreak
\question
	A region of the Eagle Nebula has a thickness of \SI{50}{\parsec}. The near edge of the nebula is \SI{6900}{\lightyear} from Earth.

	An A\num{1} subgiant was observed along the line of sight of the nebula, and spectroscopic analysis showed that the star had an absolute magnitude of \num{-0.01}. What is the difference in apparent magnitude if the star was behind the nebula instead of in front of it?

	You may assume a fixed interstellar extinction of 0.8 magnitude due to the nebula.

	\begin{oneparchoices}
		\choice	\num{0.77}
		\choice	\num{0.80}
		\choice	\num{0.83}
		\correctchoice	\num{0.85}
		\choice	\num{0.87}
	\end{oneparchoices}
	\begin{solution}
		Using the distance modulus equation found in the Formula Book, the difference between the apparent magnitudes in the 2 positions \textit{purely} due to distance is:
		\begin{flalign*}
		\SI{6900}{\lightyear} &= \frac{6900}{3.26} \approx \SI{2116}{\parsec} &\\
		\log\left({\frac{2116+50}{10}}\right) &\times 5 - \log\left({\frac{2116}{10}}\right) \times 5 = 0.05 &\\
		\text{Difference} &= \text{interstellar extinction} + 0.05 = 0.8 + 0.05 = 0.85
		\end{flalign*}
	\end{solution}

\filbreak
\question
	A planet with a strong magnetic field tends to collect and trap highly energetic and charged particles emitted by the Sun in belts around its equator. Jupiter’s radiation belts are particularly intense, and pose a major radiation hazard to any orbiting spacecraft or human.

	Which of the following methods is \textbf{least effective} in mitigating radiation damage caused by charged particles?
	\begin{choices}
		\choice	Orbiting the planet in an highly elliptical polar orbit.
		\choice	Storing food and water in the walls of the spacecraft.
		\correctchoice	Constructing a Faraday cage of metal wire mesh around the essential parts of the spacecraft.
		\choice	Constructing a lead box around essential parts of the spacecraft.
		\choice	All of the above methods are feasible.
	\end{choices}
	\begin{solution}
		A Faraday cage would neutralise the electric fields of electromagnetic waves from penetrating the interior, but would not actually stop charged particles, which are highly energetic, and hence have relativistic velocities. Thus, these particles would not be effectively stopped by a Faraday Cage --- shielding is required to slow these particles down to a stop.
	\end{solution}

\filbreak
\question
	Which of the following statements about large telescopes is correct?
	\begin{enumerate}[align=left,labelsep=0pt,leftmargin=0pt,labelsep=3pt,itemindent=23pt,parsep=6pt,label=\textbf{\Roman{*}}.]
		\item For large observatory-class telescopes, perfectly spherical optical elements are recommended to \hspace*{20pt} minimize aberrations.
		\item \textbf{All} space telescopes follow a Cassegrain-Nasmyth design.
		\item In recent decades, refractors are generally not seen among large telescopes. Instead, reflectors are \hspace*{20pt} more popular.
	\end{enumerate}

	\begin{oneparchoices}
		\choice \textbf{I} only
		\choice \textbf{I} and \textbf{II} only
		\choice \textbf{II} only
		\correctchoice \textbf{III} only
		\choice \textbf{I}, \textbf{II}, and \textbf{III} only
	\end{oneparchoices}
	\begin{solution}
		\textbf{I}: using spherical optical elements will introduce spherical aberration in images. Aspheric optical elements are used instead.

		\textbf{II}: On Earth, a Nasmyth focus is usually used for observatories as light is directed outwards along the altitude axis. This allows heavy instruments (e.g. spectrometers) to be used without moving along with the telescope and straining the bearings of the observatory. In space, however, the scope can move freely in zero-gravity, allowing heavy equipment to be installed directly at the Cassegrain focus and hence Cassegrain design is more commonly used.

		\textbf{III}: Reflectors are indeed more popular in the recent century as they grant greater focal length, and have less glass to be ground (1 mirror surface compared to 2 on refractors), which makes reflector telescopes relatively cheaper as aperture size increases.
	\end{solution}

\filbreak
\question
	Bob was trying to measure the distance from the Earth to the Large Magellanic Cloud, using Type Ia supernovae and Cepheid variables. He discovered that the result he obtained using Cepheid variables was much farther than the result from using Type Ia supernovae.

	Which of the following is \textbf{not} a possible reason for the discrepancy in his results?
	\begin{choices}
		\choice	He misidentified Type Ib/Ic supernovae as Type Ia supernovae.
		\choice	He underestimated the interstellar extinction when measuring the brightness of Cepheids.
		\choice	He mistook RR Lyrae variables for classical Cepheids.
		\correctchoice	He failed to account for atmospheric disturbances on the days of his observation.
		\choice	All of the above are possible reasons for the observed discrepancy in his results.
	\end{choices}
	\begin{solution}
		Any brightness variation due to atmospheric disturbances would be averaged out over the duration of observation.
	\end{solution}

\filbreak
\question
	`Oumuamua was discovered using the Pan-STARRS telescope at Haleakala Observatory, Hawaii, on 19 October 2017, 40 days after it passed its closest point to the Sun. At this point,  `Oumuamua was \SI{0.22}{\AU} away from Earth.

	The focal length of the Pan-STARRS telescope is \SI{6.11}{\metre} with an image sensor of diameter \SI{32}{\centi\metre}. It also has an aperture diameter of \SI{1.8}{\metre}.

	Determine the \textbf{mimimum} size of an object that the telescope could resolve at the distance between the Earth and `Oumuamua on 19 October 2017 , and the actual field of view (TFOV) of the telescope.

	\hspace{21pt} \textbf{Resolution (m)} \hspace{30pt} \textbf{TFOV ($ ^{\circ} $)}
	\begin{choices}
		\correctchoice \hspace{10pt} \num{1.4e4} \hspace{60pt} 3.0
		\choice \hspace{10pt} \num{2.8e4} \hspace{60pt} 1.5
		\choice \hspace{10pt} \num{1.4e4} \hspace{60pt} 1.5
		\choice \hspace{10pt} \num{2.8e4} \hspace{60pt} 6.0
		\choice \hspace{10pt} \num{5.6e4} \hspace{60pt} 3.0
	\end{choices}
	\begin{solution}
		Resolvable diameter can be calculated using Rayleigh’s Criterion $\frac{\text{diameter}}{\text{dist}} = 1.22\left(\frac{\lambda}{d}\right)$. Distance is mentioned to be at \SI{0.22}{\AU}. Hence, the value of diameter calculated would be \SI{13900}{\metre}.

		For the calculation of the TFOV of a telescope: the two ends of the TFOV of the sky will converge at focal length of the telescope after passing through the lens.
	\end{solution}

\filbreak
\question
	Which of the following are major factors contributing to the equation of time?
	\begin{enumerate}[align=left,labelsep=0pt,leftmargin=0pt,labelsep=3pt,itemindent=23pt,parsep=6pt,label=\textbf{\Roman{*}}.]
		\item The diameter of the Earth
		\item The orbital eccentricity of the Earth’s orbit
		\item The axial tilt of the Earth’s rotation
		\item The sidereal rotation period of the Earth
	\end{enumerate}
	\begin{choices}
		\choice	\textbf{I} and \textbf{II} only
		\choice	\textbf{I} and \textbf{III} only
		\correctchoice	\textbf{II} and \textbf{III} only
		\choice	\textbf{III} and \textbf{IV} only
		\choice	\textbf{II}, \textbf{III}, and \textbf{IV} only
	\end{choices}
	\begin{solution}
		The two major components are \textbf{II} and \textbf{III}. The analemma is a reflection of the extent of difference of the hour angle between the mean Sun and real Sun across a year.

		The orbital eccentricity of the Earth causes the angular velocity of the real Sun on the ecliptic to vary along its orbit, thus the real Sun appears to lag behind or speed ahead of the mean Sun across a year.

		As Earth’s axial tilt is non-zero, the projection vector of the real Sun onto the celestial equator (the angular velocity in right ascension) does not have constant magnitude. This introduces another deviation in the RA of the real Sun from the mean Sun.

		\textbf{I} and \textbf{IV} are \textbf{incorrect} since they are intrinsic factors of the Earth that do not affect the angular velocity (or its components) of the Sun along the ecliptic.
		Anyone who picks \circled{E} is not reading the question.
	\end{solution}

\filbreak
\question
	Which of the following statements are \textbf{true}?
	\begin{enumerate}[align=left,labelsep=0pt,leftmargin=0pt,labelsep=3pt,itemindent=23pt,parsep=6pt,label=\textbf{\Roman{*}}.]
		\item   The complete analemma may be photographed \textbf{anywhere} on Earth.
		\item	The analemma may be composed from photographs of the Sun taken everyday at exactly local \hspace*{20pt} apparent noon.
		\item	The Sun spends equal fractions of the year tracing out each lobe of the analemma.
		\item	The analemma photographed from the Equator has its long axis exactly perpendicular to the \hspace*{20pt} horizon.
	\end{enumerate}
	\begin{choices}
		\choice	\textbf{I} only
		\choice	\textbf{I} and \textbf{II}
		\choice	\textbf{II} and \textbf{III}
		\choice	\textbf{II} and \textbf{IV}
		\correctchoice	None of the above options are correct.
	\end{choices}
	\begin{solution}
		\textbf{I} is false. To photograph the complete analemma, the Sun must culminate above the horizon at local mean noon every day of the year. This is not true, say, for locations near the Earth’s geographic poles, which experience polar nights when the Sun does not rise above the horizon at all in a given day.

		\textbf{II} is false. Local apparent noon occurs when the apparent solar time (time told by a sundial) indicates the Sun has culminated on the local meridian. If the Sun is photographed exactly at local apparent noon everyday, the composed image will trace out a line segment along the meridian and not the analemma.

		\textbf{III} is false. The analemma is asymmetrical in shape (if you have seen photos). The physical reason for this is that the two components of the analemma (orbital eccentricity and axial tilt) are phase-shifted, hence the resultant sinusoidal wave does not have constant amplitude.

		\textbf{IV} is true. The long axis of the analemma always lies along the local meridian as these are the locations where the mean Sun and apparent real Sun coincide in a year. The projection of the analemma (from the real Sun) onto the local meridian traces out the culmination of the mean Sun across the year.
	\end{solution}

\filbreak
\question
	The hyperbolic excess velocity, $ v_\infty $, is defined as the velocity attained by a spacecraft as the spacecraft approaches infinity. The velocity at any given distance $ r $ from a gravitating body is given by $ v $ and the escape velocity from the gravitating body is $ v_e $.

	Which of the following expressions used to calculate the hyperbolic excess velocity of a gravitating body $ v_\infty $ is \textbf{correct}?
	\begin{choices}
		\choice $ v_\infty=v+v_e $
		\choice $ {v_\infty}^2=v^2+{v_e}^2 $
		\correctchoice $ v_\infty=\ \sqrt{(v-v_e)(v+v_e)} $
		\choice $\displaystyle {v_\infty}^2=v^2+\frac{{v_e}^2}{4} $
		\choice $\displaystyle v_\infty=\ \sqrt{\left(v-\frac{v_e}{2}\right)\left(v+\frac{v_e}{2}\right)} $
	\end{choices}
	\begin{solution}
		We can solve this by looking at the definitions involved.

		TE = KE(excess)

		TE = KE(excess) + escape velocity KE – GPE at the point

		By definition, a body at escape velocity has enough KE to overcome gravity, thus both terms sum to 0. We now recall that by energy conservation, TE is also given by current KE – current GPE, yielding:

		Current KE – GPE at the point = KE(excess) + escape velocity KE – GPE at the point

		Current KE = KE(excess) + escape velocity KE

		Dividing throughout yields:
		\begin{flalign*}
			v^2 &= {v_e}^2+{v_\infty}^2 &\\
			{v_\infty}^2 &=v^2-{v_e}^2=(v_e+v)(v-v_e)
		\end{flalign*}
		Square-rooting yields the desired answer.
	\end{solution}

\filbreak
\question
	Figure \ref{q26} shows the stellar evolution of a low-mass main sequence star such as our Sun.
	\begin{figure}[H]
		\centering
		\includegraphics[width=0.7\linewidth]{Graphics/Questions/HeliumFlash/heliumflash}
		\renewcommand{\figurename}{Figure}
		\caption{The evolutionary path of a star like the Sun}
		\label{q26}
	\end{figure}
	Which of the following nuclear equations best describes the dominant source of energy production during event 9?
	\begin{choices}
		\correctchoice	\ce{3 ^4_2He -> ^12_6C}
		\choice	\ce{2 ^1_1H -> ^2_1H}
		\choice	\ce{4 ^1_1H -> ^4_2He}
		\choice	\ce{^12_6C + ^1_1H -> ^13_7N}
		\choice	\ce{^15_7N + ^1_1H -> ^16_8O}
	\end{choices}
	\begin{solution}
		Event 9 is the helium flash, during which three helium-3 nuclei fuse to form one carbon-12 nucleus.
	\end{solution}

\filbreak
\question
	Manhattanhenge is an event during which the setting sun or the rising sun is aligned with the main street grid of Manhattan, New York City. It is known that Manhattanhenge sunsets occur in pairs annually.

	Given that a Manhattanhenge sunset occurred on 30 May 2019, on what date does the other Manhattanhenge sunset occur in 2019?

	\begin{oneparchoices}
		\choice	11 June
		\correctchoice	12 July
		\choice	30 August
		\choice	30 September
		\choice	11 October
	\end{oneparchoices}
	\begin{solution}
		Manhattanhenge occurs twice a year because on those dates, the sunset’s azimuth aligns with the direction of the street. This implies that Manhattenhenge sunsets are equally spaced around the summer solstice, and that the other Manhattenhenge sunset must happen \textbf{after} the summer solstice.

		Given that the summer solstice typically falls on 21 June, and 30 May is 20 days earlier, it follows that the next date is roughly 20 days after 21 June, which is 12th July.
	\end{solution}

\filbreak
\question
	Being an inquisitive astronomer, Keven wants to determine the \textbf{speed} of a star that is moving away from Earth, with respect to an observer from the Earth (i.e himself).

	Keven is provided with the following information:
	\begin{itemize}[leftmargin=10pt]
		\item The spectral line of iron from the star at $ \lambda_\text{rest} $ (wavelength recorded with respect to the frame of the star) is redshifted by a magnitude $ \Delta\lambda $.

		We can assume that $ \Delta\lambda \ll \lambda_\text{rest} $. The wavelengths are all recorded in metres.
		\item The proper motion is $ \mu $, recorded in arc seconds per year.
		\item The parallax of the star is given by $ p $, recorded in arc-seconds. We can assume that $ p $ is small enough such that $ \sin{p} \approx p $.
	\end{itemize}
	Which of the following expressions best represents the expression that Keven should use to determine the speed of the star relative to him, in \si{\km\per\s}?

	Hint: To ease your calculation, note that $\displaystyle \frac{\SI{1}{\AU}}{\SI{1}{\year}} $ is approximately \SI{4.74}{\km\per\s}.
	\begin{choices}
		\choice $\displaystyle \sqrt{\left(\frac{c\Delta\lambda}{\lambda_\text{rest} + \Delta\lambda}\right)^2 + \left(\frac{4740p}{\mu}\right)^2}$
		\choice $\displaystyle \sqrt{\left(\frac{c\Delta\lambda}{1000\lambda_\text{rest}}\right)^2 + \left(\frac{4.74\mu}{p}\right)^2}$
		\choice $\displaystyle \sqrt{\left(\frac{c\Delta\lambda}{\lambda_\text{rest}}\right)^2 + \left(\frac{4.74p}{\mu}\right)^2}$
		\correctchoice $\displaystyle \sqrt{\left(\frac{c\Delta\lambda}{1000\lambda_\text{rest}}\right)^2 + \left(\frac{4.74p}{\mu}\right)^2}$
		\choice $\displaystyle \sqrt{\left[\frac{c\Delta\lambda}{1000\left(\lambda_\text{rest} + \Delta\lambda\right)}\right]^2 + \left(\frac{4740\mu}{p}\right)^2}$
	\end{choices}
	\begin{solution}
		Recall that for relativistic redshift, $\displaystyle z=\frac{\Delta\lambda}{\lambda_\text{rest}} = \frac{v_r}{c} $,  where $ v_r $ represents radial velocity.

		Hence, it follows that $\displaystyle v_r = \frac{c\Delta\lambda}{\lambda\text{rest}}\si{\m\per\s}= \frac{c\Delta\lambda}{1000\lambda_\text{rest}} \si{\km\per\s} $ for the radial component of the velocity of the star.

		Next, for a given parallax angle, we know that $\displaystyle \tan{p} \approx p = \frac{\SI{1}{\AU}}{r} \implies r = \frac{\SI{1}{\AU}}{p} $

		Furthermore, the tangential velocity is given by $\displaystyle v_\theta = r\mu = \SI{1}{\AU} \times \frac{p}{\mu} = \frac{4.74p}{\mu}\si{\km\per\s} $

		Hence, the speed is given by $\displaystyle v = \sqrt{v_r^2+v_\theta^2} = \sqrt{\left(\frac{c\Delta\lambda}{1000\lambda_\text{rest}}\right)^2 + \left(\frac{4.74p}{\mu}\right)^2} $.
	\end{solution}

\filbreak
\question
	Some astronomers suggest that the search for extra-terrestrial life should include planets and moons outside of the circumstellar habitable zone. A possible reason is:
	\begin{choices}
		\choice Liquid water is the only possible solvent for biochemical reactions to occur in.
		\choice Many other celestial bodies are well-known sources of radio emissions.
		\correctchoice Besides radiation from a star, other mechanisms could generate heat on a celestial body, to provide energy for respiration.
		\choice All of the above.
		\choice None of the above; they simply have too much free time.
	\end{choices}
	\begin{solution}
		This question is testing the student’s attention to details based on how the statements are worded.

		\circled{A}	Methane (e.g. on Titan) is another likely solvent, and if \circled{A} was true, it would only be detrimental to the case.

		\circled{B}	 Missing the point; giving out radio does not equate to life or intelligent life, e.g. if the source is a pulsar.

		\circled{C}	True; Radioactive heating and tectonics due to gravitational forces are two such mechanisms.
	\end{solution}

\filbreak
\question
	Suppose that scientists are about to place a spherical rock of uniform mass \SI{5e11}{\kg} and radius \SI{1}{\km} in orbit around Jupiter. However, they would prefer it if the rock were as close as possible to Jupiter’s surface.

	Which of the following distances, from Jupiter's \textbf{surface} to that of the rock, is the most ideal?

	Assume that the rock is held together only by gravity, and that Jupiter is perfectly spherical.
	\begin{choices}
		\choice	\SI{126000000}{\km}
		\choice	\SI{12600000}{\km}
		\choice	\SI{1260000}{\km}
		\correctchoice	\SI{126000}{\km}
		\choice	It does not matter where the satellite is placed if it is not going to crash.
	\end{choices}
	\begin{solution}
		Density of body, $ \displaystyle \rho_b=\frac{\num{5e11}}{\frac{4}{3}\pi{1000}^3} = 500\times\frac{3}{4\pi} $

		Density of planet, $ \displaystyle
		\rho_p = \frac{\num{1.899e27}}{\frac{4}{3}\pi\left(\num{7.149e7}\right)^3} = 5197.43\times \frac{3}{4\pi} = \SI{1240}{\kg\per\metre\cubed} $

		Density ratio,
		$\displaystyle \frac{\rho_p}{\rho_b} = 10.395 $

		Roche limit $ = 1.26 R_J \times \left(10.395\right)^\slantfrac{1}{3} = 2.75 R_J$.
		Since we want distance from surface, we find $ 1.75 R_J \approx \SI{125000}{\km}$. Thus we can go as low as option D.
	\end{solution}

\filbreak
\question
	A possible reason why intelligent life might be rare, is that there exist factors that prevent its development or prematurely ends it. This is known as the Great Filter hypothesis. Which of the following are examples of such a ‘Great Filter’?
	\begin{enumerate}[align=left,labelsep=0pt,leftmargin=0pt,labelsep=3pt,itemindent=23pt,parsep=6pt,label=\textbf{\Roman{*}}.]
		\item Most, if not all other planets, do not have the necessary requirement(s) to support life in the first \hspace*{20pt} place (Rare Earth Hypothesis).
		\item It is highly improbable that inorganic molecules develop into living cells.
		\item Rational intelligent life is unwilling to communicate or transmit signals to space that indicate its \hspace*{20pt} existence, over fears of elimination by a more advanced alien civilisation.
		\item Many catastrophes result in large or even full-scale extinction events before intelligent life can \hspace*{20pt} develop, engineer, and build technology to prevent its demise.
		\item Intelligent life tends to be self-destructive in nature, whether intentional (e.g. wiping itself out by \hspace*{20pt} nuclear warfare) or unintentional (e.g. causing irreversible ecological collapse).
	\end{enumerate}
	\begin{choices}
		\choice \textbf{I}, \textbf{II}, and \textbf{III} only
		\choice \textbf{IV} and \textbf{V} only
		\choice \textbf{III}, \textbf{IV} and \textbf{V} only
		\correctchoice \textbf{I}, \textbf{II}, \textbf{IV}, and \textbf{V} only
		\choice All of the above
	\end{choices}
	\begin{solution}
		\textbf{I} is true, and Earth is the only known planet that has passed this filter. Planets in the habitable zone are candidates, but there could likely be some unknown factor (e.g. lack of water in the first place, hostile chemistry in its environment, etc.) that prevents life from forming.

		\textbf{II} is true, and again we only know Earth to be the only planet with such a development.

		\textbf{III} is tricky and false; Elimination by an advanced alien civilisation is a ‘Great Filter’ event, but being unwilling to communicate over this hypothetical fear doesn’t result in a filter per se.

		\textbf{IV} is true, and Earth is one of the lucky few (or the only one) which hasn’t experienced a full-scale extinction event while having recovered from many large-scale ones.

		\textbf{V} is true, and some argue that humanity is close to destroying itself.
	\end{solution}

\filbreak
\question
	Which of the following statements is \textbf{false}?
	\begin{choices}
		\choice A dark nebula that consists of mostly un-ionised hydrogen is likely designated as a HI region.
		\choice A nebula with a large HII region is highly likely to be a star-forming region.
		\choice There is no such thing as a HIII region, because hydrogen only has one valence electron.
		\choice High levels of OIII is responsible for green-cyan spectral colours in planetary nebulae.
		\correctchoice None of the above are false.
	\end{choices}
	\begin{solution}
		I, II, III refer to that element’s ionisation state. All of them are true statements, including D. That surprised many astronomers when it was initially discovered as ionised oxygen rarely occurs on Earth.
	\end{solution}

\filbreak
\question
	A physics student derived the following formula from the Stefan-Boltzmann law to estimate the radius, surface temperature, or luminosity of a given star, if any two of the three values were known, in comparison to the Sun:
	\begin{align*}
	\frac{R}{R_{\astrosun}} \approx \left(\frac{T_{\astrosun}}{T}\right)^2 \times \sqrt{\frac{L}{L_{\astrosun}}}
	\end{align*}
	where $ R $ is the radius of the star, $ R_{\astrosun} $ is the radius of the Sun, $ T $ is the surface temperature of the star, $ T_{\astrosun} $ is the surface temperature of the Sun, $ L $ is the luminosity of the star, and $ L_{\astrosun} $ is the luminosity of the Sun.

	He then investigated two different stars, $ X $ and $ Y $. Both stars have a surface temperature of about \SI{8000}{\K}, but $ X $ is 1000 times \textbf{less} luminous than the Sun, whereas Y is \num{100000} times \textbf{more} luminous.

	Suggest identities for $ X $ and $ Y $, given that their estimated radius is of the correct order of magnitude from this estimation formulae.

	\hspace{21pt} \textbf{Star \textit{X}} \tabto{6.3cm} \textbf{Star \textit{Y}}
	\begin{choices}
		\correctchoice white dwarf         \tabto{5.5cm} white supergiant
		\choice white dwarf         \tabto{5.5cm} main-sequence
		\choice main-sequence       \tabto{5.5cm} white supergiant
		\choice small main-sequence \tabto{5.5cm} large main-sequence
		\choice None of the above combinations are valid.
	\end{choices}
	\begin{solution}
		$ X $ is about 0.01 of the solar radius while $ Y $ is about 165 times larger; \circled{A} is the correct option. Refer to Procyon B and Deneb for approximate real-life counterparts.

		Maths isn’t actually needed to solve this question. Recall that for main-sequence stars, luminosity steadily increases with temperature. Thus we can straight away rule out that $ X $ is a main-sequence star: no main-sequence star can be both hotter yet less luminous than our sun. Similarly, the extreme luminosity for $ Y $ suggests it must either be a massive OB main-sequence star, or a supergiant. OB stars have much higher surface temperatures than \SI{8000}{\K}, ruling them out as a possibility.
	\end{solution}

\filbreak
\question
	Which of the following claims about the Solar System is \textbf{false}?
	\begin{choices}
		\choice Io, a moon of Jupiter, is subjected to extreme tidal heating. As such, it is the most volcanically active object in our Solar system, with several of its volcanoes being taller than Mount Everest on Earth.
		\choice Mercury has a very thin atmosphere (approximately \SI{1}{\nano\pascal}), but due to its extreme proximity to the sun, most of it is constantly lost from the solar wind, and it is shaped like a comet’s tail behind the planet.
		\choice Uranus, due to its axial tilt of about \ang{98}, ends up having its poles facing the Sun during a solstice. Yet, strangely, the temperature of its equator is higher than that of its poles.
		\choice \SI{50}{\km} above the deadly sulfuric acid clouds of Venus, there exists a zone at \SI{1}{\atm}, with an average temperature of \SI{25}{\degreeCelsius}, theoretically making it possible for humans to stay in floating cities on this otherwise inhospitable planet.
		\correctchoice Titan, a moon of Saturn, is the largest moon by radius in the Solar System. It is also the only other place in the solar system, besides Earth and Venus, to experience precipitation.
	\end{choices}
	\begin{solution}
		Ganymede is the largest moon by both mass and radius (Titan is second by both metrics), and the gas giants experience precipitation as well (e.g. metallic hydrogen rain in Jupiter, diamond rain on all four gas giants, etc).
	\end{solution}

\filbreak
\question
	Consider the effects of axial tilt and axial precession on the position of the celestial poles.

	Earth’s axial tilt varies from \ang{22.1} to \ang{24.5} over a period of \num{41000} years. The current tilt is \ang{23.44} and decreasing. Axial precession has a period of around \num{25800} years. By the year 20350, the North celestial pole will be very near Thuban (RA: 14 h 04 m; Dec: \ang{+64;22;}, J2000.0).

	Using the RA and Dec system of that date, what would be the declination of the Sun during the northern vernal equinox in the year 20350?

	\begin{oneparchoices}
		\choice	\ang{1}
		\choice	\ang{25}
		\correctchoice	\ang{0}
		\choice	\ang{22}
		\choice	\ang{24}
	\end{oneparchoices}
	\begin{solution}
		Given that we are using the RA/Dec of that date, none of the given information matters. By the definition of the equinox, the Sun lies on the celestial equator at the equinox of that date, thus declination of the Sun at then must be \ang{0}.
	\end{solution}

\filbreak
\begin{EnvUplevel}
	%\vspace{20pt}
	\textbf{Use the following information to answer questions \ref{q36} and \ref{q37}.}

	Consider a bi-elliptical transfer, a more general case of the Hohmann transfer. Figure \ref{q36f} illustrates this process.
	\begin{figure}[H]
		\centering
		\includegraphics[width=0.7\linewidth]{Graphics/Questions/Bielliptical/bielliptical}
		\renewcommand{\figurename}{Figure}
		\caption{A bi-elliptical transfer from Earth to Mars}
		\label{q36f}
	\end{figure}
	\begin{itemize}[leftmargin=10pt]
		\item Starting from an initial circular orbit of radius $ r_1 $, at the position indicated \circled{1}, a prograde burn puts the spacecraft on the first elliptical transfer.
		\item Next, at the position indicated \circled{2}, at a distance $ r_3 $ from the Sun, a second prograde burn puts the spacecraft on its second elliptical transfer
		\item Lastly, at the position indicated \circled{3}, when the final desired circular orbital radius of $ r_2 $ is reached, a \textbf{retrograde} burn circularizes the trajectory.
	\end{itemize}
	\vspace{-32pt}
\end{EnvUplevel}
\filbreak
\question\label{q36}
	Which of the following expressions represents the total transfer time of the entire bi-elliptical transfer process?

	Note: \textbf{total time} refers to the time taken to travel from position \circled{1} to \circled{3}.
	\begin{choices}
		\choice  $\displaystyle \frac{\pi}{2\sqrt{2GM}}\left[ \left(r_3-r_1\right)^\slantfrac{3}{2}+\left(r_3-r_2\right)^\slantfrac{3}{2}\right] $
		\choice $\displaystyle \frac{\pi}{2\sqrt{GM}}\left[ \left(r_3-r_1\right)^\slantfrac{3}{2}+\left(r_3-r_2\right)^\slantfrac{3}{2}\right] $
		\choice $\displaystyle \frac{\pi}{\sqrt{2GM}}\left(r_3\right)^\slantfrac{3}{2} $
		\choice $\displaystyle \frac{\pi}{2\sqrt{GM}}\left[ \left(r_1+r_2\right)^\slantfrac{3}{2}+\left(r_2+r_3\right)^\slantfrac{3}{2}\right] $
		\correctchoice  $\displaystyle \frac{\pi}{2\sqrt{2GM}}\left[ \left(r_1+r_3\right)^\slantfrac{3}{2}+\left(r_2+r_3\right)^\slantfrac{3}{2}\right] $
	\end{choices}
	\begin{solution}
		Recall from Kepler’s 3rd Law of Planetary Motion that $\displaystyle T=2\pi\sqrt{\frac{a^3}{GM}} $.
		Thus, the total time taken can be split into two components, time taken to go from position \circled{1} to \circled{2}, labelled as $ t_1 $, and time taken to go from position \circled{2} to \circled{3}, labelled as $ t_2 $. These are also half of the period of orbit for each of the ellipses. Hence, $\displaystyle t = \pi\sqrt{\frac{a^3}{GM}} $.

		Now, what is left is to determine exactly the semi-major axis of each ellipse.

		Notice that from \circled{1} to \circled{2}, the total distance is $ r_1 + r_3 $. Hence, the semi-major axis for the first ellipse is $ \frac{r_1+r_3}{2} $. Hence, $\displaystyle t_{1} = \pi\sqrt{\frac{\left(r_1+r_3\right)^3}{8GM}} $.

		Similarly, from 2 to 3, the total distance is $ r_2 + r_3 $. Hence, the semimajor axis for the second ellipse is $\displaystyle \frac{r_2 + r_3}{2} $. Hence,
		$\displaystyle t_2 = \pi\sqrt{\frac{\left(r_2+r_3\right)^3}{8GM}} $.

		Thus, the required expression is $\displaystyle t = t_1+t_2=
		\frac{\pi}{2\sqrt{2GM}}\left[ \left(r_1+r_2\right)^\frac{3}{2}+\left(r_1+r_3\right)^\frac{3}{2}\right] $
	\end{solution}

\filbreak
\question\label{q37}
	To measure the efficiency of fuel being used, we consider the quantity known as the \textit{delta-v}, or $ \Delta v $. Simply put, $ \Delta v $ is the \textbf{magnitude} of the difference between the velocity before and after a burn occurs.

	Which of the following expressions correctly represents the $ \Delta v $ quantity associated with the second burn?
	\begin{choices}
		\choice $\displaystyle \sqrt{\frac{2GM\left(r_2-r_1\right)}{r_3\left(r_1+r_3\right)}} $
		\choice $\displaystyle \sqrt{\frac{2GM\left(r_3-r_1\right)}{r_3\left(r_2+r_3\right)}} $
		\choice $\displaystyle \sqrt{\frac{2GM}{r_3}}\left(\sqrt{\frac{r_1}{r_1+r_3}}-\sqrt{\frac{r_2}{r_2+r_3}}\right) $
		\correctchoice $\displaystyle \sqrt{\frac{2GM}{r_3}}\left(\sqrt{\frac{r_2}{r_2+r_3}}-\sqrt{\frac{r_1}{r_1+r_3}}\right) $
		\choice 0
	\end{choices}
	\begin{solution}
		Recall from the vis-viva equation, that $\displaystyle v^2=GM\left(\frac{2}{r}-\frac{1}{a}\right) $. Hence, what we need to determine is the velocity before and after the 2nd retrograde burn.

		Note that such a quantity is non-zero, since it is a burn. Despite the spacecraft being at the same position $ r $, the orbital semi-major axis is now different, hence causing a difference in the $ \Delta v $ quantity. Note that $ a_i $ and $ a_f $ which refers to the semimajor axes of the initial and final ellipse are already calculated in 3.
		\begin{flalign*}
		v_i &= \sqrt{GM\left(\frac{2}{r_3}-\frac{1}{a_i}\right)} &\\
			&= \sqrt{GM\left(\frac{2}{r_3}-\frac{2}{r_1+r_3}\right)} &\\
			&= \sqrt{\frac{2GM\left(r_1\right)}{r_3\left(r_1+r_3\right)}}
		\end{flalign*}

		and
		\begin{flalign*}
			v_f &= \sqrt{GM\left(\frac{2}{r_3}-\frac{1}{a_f}\right)} &\\
			 	&= \sqrt{GM\left(\frac{2}{r_3}-\frac{2}{r_2+r_3}\right)} &\\
			 	&=\sqrt{\frac{2GM\left(r_2\right)}{r_3\left(r_2+r_3\right)}}
		\end{flalign*}

		Henceforth,
		\begin{flalign*}
		\Delta v &= \sqrt{\frac{2GM(r_{2})}{r_3}} - \sqrt{\frac{2GM(r_1)}{r_3(r_1 + r_3)}} &\\
		&= \sqrt{\frac{2GM}{r_3}}\left(\sqrt{\frac{r_2}{r_2 + r_3}} - \sqrt{\frac{r_1}{r_1 + r_3}}\right)
		\end{flalign*}

		Note that $\displaystyle \sqrt{\frac{r_2}{r_2+r_3}} - \sqrt{\frac{r_1}{r_1+r_3}} > 0 $ since $ r_2 > r_1 $ and we already demanded that $ \Delta v $ is a non-negative quantity (since it is the difference between two quantities).
	\end{solution}

\filbreak
\question
	With respect to a hypothetical observer on the Sun, approximately how often do the phases of the Moon as seen by the observer repeat?

	Disregard eclipses and occultations.

	\begin{oneparchoices}
		\choice	24 hours
		\choice	27.3 days
		\choice	29.5 days
		\choice	365.25 days
		\correctchoice	None of the preceding
	\end{oneparchoices}
	\begin{solution}
		As seen from the Sun, the phase of the Moon is always full.
	\end{solution}

\filbreak
\question
	According to the formula booklet, the formula for the Jeans length is
	\begin{align*}
		R_J = \sqrt{\frac{15k_{B}T}{4 \pi G\langle m\rangle\langle\rho\rangle}}.
	\end{align*}
	Here, the formula assumes that one is dealing with the collapse of a spherical nebula of uniform density.

	Which of the following statements about the Jeans length, $ R_J' $, is true if, instead of a uniform distribution, the mass of the nebula were to be concentrated on the surface of the spherical nebula, i.e. a spherical shell nebula?
	\begin{choices}
		\choice $ R_{J}' > R_J $
		\choice $ R_{J}' = R_J $
		\correctchoice $ R_J' < R_J $
		\choice There is no valid expression for $ R_J'$, since the quantity representing average density, $ \langle\rho\rangle $, is ill-defined.
		\choice The expressions in \circled{A} to \circled{C} are all possible, depending on the dominant element in the nebula.
	\end{choices}
	\begin{solution}
		To answer this question, we first consider the main criterion for the formation of a star from the collapse of a nebula---gravitational pressure dominates over thermal pressure.
		Instead of looking at the individual pressure terms, we can instead consider the total energy of the system $ E = U + \Omega $, where $ U $ refers to the kinetic energy of the particles in the nebula, and $ \Omega $ refers to the gravitational potential energy of the particles in the nebula.

		$ E < 0 \iff \text{gravitationally bound} \iff \text{collapse} $

		Hence, what we are comparing is just the $ \Omega $ needed to balance $ U $.

		For which the mass is concentrated on the surface, the gravitational potential energy is `higher' (more negative), $ \Omega' < \Omega $. Hence, $ U^\prime > U $. Physically, we need the nebula to contract more to generate greater thermal pressure to work against this higher gravitational potential energy (i.e. for $ U $ to be higher, the radius of the nebula at equilibrium would have to be smaller).  This means that $ R_{J}'< {R}_{J} $.

		\circled{D}: An ill-defined $ \rho $ does not mean that the equation loses its meaning, especially since we are considering $ \langle\rho\rangle $ (average density) instead of $ \rho $. Here, we are simply looking at the nebula's total mass divided by its total volume, which is obviously well-defined.

		\circled{E}: The main element in a nebula is probably the same (hydrogen). Even if two nebulae have different dominant element, we are asking for the expression of Jeans Length for which $ \langle m\rangle $ is kept constant.

		\textbf{NOTE:} For a spherical shell nebula, the expression for Jeans Length is:

		$\displaystyle R_J^\prime = \sqrt{\frac{9k_BT}{8\pi G\langle m\rangle\langle\rho\rangle}} < \sqrt{\frac{15k_BT}{4\pi G\langle m\rangle\langle\rho\rangle}}=R_J $
	\end{solution}

\filbreak
\question
	Which one of the following celestial objects is \textbf{not} fictional?

	\tabto{23pt}\parbox[t]{4cm}{\textbf{Object}}  \parbox[t]{\textwidth-4cm}{\textbf{Description}}
	\begin{choices}
		\choice \parbox[t]{4cm}{Tachyon Ring}     \parbox[t]{(\textwidth-4cm)-23pt}{A stream of relativistic particles travelling faster than the speed of light due to its high-speed orbit ($ > 0.95c $) around a high-velocity ($ > 0.95c $) neutron star.}
		\choice	\parbox[t]{4cm}{Tired Star}       \parbox[t]{(\textwidth-4cm)-23pt}{A special type of redshifted star discovered by Fritz Zwicky, its high surface gravity cause photons to lose energy over time, resulting in redshift of its emission spectra.}
		\correctchoice	\parbox[t]{4cm}{The Blazar}       \parbox[t]{(\textwidth-4cm)-23pt}{An active galactic nucleus from another galaxy that is emitting a relativistic jet of ionised particles at nearly the speed of light ($ > 0.95c $), with the jet facing the Earth.}
		\choice	\parbox[t]{4cm}{Type X Supernova} \parbox[t]{(\textwidth-4cm)-23pt}{A supernova resulting from a neutron star crossing the Chandrasekhar limit and exploding into a burst of high energy X-rays and gamma rays.}
		\choice	\parbox[t]{4cm}{Poison Nebula}    \parbox[t]{(\textwidth-4cm)-23pt}{A HII region in space consisting of ionised formaldehyde and cyanide and its derivatives, detectable by shifts in its hydrogen emission spectra as a result.}
	\end{choices}
	\begin{solution}
		\circled{A}	Yes, Tachyons are hypothetical for travelling faster than the speed of light – but they have yet to be observed. Also, one can’t simply sum up the speed of objects this way to obtain a speed greater than $ c $.

		\circled{B}	Tired \textit{light}.

		\circled{C}	Real.

		\circled{D}	The Chandrasekhar limit applies to white dwarfs for Type Ia; Type X doesn’t exist.

		\circled{E}	While formaldehyde and cyanide are indeed found in space, they are likely trace compounds compared to Hydrogen – and would not be responsible for changing the hydrogen emission spectra.
	\end{solution}

\filbreak
\question
	The upcoming James Webb Space Telescope (JWST) will observe primarily in infrared (\SI{1.3}{\micro\m}), while the Hubble Space Telescope (HST) observes primarily in visible light (\SI{500}{\nano\m}). Given that the JWST has the same angular resolution as the HST, approximately how much larger must the surface area of the aperture of JWST be compared to the HST, assuming both mirrors are perfect circles?

	\begin{oneparchoices}
		\choice $2.6\times$
		\correctchoice $6.8\times$
		\choice $3.2\times$
		\choice $8.4\times$
		\choice None of the preceding
	\end{oneparchoices}
	\begin{solution}
		Rayleigh Criterion + small angle approximation yield:
		$\displaystyle \Delta\theta_\text{min} = 1.22\frac{\lambda}{D} $

		Clearly, for both telescopes to have the same angular resolution, JWST needs to have a diameter that is $ \frac{1.3}{0.5} = 2.6\times $ larger. Since we want the surface area ratio, squaring 2.6 yields the answer, 6.8.
	\end{solution}

\filbreak
\question
	Which optical phenomenon is \textbf{not} caused by interplanetary dust?
	\begin{choices}
		\choice Zodiacal light
		\choice Gegenschein
		\correctchoice Airglow
		\choice False dawn
		\choice None of the above.
	\end{choices}
	\begin{solution}
		False dawn and zodiacal light are the same phenomenon, and are caused by scattering of sunlight by interplanetary dust. Gegenschein is caused by direct backscatter of light by interplanetary dust towards the antisolar point, which causes Gegenschein. Airglow is caused by recombination of ionized air molecules and cosmic ray ionization.
	\end{solution}

\filbreak
\begin{EnvUplevel}
	\vspace{20pt}
	\textbf{Use	Table \ref{q43.1} to answer questions \ref{q43}, \ref{q44}, and \ref{q45}.}
	\begin{figure}[H]
		\centering
		\begin{tabularx}{0.8\textwidth}{@{}lXlccr@{}}
			\toprule
			\textbf{Object} & \textbf{Type}    & \textbf{\textbf{\begin{tabular}[c]{@{}l@{}}Angular \\[-6pt] size (')\end{tabular}}} & \textbf{RA} & \textbf{Declination} & \textbf{\begin{tabular}[c]{@{}r@{}}Apparent \\[-6pt] magnitude\end{tabular}} \\ \midrule
			IC1805  & bright nebula    & 212  & 02 h 34 m & \ang{+61;31;} & 6.50         \\
			IC1848  & bright nebula    & 168  & 02 h 52 m & \ang{+60;29;} & 6.50         \\
			NGC7789 & open cluster     & 16.0 & 23 h 58 m & \ang{+56;48;} & 6.70         \\
			NGC3918 & planetary nebula & 0.15 & 11 h 51 m & \ang{-57;17;} & 8.50         \\
			NGC362  & globular cluster & 12.9 & 01 h 03 m & \ang{-70;44;} & 6.58         \\
			B68     & dark nebula      & 3.50 & 17 h 23 m & \ang{-23;50;} & NA           \\ \bottomrule
		\end{tabularx}
		\renewcommand{\figurename}{Table}
		\caption{Celestial objects and their optical data}
		\label{q43.1}
	\end{figure}
	\vspace{-32pt}
\end{EnvUplevel}
\filbreak
\question\label{q43}
	The Island of Coll in Scotland is a designated dark sky site with coordinates \ang{56;37;} N, \ang{06;32;} W. At local midnight of the autumnal equinox, which of the objects in Table \ref{q43.1} can be seen?

	Assume that the current sky follows the J2000.0 Epoch.
	\begin{choices}
		\correctchoice	IC1805, IC1848 and NGC7789 only
		\choice	IC1805, IC1848, NGC7789 and B68 only
		\choice	IC1805, IC1848, NGC362 and B68 only
		\choice	NGC3918 and NGC362 only
		\choice	NGC362, NGC3918, NGC7789 and B68 only
	\end{choices}
	\begin{solution}
		At autumnal equinox, the Sun has RA = 12 h. Thus, at local midnight of the autumnal equinox, the local sidereal time is 00 h.  Thus pick objects that have similar RA, bearing in mind declination considerations.
	\end{solution}

\filbreak
\question\label{q44}
	The Great Barrier Island in New Zealand is another such offshore island which is excellent for stargazing. Its coordinates are \ang{36;13;50.3} S,\ang{175;28;30.9} W.

	Given that at this location, NGC 362 is currently crossing the local meridian, which objects in Table \ref{q43.1} will \textbf{not} be seen crossing the local meridian within the next 6 hours?
	\begin{choices}
		\choice	NGC 3918 and B68 only
		\choice	IC 1805, IC 1848, and NGC 7789 only
		\choice	IC 1805, IC 1848, and B68 only
		\choice	IC 1805, IC 1848, NGC 7789, and NGC 3918 only
		\correctchoice	IC 1805, IC 1848, NGC 7789, NGC 3918, and B68 only
	\end{choices}
	\begin{solution}
		Given the RA of NGC 362 is 01 h 03 m, objects that have an RA of up to 07 h 03 m will cross the local meridian within the next 6 hours. The only two objects that have an RA within this range are totally invisible from this location, thus yielding \circled{E} as the answer.
	\end{solution}

\filbreak
\question\label{q45}
	From a certain location, NGC7789 is observed to be circumpolar. On New Year’s Day, NGC7789’s altitude from the ground was measured. The object reached its highest altitude at around 1801 hours local time at \ang{80.4} and its lowest at around 0600 hours local time at \ang{14.1}. What is the latitude of the location?
	\begin{choices}
		\choice	\ang{43} N
		\correctchoice	\ang{47} N
		\choice	\ang{43} S
		\choice	\ang{47} S
		\choice	\ang{47} E
	\end{choices}
	\begin{solution}
		Average the largest and lowest altitude to get the altitude of the north celestial pole. Thence, calculate the latitude accordingly.
	\end{solution}

\filbreak
\question
	Consider the equation for hydrostatic equilibrium in the Formula Book. Given that we have a star in such a hydrostatic equilibrium, which of the following statements is \textbf{false}?
	\begin{choices}
		\choice The pressure within a star always decreases as we move outwards from the core.
		\correctchoice For a star in hydrostatic equilibrium, $\displaystyle P = \frac{\rho_{r}GM_r}{r} $ at all points in the star.
		\choice Outside of the star where $\displaystyle \rho \approx 0 , \dod{P}{r} \approx 0 $
		\choice For a star to be in hydrostatic equilibrium, the local pressure gradient must equal the local gravitational acceleration multiplied by the local density at all points in the star.
		\choice An object does not need to be undergoing fusion in order to be in hydrostatic equilibrium.
	\end{choices}
	\begin{solution}
		\circled{A}: Observe that density, radius and mass are all positive: thus $ \dod{P}{r} < 0 $.

		\circled{C}: Plug in density = 0 into the equation to trivially obtain $ \dod{P}{r} = 0 $.

		\circled{D}: Note that $\displaystyle \frac{GM_r}{r^2} $ is the local gravitational acceleration, $ g $

		\circled{E}: Terms involving nuclear fusion do not appear at all. This makes sense: the Earth and the planets are in hydrostatic equilibrium, too!

		\circled{B} is false as density and mass are functions of $ R $, thus solving this differential equation is far more complicated than it seems...
	\end{solution}

\filbreak
\question
	While camping in a secluded island on 25 May 2019, Ryan noticed a meteor in the night sky. Which of the following would the meteor most certainly be associated with?
	\begin{choices}
		\choice	Eta Aquarids
		\choice	Orionids
		\choice	Geminids
		\choice	Anti-helion source
		\correctchoice	There is insufficient information to answer the question.
	\end{choices}
	\begin{solution}
		While it is \textit{likely} that the meteor is associated with the Eta Aquariids, it is also possible that the meteor is a sporadic (i.e. not associated with any meteor shower). Thus, to confirm the identity of a meteor, one would need to track its trajectory and see if it matches the associated radiant of the shower. This information is not given.
	\end{solution}

\filbreak
\question
	Compared to the Sun, Betelgeuse has nearly no hydrogen absorption lines. Instead, Betelgeuse displays numerous oxide absorption lines like \ce{TiO}. Why is this the case?
	\begin{choices}
		\correctchoice	Due to the lower surface temperature of Betelgeuse, most hydrogen atoms do not have electrons in the right energy level to absorb/re-emit photons.
		\choice	As Betelgeuse is a highly evolved star, its hydrogen shell has been ejected in the surrounding planetary nebula.
		\choice	Oxide molecules preferentially absorb radiation, thus suppressing the hydrogen absorption line.
		\choice	All hydrogen in Betelguese has been used up in the s-process, creating the raw elements for heavy metal oxides like TiO.
		\choice	Due to the high velocity of Betelgeuse, the hydrogen absorption lines are all Doppler-shifted into oxide absorption lines.
	\end{choices}
	\begin{solution}
		Ans: Betelgeuse is not a planetary nebula and is too massive to form one, so \circled{B} is false. \circled{C} is false as each compound/atom has its own unique set of absorption lines. For similar reasons \circled{E} is false. \circled{D} is false: the outer envelope does not undergo nuclear fusion at all.
	\end{solution}

\filbreak
\question
	Which of the following statements are possible information that can be gleaned from observing the light curve of an object?
	\begin{enumerate}[align=left,labelsep=0pt,leftmargin=0pt,labelsep=3pt,itemindent=23pt,parsep=6pt,label=\textbf{\Roman{*}}.]
		\item The rotation period of a comet
		\item Whether the object is an intrinsic variable star or an extrinsic variable star
		\item The age of a star
		\item The type of supernova that has occurred
	\end{enumerate}
	\begin{choices}
		\choice I, II, and III only
		\correctchoice I, II, and IV only
		\choice I, III, and IV only
		\choice II, III, and IV only
		\choice All of the above are possible.
	\end{choices}
	\begin{solution}
		\textbf{III} is false: generally more information is required (e.g. main-sequence fitting of an associated star cluster).
	\end{solution}

\filbreak
\question
	A supernova shines with a luminosity $ 10^{10} $ times that of the Sun. If such a supernova appears in our sky \textit{equally} as bright as Venus does when it approaches superior conjunction, how far away would the supernova be, in kiloparsecs?

	Assume that both Venus and Earth have perfectly circular orbits, and assume that no atmospheric heat from Venus is radiated into space. The albedo of Venus is $ 0.689 $.

	\begin{choices}
		\choice \SI{1.11e21}{\kilo\parsec}
		\choice \SI{2.07e14}{\kilo\parsec}
		\choice \SI{2610}{\kilo\parsec}
		\choice \SI{72}{\kilo\parsec}
		\correctchoice \SI{36}{\kilo\parsec}
	\end{choices}
	\begin{solution}
		Luminosity of supernova $ = 10^{10} \times L_{\astrosun} = 10^{10} \times \SI{3.846e26}{\W} = \SI{3.846e36}{\W}$

		Radiant flux at Venus,

		$ \Phi_{\venus} = \displaystyle{ \frac{L_{\astrosun}}{4\pi d_{\text{\venus}}^2} = \frac{\SI{3.846e26}{\W} }{4\pi \left(\SI{1.082e11}{\m}\right)^2 } = \SI{2614}{\W\per\m\squared} } $

		Reflected radiation of Venus,
		\begin{flalign*}
		L_{\venus} &= \Phi_{\venus} \times \text{albedo}_{\venus}\times \text{disc area of Venus} &\\
		&= \SI{2614}{\W\per\m\squared} \times 0.689 \times \pi\left(\SI{6.051e6}{\m}\right)^2 &\\
		&= \SI{2.072e17}{\W}
		\end{flalign*}
		Intensity of reflected radiation of Venus at superior conjunction to Earth,
		\begin{flalign*}
		I_{\venus} &= \frac{L_{\venus}}{4\pi \left(d_{\venus} + d_{\bigoplus}\right)^2} &\\
		&= \frac{\SI{2.072e17}{\W}}{4\pi \left(\SI{1.082e11}{\m} + \SI{1.496e11}{\m}\right)^2} &\\
		&= \SI{2.481e-7}{\W\per\m\squared} = I_{\text{supernova}}
		\end{flalign*}
		\begin{flalign*}
		\implies 4\pi d_{\text{supernova}}^2 &= \frac{L_{\text{supernova}}}{I_{\text{supernova}}} &\\
		\implies d_{\text{supernova}} &= \sqrt{\frac{L_{\text{supernova}}}{4\pi I_{\text{supernova}}}} &\\
		&= \sqrt{\frac{\SI{3.846e36}{\W}}{4\pi  \left(\SI{2.481e-7}{\W\per\m\squared}\right) }} &\\
		&= \SI{1.1107e21}{\m}
		\end{flalign*}
		Hence distance to supernova in parsecs

		$ = \displaystyle{ \frac{\SI{1.1107e21}{\m}}{\SI{3.2616}{\lightyear\per\parsec} \times \SI{9.4605e15}{\m\per\lightyear}} = 35 996 \approx \textbf{\SI{36}{\kilo\parsec}} } $
	\end{solution}
\end{questions}
\end{document}
