\documentclass[a4paper,11pt,draft]{exam}

%%%%%%%%%%%%%%%%%%%%%%%%%%%%%%%%%%%%%%%%%%%%%%%%%%%%%%%%%%%%%%%%%%%%%%%%%%%%%%%%%%%%
%
% PACKAGE IMPORTS %%%%%
%
%%%%%%%%%%%%%%%%%%%%%%%%%%%%%%%%%%%%%%%%%%%%%%%%%%%%%%%%%%%%%%%%%%%%%%%%%%%%%%%%%%%%

%%%%% SOME MISC IMPORTS TO DEFINE STUFF FIRST %%%%%
\usepackage{etoolbox}
\usepackage{pgfkeys}

%%%%% FONTS & SYMBOLS %%%%%
\usepackage[utf8]{inputenc}
\usepackage[T1]{fontenc}
\pgfkeys{
	/ac/.is family, /ac/.cd,
	font/.is choice,
	font/lm/.code={
			\usepackage{lmodern}
			\usepackage{amsfonts,amssymb}
			\providecommand{\tablining}{}
			\providecommand{\propold}{}
			% \renewcommand{\textssc}[1]{\textsc{#1}}
			% \newcommand{\boldsmallcaps}[1]{{\fontfamily{cmr}\textsc{\textbf{#1}}}}
			\usepackage{inconsolata}
		},
	font/mpro/.code={
			\usepackage{MnSymbol}
			\usepackage[
				minionint,
				lf,
				mathtabular,
				loosequotes,
				swash,
				opticals,
				footnotefigures]{MinionPro}
			\providecommand{\tablining}{\figureversion{tab}}
			\providecommand{\propold}{\figureversion{osf}}
			\newcommand{\boldsmallcaps}[1]{\textssc{\textbf{#1}}}
			\usepackage{inconsolata}
		},
	font/lmodern/.code=\pgfkeysalso{font/lm},
	font/minionpro/.code=\pgfkeysalso{font/mpro},
}
\providerobustcmd*{\setfont}[1]{\pgfqkeys{/ac}{#1}}
\usepackage{amsmath}
\usepackage{wasysym}
\usepackage{microtype}

%%%%% GEOMETRY, PAGE SETUP, SPACING, PARAGRAPHING %%%%%
\usepackage[margin=2.5cm,a4paper]{geometry}
\usepackage{titlesec}
\usepackage{multicol}
\usepackage{multirow}
\usepackage{parskip}
\usepackage{tabto}
\usepackage{pdflscape}
\usepackage{enumitem}
\usepackage{adjustbox}
\usepackage[super]{nth}

%%%%% SCIENCE FORMATTING %%%%%
\usepackage{physics}
\usepackage[
	arc-separator = \,,
	retain-explicit-plus,
	%inter-unit-product =\cdot,
	detect-weight=true,
	detect-family=true,
	range-phrase=--,
	range-units=single
]{siunitx}
\usepackage[version=4]{mhchem}
\usepackage[makeroom]{cancel}
\renewcommand{\CancelColor}{\color{red}}

%%%%% GRAPHICS, CAPTIONS, TABLES %%%%%
\usepackage[table,dvipsnames]{xcolor}
\usepackage{graphicx}
\usepackage{float}
\usepackage{tikz,tikz-3dplot}
\usepackage{pgfplots}
\usepackage{pdfpages}
\usepackage[
	justification=centering,
	labelfont={small,bf},
	font={small}
]{caption}
\usepackage{subcaption}
\usepackage{array}
\usepackage{tabularx}
\usepackage{booktabs}
\usetikzlibrary{
	calc,
	arrows,
	arrows.meta,
	positioning,
	decorations.pathreplacing,
	decorations.markings,
	decorations.text,
	calligraphy,
	pgfplots.dateplot
}
\pgfplotsset{compat=1.17}
\usepackage[outline]{contour}
\contourlength{1pt}
\newcommand*\circled[1]{
	\begin{tikzpicture}[baseline=(char.base)]
		\node[shape=circle, draw, minimum size=1.5em, inner sep=0pt, thick] (char) {#1};
	\end{tikzpicture}
}
\tikzset{
	mid arrow/.style={postaction={decorate,decoration={
							markings,
							mark=at position .15 with {\arrow[#1]{stealth}}
						}}},
	>=stealth
}

%%%%% REFERENCES AND LINKS %%%%%
\usepackage{hyperref}
\usepackage[noabbrev]{cleveref}

%%%%% MISCELLANEOUS %%%%%
\usepackage[useregional,calc]{datetime2}
\DTMlangsetup[en-GB]{ord=raise}

%%%%%%%%%%%%%%%%%%%%%%%%%%%%%%%%%%%%%%%%%%%%%%%%%%%%%%%%%%%%%%%%%%%%%%%%%%%%%%%%%%%%
%
% MATHS MACROS %%%%%
%
%%%%%%%%%%%%%%%%%%%%%%%%%%%%%%%%%%%%%%%%%%%%%%%%%%%%%%%%%%%%%%%%%%%%%%%%%%%%%%%%%%%%

\newcommand{\tomb}{\quad\blacksquare{}}

%%%%%%%%%%%%%%%%%%%%%%%%%%%%%%%%%%%%%%%%%%%%%%%%%%%%%%%%%%%%%%%%%%%%%%%%%%%%%%%%%%%%
%
% CREF AND HYPERREF SETUP %%%%%
%
%%%%%%%%%%%%%%%%%%%%%%%%%%%%%%%%%%%%%%%%%%%%%%%%%%%%%%%%%%%%%%%%%%%%%%%%%%%%%%%%%%%%

\crefdefaultlabelformat{#2\textbf{#1}#3}

\creflabelformat{equation}{#2\textbf{(#1)}#3}
\creflabelformat{figure}{#2\textbf{#1}#3}

\crefname{equation}{\textbf{equation}}{\textbf{equations}}
\Crefname{equation}{\textbf{Equation}}{\textbf{Equations}}
\crefname{figure}{\textbf{Figure}}{\textbf{Figures}}
\Crefname{figure}{\textbf{Figure}}{\textbf{Figures}}
\crefname{table}{\textbf{Table}}{\textbf{Tables}}
\Crefname{table}{\textbf{Table}}{\textbf{Tables}}
\crefname{appendix}{\textbf{Appendix}}{\textbf{Appendices}}
\Crefname{appendix}{\textbf{Appendix}}{\textbf{Appendices}}
\crefname{section}{\textbf{\S}}{\textbf{\S}}
\Crefname{section}{\textbf{\S}}{\textbf{\S}}
\crefname{algorithm}{\textbf{Algorithm}}{\textbf{Algorithms}}
\Crefname{algorithm}{\textbf{Algorithm}}{\textbf{Algorithms}}

%%%%% SPECIFIC TO EXAM CLASS%%%%%
\creflabelformat{question}{#2\textbf{#1}.#3}
\creflabelformat{partno}{(#2\textbf{#1}#3)}
\creflabelformat{subpart}{(#2\textbf{#1}#3)}

\crefname{question}{question}{questions}
\Crefname{question}{Question}{Questions}
\crefname{partno}{}{}
\Crefname{partno}{}{}
\crefname{subpart}{}{}
\Crefname{subpart}{}{}

\hypersetup{
	colorlinks   = true,            %Colours links instead of ugly boxes
	urlcolor     = NavyBlue,        %Colour for external hyperlinks
	linkcolor    = Magenta,         %Colour of internal links
	citecolor    = Aquamarine       %Colour of citations
}

%%%%%%%%%%%%%%%%%%%%%%%%%%%%%%%%%%%%%%%%%%%%%%%%%%%%%%%%%%%%%%%%%%%%%%%%%%%%%%%%%%%%
%
%%%%% EXAM CLASS SETUP %%%%%
%
%%%%%%%%%%%%%%%%%%%%%%%%%%%%%%%%%%%%%%%%%%%%%%%%%%%%%%%%%%%%%%%%%%%%%%%%%%%%%%%%%%%%

%%%%% CHOICES ON ONE PAGE %%%%%
\BeforeBeginEnvironment{choices}{\par\nopagebreak\minipage{\linewidth}}
\AfterEndEnvironment{choices}{\endminipage}

%%%%% QUESTION/CHOICE LABELS %%%%%
\renewcommand{\questionlabel}{\thequestion.\hfill}
\renewcommand{\subpartlabel}{(\thesubpart)}
\renewcommand{\choicelabel}{\circled{\thechoice}}

%%%%% POINTS FORMATTING %%%%%
\renewcommand{\questionshook}{
	\setlength{\rightpointsmargin}{1.75cm}
	\setlength{\itemsep}{30pt}
}

%%%%% QUESTION/PART/SUBPART INDENTATION %%%%%
\renewcommand{\partshook}{
	\renewcommand\makelabel[1]{\rlap{##1}\hss}
	% \setlength{\itemsep}{6pt}
}

\renewcommand{\subpartshook}{
	\renewcommand\makelabel[1]{\rlap{##1}\hss}
	% \setlength{\itemsep}{6pt}
}

\renewcommand{\choiceshook}{
	\setlength{\labelsep}{10pt}
	\settowidth{\leftmargin}{\circled{W}.\hspace{5pt}\hspace{0em}}
	\setlength{\itemsep}{10pt}
}

\renewcommand{\solutiontitle}{
	\noindent\textbf{Solution:}\par\noindent
}

%%%%% ONEPAR CHOICES SPREAD %%%%%
\patchcmd{\oneparchoices}{\penalty -50\hskip 1em plus 1em\relax}{\hfill}{}{}
\patchcmd{\oneparchoices}{\penalty -50\hskip 1em plus 1em\relax}{\hfill}{}{}

%%%%% SOLUTION ENVIRONMENT %%%%%
\SolutionEmphasis{\color{NavyBlue}}
\correctchoiceemphasis{\color{NavyBlue}\bfseries\boldmath}
\marksnotpoints{}
\pointsinrightmargin{}
\pointsdroppedatright{}
\pointformat{\bfseries\textbf[\themarginpoints]}

%%%%% REDEFINE COVER PAGINATION AS ARABIC %%%%%
\makeatletter
\renewenvironment{coverpages}{%
	\ifnum \value{numquestions}>0\relax
		\ClassError{exam}{%
			Coverpages cannot be used after questions have begun.\MessageBreak
		}{%
			All question, part, subpart, and subsubpart environments
			\MessageBreak
			must begin after the cover pages are complete.\MessageBreak
		}%
	\fi
	\@coverpagestrue
	% \pagenumbering{arabic}%
	\adj@hdht@ftht
	\thispagestyle{headandfoot}
}{%
	\clearpage
	% \setcounter{num@coverpages}{\value{page}}%
	% \addtocounter{num@coverpages}{-1}%
	% \pagenumbering{arabic}%
	% Bugfix, Version 2.307\beta, 2009/06/11:
	% We have to say \@coverpagesfalse before \adj@hdht@ftht
	% because we're still inside the group created by the
	% coverpages environment and we want to set the
	% extraheadheight and extrafootheight to the values correct
	% for the first non-cover page:
	\@coverpagesfalse
	\adj@hdht@ftht
}
\makeatother

%%%%% SHOW 'SOLUTIONS' IN TITLEPAGE IF ANSWERS PRINTED %%%%%
\providerobustcmd*{\printsolntitle}{
	\ifprintanswers{
		\vspace{20pt}
		\fbox{\fbox{\Huge{\textssc{\textbf{\textcolor{red}{solutions}}}}}}
		\vspace{20pt}
	}
	\fi{}
}
\providerobustcmd*{\envupspace}{\ifanswers{\vspace{20pt}}}

%%%%% REMOVE SPACES FROM EnvFullwidth command
\BeforeBeginEnvironment{EnvFullwidth}{\vspace{-10pt}}
\AfterEndEnvironment{EnvFullwidth}{\vspace{-35pt}}

%%%%%%%%%%%%%%%%%%%%%%%%%%%%%%%%%%%%%%%%%%%%%%%%%%%%%%%%%%%%%%%%%%%%%%%%%%%%%%%%%%%%
%
%%%%% SIUNITX SETUP %%%%%
%
%%%%%%%%%%%%%%%%%%%%%%%%%%%%%%%%%%%%%%%%%%%%%%%%%%%%%%%%%%%%%%%%%%%%%%%%%%%%%%%%%%%%

\DeclareSIUnit{\year}{y}
\DeclareSIUnit{\AU}{AU}
\DeclareSIUnit{\parsec}{pc}
\DeclareSIUnit{\lightyear}{ly}
\DeclareSIUnit{\earthmass}{\textit{M}_{\earth}}
\DeclareSIUnit{\jupitermass}{\textit{M}_{J}}
\DeclareSIUnit{\solarmass}{\textit{M}_{\astrosun}}
\DeclareSIUnit{\atm}{atm}

%%%%%%%%%%%%%%%%%%%%%%%%%%%%%%%%%%%%%%%%%%%%%%%%%%%%%%%%%%%%%%%%%%%%%%%%%%%%%%%%%%%%
%
%%%%% MISCELLANEOUS %%%%%
%
%%%%%%%%%%%%%%%%%%%%%%%%%%%%%%%%%%%%%%%%%%%%%%%%%%%%%%%%%%%%%%%%%%%%%%%%%%%%%%%%%%%%

%%%%% SET NUMBERED AND BULLETED LIST MARGIN
\setlist[itemize, 1]{left=0pt}
\setlist[enumerate, 1]{left=0pt,label=\arabic*.}

%%%%% PURPOSE FORGOTTEN %%%%%
\AtBeginEnvironment{tabularx}{
	\tablining
	\sisetup{text-rm={\tablining}}
}

\renewcommand\tabularxcolumn[1]{m{#1}}

\titlelabel{\vspace{-4cm}\thetitle\hspace{10pt}}
\setlength{\jot}{8pt}

%%%%% COPY-PASTABLE PDF %%%%%
\input{glyphtounicode}
\pdfgentounicode=1

%%%%% MICROTYPE SETUP FOR HYPHENATION %%%%%
\pretolerance=2500
\tolerance=4500
\emergencystretch=0pt
\righthyphenmin=4
\lefthyphenmin=4

%%%%% DEFAULT CENTRED FIGURES %%%%%
\AtBeginEnvironment{figure}{\centering}

%%%%% RADIO BUTTONS %%%%%
\makeatletter
\providerobustcmd*{\radiobutton}{%
	\@ifstar{\@radiobutton0}{\@radiobutton1}%
}
\providerobustcmd*{\@radiobutton}[1]{%
	\begin{tikzpicture}
		\pgfmathsetlengthmacro\radius{height("X")/2}
		\draw[radius=\radius] circle;
		\ifcase#1 \fill[radius=.6*\radius] circle;\fi
	\end{tikzpicture}
}
\makeatother

%%%%%%%%%%%%%%%%%%%%%%%%%%%%%%%%%%%%%%%%%%%%%%%%%%%%%%%%%%%%%%%%%%%%%%%%%%%%%%%%%%%%
%
% ASTROCHALLENGE SETUP AND MACROS %%%%%
%
%%%%%%%%%%%%%%%%%%%%%%%%%%%%%%%%%%%%%%%%%%%%%%%%%%%%%%%%%%%%%%%%%%%%%%%%%%%%%%%%%%%%

%%%%% ASTROCHALLENGE PAPER DATE %%%%%
\providerobustcmd*{\setacpaperdate}[1]{\DTMsavedate{theacpaperdate}{#1}}
\providerobustcmd*{\acpaperdate}{\DTMusedate{theacpaperdate}}

%%%%% ASTROCHALLENGE SMALL CAPS %%%%%
\providecommand{\astrochallenge}{\textbf{\textssc{AstroChallenge \propold\DTMfetchyear{theacpaperdate}}}}

%%%%% AUTOMATIC CATEGORY AND ROUND WITH KEY-VALUE SYSTEM %%%%%
\providebool{ismcq}
\providebool{isteam}
\providebool{isobs}
\providebool{isdaq}

\pgfkeys{
	/ac/.is family, /ac/.cd,
	category/.is choice,
	round/.is choice,
	% DEFINE ACTIONS FOR CATEGORY
	/ac/category/.cd,
	jnr/.code={\edef\category{Junior}},
	snr/.code={\edef\category{Senior}},
	junior/.code=\pgfkeysalso{category/jnr},
	senior/.code=\pgfkeysalso{category/snr},
	% DEFINE ACTIONS FOR CHOICES
	/ac/round/.cd,
	mcq/.code={\edef\round{MCQ}\booltrue{ismcq}},
	team/.code={\edef\round{Team}\booltrue{isteam}},
	obs/.code={\edef\round{Observation}\booltrue{isobs}},
	daq/.code={\edef\round{Data Analysis}\booltrue{isdaq}},
	% ALTERNATIVE NAMES
	MCQ/.code=\pgfkeysalso{round/mcq},
	Team/.code=\pgfkeysalso{round/team},
	Obs/.code=\pgfkeysalso{round/obs},
	DAQ/.code=\pgfkeysalso{round/daq},
}
\providecommand*{\setcatround}[1]{\pgfqkeys{/ac}{#1}}
\providecommand*{\catround}{\textbf{\textssc{\category{} \round{} Round}}}

%%%%% ASTROCHALLENGE TITLING %%%%%
\title{{\Huge\astrochallenge{}}}
\author{\textcopyright \ National University of Singapore Astronomical Society \\
	\textcopyright \ Nanyang Technological University Astronomical Society \\
}

%%%%% ACCESS TITLE COMMANDS %%%%%
\makeatletter
\let\newtitle\@title
\let\newauthor\@author
\makeatother

%%%%% EXAM HEADER/FOOTER SETUP %%%%%
\coverheader{\small{\astrochallenge{}}}{}{\small{\catround{}}}
\header{\small{\astrochallenge{}}}{}{\small{\catround{}}}
\headrule{}

%%%%% PAGE NUMBERING: 'Page X of total' %%%%%
\coverfooter{}{Page \thepage{} of \totalnumpages{}}{\oddeven{\textbf{[Turn over]}}}
\footer{}{Page \thepage{} of \totalnumpages{}}{
	\oddeven{
		\iflastpage{\textbf{[End of paper]}}{\textbf{[Turn over]}}
	}
}

%%%%% ASTROCHALLENGE INSTRUCTIONS: MCQ %%%%%
\providerobustcmd*{\acmcqinst}{
	\begin{enumerate}[itemsep=8pt]
		\item This paper consists of \textbf{\totalnumpages} printed pages, including this cover page.
		\item Do \textbf{NOT} turn over this page until instructed to do so.
		\item You have \textbf{2 hours} to attempt all questions in this paper. If you think there is more than one correct answer, choose the most correct answer.
		\item At the end of the paper, submit this booklet together with your answer script.
		\item Your answer script should clearly indicate your name, school, and team.
		\item It is \textit{your} responsibility to ensure that your answer script has been submitted.
	\end{enumerate}
}

%%%%% ASTROCHALLENGE INSTRUCTIONS: TEAM %%%%%
\providerobustcmd*{\acteaminst}{
	\begin{enumerate}[itemsep=8pt]
		\item This paper consists of \textbf{\totalnumpages} printed pages, including this cover page.
		\item Do \textbf{NOT} turn over this page until instructed to do so.
		\item You have \textbf{2 hours} to attempt all questions in this paper.
		\item At the end of the paper, submit this booklet together with your answer script.
		\item Your answer script should clearly indicate your name, school, and team.
		\item It is your responsibility to ensure that your answer script has been submitted.
		\item The marks for each question are given in brackets in the right margin, like such: \textbf{[2]}.
		\item The \textbf{alphabetical} parts (i) and (l) have been intentionally skipped, to avoid confusion with the Roman numbering of (i).
	\end{enumerate}
}

%%%%% ASTROCHALLENGE INSTRUCTIONS: DAQ %%%%%
\providerobustcmd*{\acdaqinst}{
	In this part of \astrochallenge{}, you will work with a moderately large (approx. \num{4000} points) data set. You will process this data set, analyse it, observe trends, and draw conclusions. \textbf{There are no right or wrong answers}; you will be marked solely on the quality of your analysis, even if your statistical methods are incorrect.\\[8pt]
	We \textbf{strongly} recommend you use industry-standard tools like \texttt{Microsoft Excel}\texttrademark, \texttt{RStudio} or various \texttt{Python} libraries to process the data.\par
}

%%%%% ASTROCHALLENGE INSTRUCTIONS: OBS %%%%%
\providerobustcmd*{\acobsinst}{
	\begin{enumerate}[itemsep=8pt]
		\item This paper consists of \textbf{\totalnumpages} printed pages, including this cover page.
		\item Do \textbf{NOT} turn over this page until instructed to do so.
		\item You have \textbf{1 hour and 30 minutes} to attempt all questions in this paper.
		\item At the end of the paper, submit this booklet together with your answer script.
		\item Your answer script should clearly indicate your name, school, and team.
		\item It is your responsibility to ensure that your answer script has been submitted.
	\end{enumerate}
}

%%%%% ASTROCHALLENGE INSTRUCTIONS %%%%%
\providerobustcmd*{\acinstructions}{
	\ifbool{ismcq}{\acmcqinst}{
		\ifbool{isteam}{\acteaminst}{
			\ifbool{isdaq}{\acdaqinst}{
				\ifbool{isobs}{\acobsinst}{}
			}
		}
	}
}

%%%%% ASTROCHALLENGE INSTRUCTION BOX %%%%%
\providerobustcmd*{\acinstbox}{
	\vspace*{20pt}
	\fbox{
		\fbox{
			\parbox{0.8\textwidth}{
				\vspace*{5pt}
				\begin{center}
					{\large{\textbf{\textssc{please read these instructions carefully.}}}}
				\end{center}
				\vspace*{10pt}

				\acinstructions{}

				\vspace*{5pt}
			}
		}
	}
}

%%%%% ASTROCHALLENGE COVER PAGE %%%%%
\providerobustcmd*{\accoverpage}{
	\begin{coverpages}
		\begin{center}
			\includegraphics[width=0.8\linewidth]{../graphics/misc/logo.jpg}

			\newtitle{}

			{\Huge{\catround{}}}

			\printsolntitle{}

			\acpaperdate{}

			\acinstbox{}

			\vspace*{20pt}

			\newauthor{}
		\end{center}
	\end{coverpages}
}


\coverfirstpageheader{\footnotesize{\textsc{\textbf{AstroChallenge \figureversion{osf}2019}}}}{}{\footnotesize{\textsc{\textbf{Junior Team Round}}}}
\firstpageheader{\footnotesize{\textsc{\textbf{\figureversion{osf}AstroChallenge 2019}}}}{}{\footnotesize{\textsc{\textbf{Junior Team Round}}}}
\runningheader{\footnotesize{\textsc{\textbf{AstroChallenge \figureversion{osf}2019}}}}{}{\footnotesize{\textsc{\textbf{Junior Team Round}}}}

\renewcommand{\thequestion}{\alph{question}}
\renewcommand{\questionlabel}{(\thequestion)}
\renewcommand{\thepartno}{\roman{partno}}

\renewcommand{\questionshook}{%
	\settowidth{\leftmargin}{-1pt}%
	\setlength{\rightpointsmargin}{1.75cm}%
	\setlength{\itemsep}{10pt}%
	\setlength{\parsep}{10pt}%
}

\begin{document}
\begin{coverpages}
	\begin{center}
		\includegraphics[width=0.8\linewidth]{Graphics/Misc/logo.jpg}
	
		\newtitle
		
		{\Huge{\textssc{\textbf{Junior Team Round}}}}
		
		%\printsolns
		
		{\Large{Monday, \nth{3} June 2019}}
		
		\vspace*{20pt}
		\fbox{
			\fbox{
				\parbox{0.9\textwidth}{
					\vspace*{5pt}
					\begin{center}
						{\Large{\textsc{\textbf{Please read these instructions carefully.}}}}
					\end{center}
					\vspace*{10pt}
					
					\begin{enumerate}[itemsep=8pt]
						\item This paper consists of \textbf{\totalnumpages} printed pages, including this cover page.
						\item Do \textbf{NOT} turn over this page until instructed to do so.
						\item You have \textbf{2 hours} to attempt all questions in this paper. 
						\item At the end of the paper, submit this booklet together with your answer script.
						\item Your answer script should clearly indicate your name, school, and team.
						\item It is your responsibility to ensure that your answer script has been submitted.
						\item The marks for each question are given in brackets in the right margin, like such: \textbf{[2]}.
						\item The \textbf{alphabetical} parts (i) and (l) have been intentionally skipped, to avoid confusion with the Roman numbering of (i).
					\end{enumerate}
					\vspace*{5pt}
				}
			}
		}
	\end{center}
\pagenumbering{arabic}                     
\end{coverpages}
\newpage
\section{Geminids, Galaxies, and Gear}
\begin{questions}
\question[3]
	The Geminids are a meteor shower that is active in December. Copy the constellation diagram in Figure \ref{q1a} and the RA/Dec lines onto your answer paper. Draw the radiant of the meteor shower, and use arrows to delineate the direction of travel of the meteors.
	\droppoints
	
	\begin{figure}[H]
		\centering
		\includegraphics[width=0.75\linewidth]{Graphics/Questions/Geminids/Gemini}
		\renewcommand{\figurename}{Figure}
		\caption{The constellation Gemini\protect\footnotemark}
		\label{q1a}
	\end{figure}
	\footnotetext{\href{https://www.iau.org/static/public/constellations/pdf/GEM.pdf}{IAU chart of Gemini}}
	\begin{solution}
		Arrows pointing outwards in all directions. Because meteor shower particles are all travelling in parallel paths, at approximately the same velocity, they will all appear to radiate from a single point in the sky to an observer on the Earth below.
		
		\textbf{NOTE}: This solution has been modified from the original to include the precise radiant (RA: 07h 28m, Dec: \ang{+32}) of the Geminids. It is left to  the marker's discretion as to how (s)he wishes to award/deduct marks for radiants that deviate from the original.
		
		\begin{figure}[H]
			\centering
			\begin{tikzpicture}[>=Stealth]
				\node[anchor=south west,inner sep=0] (image) at (0,0) {\includegraphics[width=0.8\linewidth]{Graphics/Questions/Geminids/Gemini}};
				\begin{scope}[x={(image.south east)},y={(image.north west)}]	
					\draw[draw=red,fill=red] (0.35,0.77) circle[radius=3pt,label=right:radiant] node(radiant)[label={[label distance=25pt]right:\footnotesize \textbf{radiant and direction}}]  {};
					
					\foreach \angle in {0,30,60,...,359}
					\draw[->,below,,rotate around={\angle:(radiant.center)}] ($ (radiant.center) + (0,0.015) $) -- ($ (radiant.center) + (0,0.1) $);
				\end{scope}				
			\end{tikzpicture}
		\end{figure}
	\end{solution}
	
\question
	The Geminids shower is spectacular when viewed from the Northern hemisphere. However, while it is still visible in the Southern hemisphere, the further South one is, the lower the rate of meteors observed. Account for this observation in terms of:
	\begin{parts}
		\part[2] latitude, and
		\droppoints
		
		\part[2] elevation. \textbf{HINT}: the Earth is round.
		\droppoints
	\end{parts}

\begin{solution}
	The Geminids have a declination of \ang{+32}. Thus, observers in the Northern hemisphere will see higher rates as the radiant is higher in the sky and above the horizon. In the Southern hemisphere and before midnight, the radiant point is at or below the horizon and few meteors will be observed (if at all) because the atmosphere shields the Earth from most of the debris, and only those meteors which happen to be travelling exactly (or very nearly) tangential to the Earth's surface can be observed. 
	
	The radiant only rises above the horizon after midnight in the Southern hemisphere, and hence there is limited time for observation before sunrise.
\end{solution}
	
\question[3]
	List \textbf{three} characteristics that differentiate spiral and elliptical galaxies.
	\droppoints
	\begin{solution}
		Choose any three below.
		
		For \textit{spiral} galaxies:
		\begin{itemize}
			\item flat, rotating disk containing stars, gas and dust, and a central concentration of stars known as the bulge. 
			\item Presence of ongoing star formation, especially within the spiral arms.
			\item Often surrounded by a much fainter halo of stars, many of which reside in globular clusters. 
			\item Spiral structures that extend from the centre into the galactic disc. 
		\end{itemize}
		For \textit{elliptical galaxies}:
		\begin{itemize}
			\item having an approximately ellipsoidal shape and a smooth, nearly featureless brightness profile 
			\item more three-dimensional, without much structure
			\item Stars are in somewhat random orbits around the centre.
			\item Without significant interstellar gas/dust, and thus have negligible star formation rates.
		\end{itemize}
	\end{solution}
	
\question[4]
	Hence, or otherwise, identify and explain \textbf{two} visually observable features of the Milky Way that results in the conclusion that it is a spiral galaxy, rather than an elliptical galaxy.
	\droppoints
	\begin{solution}
		\begin{itemize}
			\item From the Earth, the Milky Way presents a long, thin strip of stars. This suggests a disk as seen edge-on from within the disk, rather than an ellipsoid or other irregular shape.
			\item There is also a clear bulge at the centre, which is typical of spirals.
			\item The presence of emission nebulae containing young stars strongly suggest the Milky Way is a spiral rather than an elliptical galaxy.
		\end{itemize}
	\end{solution}

\filbreak	
\question
	Before operating a pair of binoculars, one should know and understand its technical specifications and optical performance. Table \ref{q1e} lists some data on a particular pair of binoculars:
	\begin{figure}[H]
		\centering
		\begin{tabularx}{0.75\textwidth}{@{}Xl@{}}
			\toprule
			\textbf{Label}           & \textbf{Data}            \\ \midrule
			Brand                    & Celestron                \\
			Model                    & Skymaster Pro $ 20\times 80 $\\
			Linear field of view     & \SI{56}{\m}/\SI{1000}{\m}\\
			Dioptre adjustment range & $ \pm 4 $                \\
			Twilight factor          & 40                       \\
			Prism type               & BAK-4                    \\
			Nitrogen-filled?         & Yes                      \\
			Eye relief               & \SI{15.5}{\milli\m}      \\
			Mass                     & \SI{2449}{\g}            \\
			Relative brightness      & 16                       \\
			Inter-pupillary distance & 56-72mm                  \\ \bottomrule
		\end{tabularx}
		\renewcommand{\figurename}{Table}
		\caption{Optical and technical data of a pair of Celestron Skymaster Pro binoculars}
		\label{q1e}
	\end{figure}
	Using the information in Table \ref{q1e}, write down the:
	\begin{parts}
		\part[1] magnification, and
		\droppoints
		\part[1] aperture size of the pair of binoculars.
		\droppoints
	\end{parts}
	\begin{solution}
		Magnification: $ 20\times $, aperture: \SI{80}{\milli\metre}
	\end{solution}
	
\question[2]
	List \textbf{two} advantages that a pair of binoculars has over a telescope.
	\droppoints
	\begin{solution}
		Any two from the list below:
		\begin{itemize}
			\item Telescopes provide images that are upside down while binoculars do not have this problem.
			\item Binoculars show a larger area of the sky than telescopes, making it easier to star hop.
			\item Binoculars are easier to operate than telescopes.
			\item Binoculars are cheaper than telescopes.
			\item $7 \times 50$ binoculars gather more light than a \SI{50}{\milli\metre} telescope.
		\end{itemize}
	\end{solution}
	
\question[2]
	Explain how a telescope enables human observers to perceive celestial objects that are typically invisible to the unaided eye.
	\droppoints
	\begin{solution}
		A telescope can collect more light so that it can create a brighter image for our eyes. It can magnify the image so that it takes up more space on our retina.
		
		The big lens in the telescope (objective lens) collects much more light than our eye can from a distant object and focuses the light to a point (the focal point) inside the telescope.
		
		A smaller lens (eyepiece lens) takes the bright light from the focal point and magnifies it so that it uses more of your retina.
	\end{solution}
	
\end{questions}

\newpage
\section{Triple Trouble in Trifid}
Greetings, young astronomers. Three trials await you: the Red Door, the Blue Door, and the Dark Corridor. You choose which you will enter first, because you must pass all of them to escape... the Triple Trouble in the Trifid Nebula.
\subsection*{A Nebulous Problem (The Red Door --- The straightforward part)}
There are three main types of non-planetary nebulae, which might appear red, blue, or black, when photographed. We refer to them as \textit{emission}, \textit{reflection}, and \textit{dark} nebulae respectively. The Trifid Nebula is unique because it contains all three. Now, recall what you have learnt prior to AstroChallenge, and proceed with ease (hopefully). 

\begin{figure}[H]
	\centering
	\begin{subfigure}{0.4\textwidth}
		\centering
		\includegraphics[width=\textwidth]{Graphics/Questions/Trifid/Trifid_Wide.jpg}
		\renewcommand{\figurename}{Figure}
		\caption{A wide view of the Trifid Nebula}
	\end{subfigure}
	\hspace{1.5cm}
	\begin{subfigure}{0.4\textwidth}
		\begin{tikzpicture}
		\contourlength{0.3pt}
		\node[anchor=south west,inner sep=0] (image) at (0,0) {\includegraphics[width=\textwidth]{Graphics/Questions/Trifid/Trifid_Zoom.jpg}};
		\begin{scope}[x={(image.south east)},y={(image.north west)}]
		
		\draw[fill=red] (0.6,0.85) circle[radius=2.5pt] node[above,fill=white,rounded corners=2.5pt,inner sep=1.5pt,outer sep=5pt] {\textbf{TC1}};
		\draw[fill=green] (0.4,0.25) circle[radius=2.5pt] node[left,fill=white,rounded corners=2.5pt,inner sep=1.5pt,outer sep=5pt] {\textbf{TC2}};
		\end{scope}
		\end{tikzpicture}
		\renewcommand{\figurename}{Figure}
		\caption{A zoomed-in view of two dense cores, TC1 and TC2, that will be discussed later in this question}
	\end{subfigure}
	\renewcommand{\figurename}{Figure}
	\caption{Two views of the Trifid Nebula.}
\end{figure}
\begin{questions}
\question[3]
	Differentiate between \textit{emission}, \textit{reflection}, and \textit{dark nebulae}, giving reasons for their respective colours.
	\droppoints
	\begin{solution}
		\begin{enumerate}[leftmargin=12pt]
			\item \textbf{Emission}: Ionised hydrogen/HII region has a peak emission in the visible spectrum that is red \textbf{[1]}.
			\item \textbf{Reflection}: Interstellar dust that scatters light (Accept: Rayleigh scattering) from nearby stars; same reason the sky is blue \textbf{[1]}.
			\item \textbf{Dark}: \textit{Dense} interstellar dust that \textit{completely} blocks light/is \textit{opaque} \textbf{[1]}.
		\end{enumerate}
		Partial marks are awarded if the correct idea is present but reasoning is incomplete, e.g. `An emission nebula is made up of energetic particles that emit red light'.
	\end{solution}

\filbreak	
\question[1]
	The dust lanes across the Trifid Nebula obscure many interesting stars. Before you can open the Red Door to enter the Dark Corridor, you will have to answer a simple question.
	
	Suggest how astronomers could study stars that are obscured behind a dense, dark nebula.
	\droppoints
	\begin{solution}
		Observe them using equipment that detects electromagnetic waves beyond the visible spectrum (full credit is also awarded if a single example such as IR, X-ray, radio, etc. is given).
	\end{solution}
	
\question[2]
	Not all parts of the dust lane are equal. Some of them form dense cores, and these regions tend to contain objects that astronomers call protostars.
	
	What is a protostar, and why are astronomers interested in studying them?
	\droppoints
	\begin{solution}
		\textbf{Protostar}: A very young star that is still gathering material from a parent molecular cloud/ still possesses an accretion disk of materials \textbf{[1]};  
		Accept: yet to undergo nuclear fusion.
		
		\textbf{Interest}: Provide models for the early Solar System’s formation/insights into exoplanet formation and stellar models \textbf{[1]}. ‘Scientific curiosity’/reasonable layman answers \textbf{[0.5]}
	\end{solution}
	
\uplevel{
	\vspace{-10pt}
	\subsection*{A Stellar Nursery in the Nebula (The Dark Corridor --- Long and Winding)}
	Congratulations! You have passed the Red Door of basic astronomy knowledge. You have now entered the Dark Corridor, and you must make sense of the data presented to you. Without further ado, let us look at two protostars, TC1A and TC2A, in the Trifid Nebula.
	
	\textbf{HINT}: Use the formula book, and make sure you explain what the values from Table \ref{q2d} are referring to.
}
\begin{EnvUplevel}
	\vspace{-10pt}
	\begin{figure}[H]
		\centering
		\begin{tabularx}{0.75\textwidth}{@{}Xll@{}}
			\toprule
			\textbf{Parameters}                           & \textbf{TC1A} & \textbf{TC2A} \\ \midrule
			Stellar mass ($ \solarmass $)                 & 4.7           & 3.1           \\
			Stellar temperature (K)                       & \num{12000}   & \num{4228}    \\
			Envelope accretion rate ($\solarmass y^{-1}$) & \num{2.3e-3}  & \num{8.22e-4} \\
			Envelope cavity angle (\textdegree)           & 14            & 6.8           \\
			Disk mass ($\solarmass$)                      & 0.11          & \num{1.08e-3} \\
			$L_\text{bol}$ ($L_{\astrosun}$)              & 308           & 139           \\ \bottomrule 
		\end{tabularx}
		\renewcommand{\figurename}{Table}
		\caption{Data on the two proto-stars TC1A and TC2A}
		\label{q2d}
	\end{figure}
	\vspace{-10pt}
\end{EnvUplevel}
\filbreak
\question[2]
	\begin{parts}
		\part[1] Calculate the radius of the protostar body in TC2A using the Stefan-Boltzmann’s Law. Indicate your choice of units, either in metres or solar radii ($ R_{\astrosun} $).
		\droppoints
		\begin{solution}
			Answer: \SI{1.533e10}{\metre} \textbf{or} approximately $ 22 R_{\astrosun} $.
			
			Half a mark is deducted for careless mistakes.
		\end{solution}
		\part Explain why your answer might appear to be counter-intuitive, but is acceptable for a protostar.
		\droppoints
		\begin{solution}
			The protostar is still in the process of coalescing into a full-fledged star, after which one would expect that it would be less dense and have a larger radius than a main-sequence star that is supposedly just 3--5$ \times $ as massive as the Sun.
		\end{solution}
	\end{parts}
	
\question[4]
	Between TC1A and TC2A, which of the two protostars is \textbf{older}? Explain the difference in their ages, using as much relevant information from Table \ref{q2d} as possible, including any derived properties.
	\droppoints
	\begin{solution}
		TC1A is more recently formed while TC2A is older \textbf{[1]}.
		
		Any three of the five below:
		\begin{itemize}[leftmargin=10pt]
			\item Both protostars have masses within the same order of magnitude \textbf{[1]};
			
			\item Yet, the disk mass of TC2A is \SI{1.08e-3}{\solarmass} while the disk mass of TC1A is \SI{0.11}{\solarmass}, implying that TC1A will continue to accrete onto the protostar for longer than TC2A;
			
			\item The envelope accretion rate of TC2A is \SI{8.22e-4}{\solarmass\per\year} while the envelope accretion rate of TC1A is \SI{2.3e-3}{\solarmass\per\year} (and even factoring that TC1A is $ 1.5\times $ as heavy, the actual accretion rate is still approximately twice of the expected amount), suggesting that TC1A is still accreting material more readily \textbf{[1]};
			
			\item The radius of TC1A is \SI{2.833e10}{\metre}, compared to TC2A at \SI{1.533e10}{\metre}, making it almost twice as large and almost eight times as voluminous (and by extension, it has a significantly lower density); this indicates that it didn’t have as much time to condense compared to TC2A;
			
			\item The envelope cavity angle of TC1A is \ang{14} while that of TC2A is \ang{6.8}, suggesting that TC1A has a much thicker accretion disk (and is thus a younger protostar). Conversely, TC2A has a flatter accretion disk. 
		\end{itemize}
	\end{solution}

\filbreak
\question[1]
	What is the eventual fate of the material that is \textbf{not} accreted onto the protostar?
	\droppoints
	\begin{solution}
		They would likely coalesce and form planets and other objects (asteroids, comets, etc) \textbf{[1]}. 
		
		Half mark for incomplete answers (e.g. `becomes an asteroid belt').
	\end{solution}
	
\question[3]
	Suppose TC1A and TC2A both ignited, and became fully-fledged main-sequence stars at the same time. Which star would be more likely to outlive the other? Explain your answer, using relevant information from Table \ref{q2d}.
	\droppoints
	\begin{solution}
		\begin{itemize}[leftmargin=10pt]
			\item TC2A is more likely to outlive TC1A \textbf{[1]};
			
			\item TC1A is more massive and likely to become hotter than TC2A once it matures (both will end up as spectral class B stars if they remain in the main sequence);
			
			\item TC1A’s luminosity is higher than TC2A (308++ vs 139+ $ L_{\astrosun} $), and is expected to rise even more as it is earlier in its formation stage; it will burn through its supply of hydrogen much faster than TC2A despite having a slightly higher starting mass. Hence, it will evolve to a red giant much faster than TC2A if they were to join the main sequence at the same time \textbf{[2]}.  
		\end{itemize}
	
		Also accept: direct reference to mass-luminosity relation.
		
		Full credit if candidates make the link between higher luminosity and surface temperature to a shorter stellar lifespan, quoting the necessary values. 
		
		Cap score at 2 marks if only mass or luminosity was mentioned.
	\end{solution}

\uplevel{
	\vspace{-10pt}
	\subsection*{The Nebula Far, Far Away (The Blue Door --- More Maths Required)}
	
	Congratulations: you have escaped from the Dark Corridor! As a final challenge, the
	Blue Door remains. 
	
	\textbf{HINT}: Pass this part to the best student in your team, if you are stuck!
}

\vspace*{-10pt}
\question[3]
	Suppose that a classical Cepheid variable (i.e. a Type I Cepheid) has suddenly appeared in the Trifid Nebula. It has a period of 10 days and an apparent magnitude of 6.39. Using this information and the Formula Book, calculate the distance from the Earth to the Trifid Nebula, in \textbf{light years}.
	\droppoints
	\begin{solution}
		(Calculation performed in R)
		Plugging in Cepheid variable formulae: Absolute magnitude of Cepheid variable is $ -4.19 $ \textbf{[1]}.
		
		Plugging values into distance modulus to calculate distances: one should arrive at \SI{4.2e3}{\lightyear} \textbf{[1 + 1]}. 
	\end{solution}
\setcounter{question}{9}
\question[1]
	Explain why it is unlikely that such a Cepheid variable would ever be discovered within the Trifid Nebula.
	\droppoints
	\begin{solution}
		Classical Cepheid variables are old stars that used to be massive main sequence stars, in transition to becoming asymptotic red giants. As the Trifid nebula is a relatively young star forming region, it is unlikely for such a star to exist in the first place \textbf{[1]}.
		
		Full credit is awarded as long as the main idea is present, since this is a 1-mark question. (e.g. too little time for Cepheid variables to form; Trifid Nebula is too young to contain them, etc.)
	\end{solution}
\end{questions}

\newpage
\section{Super Blue Blood Moon}
As you might recall, a special sort of lunar eclipse occurred on the \nth{31} of January, 2018. What made it special? The Moon on that day happened to also be `super', `blue' and `blood'! This signifies a particularly rare astronomical coincidence. \begin{figure}[H]
	\centering
	\includegraphics[width=0.7\linewidth]{Graphics/Questions/Eclipse/eclipse_timelapse}
	\caption{A timelapse of the eclipse on the \nth{31} of January 2018}
	\label{q3}
\end{figure}
In this question, we will explore and investigate these properties of the Moon. 
\begin{questions}
\uplevel{\vspace*{-10pt}\subsection*{Part I: Introduction}}
\question[1]\label{q3a}
	In a single diagram, 
	\begin{parts}
		\part[1]illustrate the relative position of the Moon, the Sun, and the Earth during a lunar eclipse;
		\droppoints
		
		\part[1] draw and label the the umbra and the penumbra of the Earth’s shadow.
		\droppoints
	\end{parts}
	\textbf{NOTE}: You should place the Moon in the appropriate region of the Earth’s shadow to indicate totality.
	\begin{solution}
		\begin{figure}[H]
			\centering
			\includegraphics[scale=1]{Graphics/Questions/Eclipse/umbra_sm.png}
		\end{figure}
		\begin{figure}[H]
			\centering
			\includegraphics[width=0.7\textwidth]{Graphics/Questions/Eclipse/lunar.png}
		\end{figure}
	\end{solution}
	
\fullwidth{
	\vspace*{-10pt}
	\subsection*{Part II: Supermoon}
	We will continue by investigating the `super' property of the eclipse mentioned above. A supermoon is defined as a full moon or a new moon that coincides with the nearest approach that the Moon makes to the Earth in its elliptical orbit.
	
	What then, counts as `nearest'? According to some sources, one commonly used indicator is:
	\begin{align}
	d_s \leq (0.1d_a + 0.9d_p)
	\end{align}
	where
	\begin{itemize}[leftmargin=10pt]
		\item $ d_s $ refers to the lunar distance at syzygy (i.e a straight-line configuration between three or more celestial objects of interest---in this case, the Sun, the Earth, and the Moon);
		\item $ d_a $ refers to the lunar distance at apogee;
		\item $ d_p $ refers to the lunar distance at perigee.
	\end{itemize}
	Figure \ref{q3} reflects the distance of the Moon from the Earth, from 2018 to 2020, inclusive.
	\begin{figure}[H]
		\centering
		\includegraphics[width=15cm]{Graphics/Questions/Eclipse/newfull.pdf}
		\renewcommand{\figurename}{Figure}
		\caption{The Earth-Moon distance, and the occurrences of full and new moons}
		\label{q3}
	\end{figure}
	\vspace{-20pt}
}
\question[2]\label{q3b}
	Define $ d_\text{cutoff} $ as the cut-off lunar distance for a full or new moon to be a supermoon, whereby $ d_\text{cutoff} = 0.1d_a + 0.9d_p $. Using any relevant data from the Formula Book, determine the value of $ d_\text{cutoff} $. 
	\droppoints
	\begin{solution}
		Note that the only information that is relevant to this part of the question is the orbital semi-major axis of the Moon (at \SI{3.843e8}{\metre}) and the orbital eccentricity, $ \varepsilon = 0.0549 $).
		From Kepler’s \nth{1} Law of Planetary Motion, we can thus determine that 
		\begin{flalign*}
			d_a &= a\left(1+\varepsilon\right)= \SI{4.0540e8}{\metre} &\\
			d_p &= a\left(1-\varepsilon\right)= \SI{3.6320e8}{\metre}
		\end{flalign*}
		Hence, $ d_\text{cutoff} = 0.1 \times d_a + 0.9 \times d_p = \SI{3.6742e8}{\metre} \approx $ \textbf{\SI{3.67e8}{\metre}}
	\end{solution}

\question[2]\label{q3c}
	With reference to your working in (\ref{q3b}), describe and explain any contradiction, if any, with the data in Figure \ref{q3}.
	\droppoints
	\begin{solution}
		Note that $ d_p $ is quoted to be at \SI{3.632e8}{\metre}, but there are data points in Figure \ref{q3} which imply that $ d_p < \SI{3.6e8}{\metre}$.
		
		This is because the orbital eccentricity quoted in the Formula Booklet refers to the \textit{mean eccentricity} of the Moon’s orbit. This means that if we consider the Moon to be at its maximum eccentricity (higher $ \varepsilon $ value), this means that it is possible for $ d_p $ to be lower, such that it is less than \SI{3.6e8}{\metre} (note that $ \varepsilon $ may vary by up to 0.08).
	\end{solution}

\filbreak
\question[3]
	Use your answer in (\ref{q3c}) to give a difference between the \textbf{absolute} visual magnitude of the Moon when it is at its perigee, and that of the Moon when it is at its apogee. 
	
	State any assumptions you have made in your calculations, and explain why they are valid. 
	\droppoints
	\begin{solution}
		\textbf{Assumption}: The solar irradiance on the Moon is kept relatively constant. This is because the difference in distance from the Sun when the Moon is at its apogee and perigee is not significant compared to the average distance of the Moon from the Sun. 
		
		This means that we do not apply the `inverse square law' to obtain the difference in intensity of the Sun rays at these two locations, since they are relatively constant.
		
		Calculation: Using either the inverse square law for the light that is reaching us or using the angular size of the Moon, we can then deduce that 
		$\displaystyle \frac{L_p}{L_a}=\left(\frac{4.0540}{3.6320}\right)^2=1.246 $
		Thus, using the formula for calculating difference in magnitude, we have 
		$\displaystyle \Delta M = 2.5 log \left(\frac{L_p}{L_a}\right) = 0.239 $
	\end{solution}

\uplevel{\vspace*{-10pt}\subsection*{Part III: Blood Moon}}

\question[1]
	\vspace{-10pt}
	Give a reason for the colouration of the `blood moon'. 	
	\droppoints
	\begin{solution}
		Rayleigh scattering $\displaystyle I \propto \frac{1}{\lambda^4} $. Hence, blue light is scattered more than red light (since $ \lambda_\text{blue} < \lambda_\text{red} $).  This means that sunlight passing through the Earth's atmosphere will have its blue component scattered, and only the red component will be incident on the Moon, giving the Moon its reddish hue during the total lunar eclipse.
	\end{solution}
	
\fullwidth{\vspace*{0pt}\textbf{Use the following information to answer questions (\ref{3g}) and (\ref{3h})}.
	
In 2013, a Japanese research team of astronomers and planetary scientists used the Subaru Telescope’s two optical cameras and observed that the super-Earth exoplanet, GJ 1214b, has a water-rich, but less extended atmosphere as compared to Earth.\vspace*{-10pt}}
\question[1]\label{3g}
	Now, suppose GJ 1214b has a moon. Given the conditions in (\ref{q3a}) and (\ref{q3b}) have been met, would it be more or less likely that eclipses would cause its moon to become a `blood moon'?
	\droppoints
	\begin{solution}
		Less likely. A less-extended atmosphere implies that the scattering effect is reduced (i.e. more blue light is incident on its satellite). Hence, it is possible that the reddish look of its own satellite will not be present.
	\end{solution}

\question[1]\label{3h}
	Would your answer in (\ref{3g}) change if the atmosphere was hydrogen-rich instead? Explain your answer.
	\droppoints
	\begin{solution}
		Yes. The final effect now is \textbf{indeterminate}. This is because the scattering effect does not just depend purely on the wavelength of the light, but also on the particle size.
		
		To be more exact, the smaller the particle size, the greater the scattering effect. Hence, a less extended atmosphere which results in a weaker scattering effect offsets the effect of a smaller particle size.
	\end{solution}

\fullwidth{\vspace*{-10pt}\subsection*{Part IV: Blue Moon}
We've understood why the Moon on the night of the \nth{31} of January, 2018 was \textbf{super} and \textbf{bloodshot}; let's proceed to investigate why it was \textit{also} \textbf{blue}. Clearly, the Moon is never really \textit{blue}, is it? Furthermore, it couldn't possibly be \textit{simultaneously} blue and bloody. 

This suggests that `blue' actually refers to something else---something rather unrelated to its colour.

Typically, a `blue moon' is the second full moon in any given calendar month of the Gregorian calendar---the one in common, daily use today.

To be able to predict the occurrences of blue moons, one would first need to know the time interval between successive full moons, with respect to an observer on the Earth. The long-term average of this value called a \textbf{synodic month}.\vspace*{-10pt}
}

\setcounter{question}{9}
\question[3]\label{q3j}
	According to the Formula Book, the length of a sidereal month is 27.322 days, and the length of a synodic month is 29.531 days.
	
	Use \textbf{only} the length of the sidereal month, and the orbital period of the Earth around the Sun (365.24 days) to derive the length of the \textit{synodic} month, as given above.
	\droppoints
	\begin{solution}
		Assume that the synodic period is given by $ T $ days. The angular displacement of the Earth with respect to the Sun is given by $\displaystyle \theta_E = \frac{T}{365.24}\times \ang{360}$.
		
		Over $ T $, the angular displacement of the Moon must be $\displaystyle \ang{360} + \theta_E=\frac{T}{27.322} \times \ang{360}$, since the Moon completes one round + the angular displacement of the Earth. 
		\begin{flalign*}
			\ang{360} &+ \frac{T}{365.24} \times \ang{360} = \frac{T}{27.322} \times \ang{360} &\\
			1 &+ \frac{T}{365.24} = \frac{T}{27.322} &\\
			\implies T &= \frac{1}{\left(\frac{1}{27.322} - \frac{1}{365.24}\right)} = \text{\textbf{29.531 days}}
		\end{flalign*}
	\end{solution}
	
\uplevel{\vspace*{0pt}In the remaining parts of this question, we can make use of (\ref{q3j}) to make suitable approximations for calendrical calculations. Bear in mind any assumptions made while answering (\ref{q3j}).\vspace*{-10pt}}
	
\question[3]
	As an inquisitive astronomer, you want to know the \textbf{maximum} number of blue moons that could occur in a calendar year. 
	
	Use the concept of a synodic month to determine the maximum number of blue moons that can happen in a calendar year. Explain your answer.
	
	\textbf{HINT}: To determine the maximum number of blue moons, one might begin by assuming that a full moon occurs at midnight on the \nth{1} of January of a given year.
	\droppoints
	\begin{solution}
		Answer: A maximum of 2 blue moons per calendar year.
		
		To determine the maximum, we can always start off with a best-case scenario: that is, on the 1st of January, there is a full moon at 00:00. Since a synodic period is approximately 29.5 days, we can expect the \textit{next} new moon to occur on the 30th of January, at approximately 12:00, which is hence a blue moon.
		
		Next, we set February to have 28 days (i.e. in a \textbf{non-leap year}). Hence, the next full moon will occur, once again, 29.5 days after 30 January---which would be on 1 March, at 00:00.
		
		Long story short, the synodic period of the Moon is \textit{nearly} as long as the length of months, such that blue moons only occur twice a year. 
	\end{solution}

\setcounter{question}{12}	
\question[2]
	\textbf{Refer to Table \ref{q3m} to answer this part.}
	\begin{figure}[H]
		\centering
		\begin{tabularx}{0.3\textwidth}{@{}lXl@{}}
			\toprule
			\textbf{Year} & \textbf{Date}    & \textbf{Time}   \\ \midrule
			2020          & 31 October & 14:47  \\
			2039          & 31 October & 22:36  \\
			2058          & 31 October & 12:51  \\
			2077          & 31 October & 10:36  \\
			2096          & 31 October & 11:13  \\ \bottomrule
		\end{tabularx}
		\renewcommand{\figurename}{Table}
		\caption{Occurrences of a blue moon in selected years on the Gregorian calendar}
		\label{q3m}
		\vspace*{-10pt}
	\end{figure}
	Give a reason for any trends that you might observe in Table \ref{q3m}.	
	\droppoints
	\begin{solution}
		\textbf{Trend}: A blue moon occurs on the same date every 19 years. 
		
		\textbf{Explanation}: This corresponds to the \textit{Metonic cycle}, which has a period of nearly 19 years, and is a common multiple of the solar year and the synodic month. This means that the phases of the Moon repeat every 19 years.
	\end{solution}
\end{questions}
\newpage
\section{Cosmology --- The Next Decade}
As a branch of astronomy, the study of cosmology offers new insights about the origin and evolution of the Universe. The upcoming decade will mark an exciting time for cosmology, as we await the next generation of cosmological experiments to be operational. 

These experiments will let humanity take a step closer to uncovering the secrets behind many great observational studies, such as dark matter and the future of the Universe.

This question will put you, the student, in the shoes of cosmologists, enabling you to understand how these interesting observations have arisen from experimental data.

\subsection*{Part I: Unobservable Matter?}
The discovery of dark matter traces back to the 1930s when the Swiss astronomer, Fritz Zwicky, made an interesting observation that galaxies in the distant Coma Cluster were orbiting one another much faster than they \textit{should} have, been given the amount of \textit{visible} mass.
\begin{questions}
\question[2]
	There is further, strong evidence for the existence of dark matter, which can be gleaned from rotation curves of galaxies, as shown by Vera Rubin and Kent Ford in the 1960s. Our galaxy, the Milky Way, is no exception to this observation.
	
	With reference to rotation curves, give a reason for the conclusion that most of the mass in the Milky Way is in some undetected form.
	\droppoints
	\begin{solution}
		Considering the rotation curve of the Milky Way, the curve is flat out at great distances from the centre of the galaxy. Theoretically, if the mass at the outer edges of the galaxy were constant, the gravitational force and hence the rotational speed should decrease at large distances according to Kepler's Second Law. The fact that the rotational speed remains relatively constant at large distances from the core indicates the presence of extra mass. It is this extra undetected mass that causes the rotational speed at the outer edges to be higher than expected.
	\end{solution}

\question[1]
	Compare the rotation curves of planets as described by Kepler's Laws, and that of galaxies. As already mentioned, galaxies do \textit{not} appear to follow Kepler's Laws; what does this tell you about the mass distribution in galaxies?
	\droppoints
	\begin{solution}
		Unlike planets where mass is concentrated at a point, mass in galaxies are distributed.
	\end{solution}

\question[1]\label{q4c}
	Given the vast distances between most stars in the Milky Way, it is almost guaranteed that no two stars will ever collide in their life-times; this implies that their motions are solely caused by gravitational interactions. 
	
	Now, consider a star orbiting the galactic centre of the Milky Way, and assume that it follows a circular orbit, and that all galactic mass is enclosed within the galactic disc.
	
	Derive a formula showing the relation between galactic mass and distance from the galactic centre. 
	\droppoints
	\begin{solution}
		Setting centripetal force equal to gravitational force:
		\begin{align*}
			\frac{mv^2}{r} = \frac{Gmm}{r^2}
		\end{align*}
		Where $ m $ is the mass of the star, $ v $ is its orbital speed, $ M $ is the enclosed galactic mass and $ r $ is the radial distance from the galactic centre. Simplifying and rearranging gives:
		\begin{align*}
			v = \sqrt{\frac{GM}{r}} \implies M = \frac{rv^2}{G}
		\end{align*}
	\end{solution}
	
\question[1]
	Hence, describe the relationship between galactic mass and distance from the galactic centre if it were found that the orbital speeds of stars are relatively constant at distances near the edge of the Milky Way.
	\droppoints
	\begin{solution}
		At distances beyond the galactic disc, if the orbital speeds of stars are constant, we see that the galactic mass is directly proportional to the distance from galactic centre.
	\end{solution}
	
\question[2]\label{q4e}
	Show that, for a \textbf{spherical} galaxy, the average density function, $\rho$, is given by
	\begin{align*}
		\vspace*{-20pt}
		\rho = \frac{3v^2}{4\pi Gr^2}.
	\end{align*}
	\droppoints
	\begin{solution}
		Assuming a spherical galaxy, the average density function is given by:
		\begin{align*}
			\rho = \frac{m}{v} = \frac{\frac{rv^2}{G}}{\frac{4}{3}\pi r^3} = \frac{3v^2}{4\pi Gr^2}
		\end{align*}
	\end{solution}

\newpage
\uplevel{\vspace{-20pt}What we are interested in, is in fact the \textit{local density function}, given by $ \displaystyle \rho(r) = \frac{v^2}{4\pi G r^2} $ instead of that in (\ref{q4e}). This refers to the density, $ \rho $, at a distance $ r $ away from the galactic centre. 
	
	Experimentally, it was discovered that dwarf galaxies, star clusters and gas clouds located at a distance of about six times the radius of the Milky Way galaxy from its centre, have roughly the same orbital speeds as the stars near the edge of the Milky Way. }
\question[3]
	\vspace*{-10pt}
	Using this experimental fact, the \textit{local} density function, and the mass-radial distance relationship you found in (\ref{q4c}), explain how the above paragraph is strong evidence for the presence of dark matter, even \textit{beyond} the edge of the Milky Way galaxy.
	\droppoints
	\begin{solution}
		The fact that the rotational speeds for objects beyond the edge of the Milky Way remain constant
		means that the local density function and the mass relation derived in (\ref{q3c}) also applies to apparently empty regions slightly beyond the edge of the Milky Way. 
		
		Since the orbital speeds of the objects in these regions are constant, the galactic mass is directly proportional to distance measured from the galactic centre. Hence, the \textbf{total} galactic mass is six times that of the visible mass. This also means that $ \frac{5}{6} $ of the galactic mass is invisible (roughly 85\%), corresponding to dark matter.
	\end{solution}

\fullwidth{\vspace{-10pt}\subsection*{Part II: The Expanding Cosmos}
	In this section, we shall look at cosmic expansion. If one were to observe galaxies and other objects outside the Milky Way, they would notice that other galaxies not gravitationally bound to the Milky Way, are generally moving away from the latter. This phenomenon was first observed by the
	American astronomer, Edwin Hubble (1889--1953).
	
	As seen in the Formula Book, Hubble's Law at a \textit{particular} instance in time is given by $ v = H_0d $, where $ v $ denotes the object's recession velocity, $ d $ is the distance of the object from the centre of the Milky Way, and $ H_0 $ is the Hubble constant \textit{at that point in time}.\vspace{-10pt}}
\question
	The critical density of the Universe is the density at which the gravitational attraction of matter
	within the universe is balanced with its expansion. Though formally shown through the Friedmann
	Equation, it is also possible to derive the critical density of the Universe using classical means. 
	
	\begin{parts}
		\part[5]\label{q4gi}Assume that the Universe is spherical and has a uniform density. Now, using Hubble's Law, show that the critical density of the Universe, $\rho$, is given by
		\begin{align*}
			\rho = \frac{3H_0^2}{8\pi G} \vspace*{-20pt}
		\end{align*}
		\droppoints
		\begin{solution}
			We start by deriving the escape velocity of the galaxy at the edge of the Universe. Applying the Principle of Conservation of Energy, $\displaystyle -\frac{GMm}{r^2} + \frac{1}{2}mv^2 = 0 + 0 \implies v = \sqrt{\frac{2GM}{r}} $
			
			where $ v $ is the recession speed of the galaxy, $ r $ is the arbitrary radial distance of the galaxy at a particular point in time, $ m $ is the mass of the galaxy and $ M $ is the enclosed mass of the rest of the universe. 
			
			Applying Hubble's Law and rearranging gives: $ \displaystyle H_{0}r = \sqrt{\frac{2GM}{r}} \implies H^2_0r^2 = \frac{2GM}{r} $
			
			For an arbitrary spherical volume, $ \displaystyle H^2_0r^2 = \frac{2G}{r}\left ( \frac{4}{3}\pi \rho r^3 \right) \implies \rho = \frac{3H^2_0}{8\pi G} $
		\end{solution}
		
		\part[1] Write down an assumption that you have made in answering (\ref{q4gi}).
		\droppoints
		\begin{solution}
			Any one of the following assumptions to be awarded a mark:
			\begin{itemize}[leftmargin=10pt]
				\item Recession speed of the object is non-relativistic
				\item Assuming a large enough distance $ r $, local motions such as the motion of stars within the galaxy and the motion due to gravitational interactions between galaxies are small compared to the Hubble's speed.
			\end{itemize}
		\end{solution}
	\end{parts}

\question[2]\label{4h}
	The value of the Hubble constant as of March 2013, was $ H_0 = \SI{67.8}{\kilo\metre\per\second\per\mega\parsec} $. It is also given that the mean density of galaxies is about one hydrogen atom per cubic centimetre; whereas, the mean density of intergalactic space, is about one-millionth that of the above. The mass of a hydrogen atom is \SI{1.67e-27}{\kilo\gram}.
	
	How would the expansion of the Universe proceed with respect to galaxies?  
	\droppoints
	\begin{solution}
		First, we convert the Hubble's constant to S.I. units:
		
		$ \displaystyle H_0 = \frac{\SI{67.8}{\kilo\metre\per\second\per\mega\parsec}}{\SI{3.08568e19}{\kilo\metre\per\mega\parsec}} = \SI{2.20e-18}{\per\second} $
		
		Next, we calculate the critical density of the universe as of March 2013:
		
		$\displaystyle \rho_0 = \frac{3(\SI{2.20e-18}{\per\second})}{8\pi(\SI{6.67e-11}{\newton\metre\squared\per\kilogram})} = \SI{8.66e-27}{\kilo\gram\per\metre\cubed} $
		
		Converting to \ce{H} atoms per cubic metre,
		$ \displaystyle \rho_0 = \frac{\SI{8.66e-27}{\kilo\gram\per\metre\cubed}}{\SI{1.67e-27}{\kilo\gram}\cdot \text{atoms}^{-1}}  = 5.19 \text{ atoms m}^{-3} $
		
		Hence, we see that galaxies are sufficiently dense to resist the expansion of space, while the space between galaxies is subject to expansion.
	\end{solution}

\setcounter{question}{9}
\filbreak
\question[2]
	The expansion of the Universe is best visualised using a loaf of raisin bread. Imagine a loaf of bread, embedded with raisins. When the bread is proofed and baked, it expands significantly. 
	
	Explain how this analogy describes the expansion of the Universe, as concluded in (\ref{4h}). 
	\droppoints
	\begin{solution}
		As the loaf rises, we observe the expansion of space between each raisin (galaxy) attached to the surface. The raisins themselves, however, do not expand. In other words, the recession of the galaxies is mainly due to the expansion of space. What is so interesting is that if one were to be on any raisin on the surface, one would observe all the other raisins `rushing
		away'. 
		This confirms Hubble's Law.
	\end{solution}
\end{questions}

\newpage
\section{Chinese Astronomy}
Chinese astronomy has a remarkable history. Just like how Greco-Roman astronomers in the Western world divided the sky into constellations, ancient Chinese astronomers divided the sky into five great regions called \textit{gong} (palaces), equated with directions North, South, East, West, and a central region. 

The latter region of stars was most significant as it symbolised the emperor surrounded by his family and generals. The rest of the sky was classified into the four cardinal directions, each associated with an animal and colour, as seen in Figure \ref{q5}.
\begin{figure}[H]
	\centering
	\includegraphics[width=0.75\textwidth]{Graphics/Questions/Chinese/chineseskywallchart.png}
	\caption{An ancient Chinese star map, complete with the Four Symbols}
	\label{q5}
\end{figure}
The animals were the Azure Dragon of the East, the Black Tortoise of the North (typically entwined with a green snake), the White Tiger of the West, and the Vermillion Bird of the South; these animals were associated with spring, winter, autumn, and summer respectively.
\begin{questions}
\filbreak
\question[1]
	Based on the information given, label the cardinal directions on the chart in \textbf{page \pageref{chinese_sky}}.
	\droppoints
	\begin{solution}
		\begin{figure}[H]
			\centering 
			\textbf{N}
			
			\centering
			\textbf{E}\includegraphics[width=0.4\textwidth]{Graphics/Questions/Chinese/chinese_sky.png}\textbf{W}
			
			\centering
			\textbf{S}
		\end{figure}
	\end{solution}

\question[1]
	Are the cardinal directions positioned correctly in the ancient Chinese star charts, as compared to a modern star chart? Explain your answer.
	\droppoints
	\begin{solution}
		Yes, they are. Remember that when reading a star map, we are meant to look upwards into the sky (rather than down into the ground). Thus the E/W directions are flipped.
		
		Award 0.5 if candidate shows recognition of the flipping of the cardinal directions, in particular East and West directions.
	\end{solution}

\fullwidth{Like the modern IAU constellations, the stars within each \textit{gong} (palace) are further grouped into 28 segments known as \textit{xiu} (mansions). They are formed by imaginary lines joining the stars together, forming patterns in the night sky.
	
\textbf{Refer to Figure 9 to answer parts (\ref{5c}) to (\ref{5e}).}
\begin{figure}[H]
	\centering
	\begin{tikzpicture}
		\node[anchor=south west,inner sep=0] (image) at (0,0) {\includegraphics[width=0.75\textwidth]{Graphics/Questions/Chinese/three_stars}};
		\begin{scope}[x={(image.south east)},y={(image.north west)}]
			\draw[red,thick] (0.33,0.55) ellipse [x radius=0.2,y radius=0.15] node[above=0.16,color=black,fill=white,rounded corners=2.5pt] {Three Stars};
			\draw[red,thick] (0.47,0.54) circle [radius=0.02] node[right=0.03,color=black,fill=white,rounded corners=2.5pt] {Turtle};
			\draw[red,thick] (0.64,0.83) ellipse [x radius=0.10,y radius=0.165] node[right=0.12,color=black,fill=white,rounded corners=2.5pt] {Well};
			\draw[red,thick,] (0.52,0.17) ellipse [x radius=0.08,y radius=0.1] node[left=0.1,color=black,fill=white,rounded corners=2.5pt] {Net};
			\draw[red,thick] (0.64,0.02) ellipse [x radius=0.02,y radius=0.02] node[above right=0.02,color=black,fill=white,rounded corners=2.5pt] {Hairy Head};
		\end{scope}
	\end{tikzpicture}
	\caption{M42 and surroundings, according to the Chinese star chart}
	\label{q5c}
\end{figure}\vspace*{-10pt}}
\question[0.5]\label{5c}
	M42, known as the Great Orion Nebula, is one of the most visible and famous deep sky objects of the winter sky. In Western astronomy, it is in a constellation easily identifiable with four stars forming a trapezium, surrounding three equally bright stars, nearly equally placed apart, in the middle.	
	
	Identify the \textit{xiu} that M42 is located within.
	\droppoints
	\begin{solution}
		Three Stars
	\end{solution}

\question[1.5]
	One can locate a reddish-orange star known as Aldebaran by forming an imaginary line from the three stars in the centre and extending the line outwards. Aldebaran is observed within an open cluster with a distinctive ‘V’ shape.
	\begin{parts}
		\part[1] State the name of this open cluster, and the IAU constellation it is located in.
		\droppoints
		\begin{solution}
			Hyades in Taurus.
		\end{solution}
		
		\part[0.5]State the \textit{xiu} this open cluster is located in. 
		\droppoints
		\begin{solution}
			Net
		\end{solution}
	\end{parts}

\question[1]\label{5e}
	In Figure \ref{q5c}, one can see a fuzzy patch of stars in another open cluster, M45. Commonly known as the Seven Sisters in Western astronomy, the name originated from its seven distinctive bright stars; the ancient Chinese astronomers dedicated a \textit{xiu} to it alone. 
	
	State the \textit{xiu} that M45 is in, \textit{and} the Western constellation that M45 is located in.
	\droppoints
	\begin{solution}
		Hairy Head, in Taurus.
	\end{solution}
	
\fullwidth{
	\textbf{Refer to Figure \ref{fig10} to answer parts (\ref{5f}) to (\ref{5h}).}
	\begin{figure}[H]
	\centering
	\begin{tikzpicture}
		\node[anchor=south west,inner sep=0] (image) at (0,0) {\includegraphics[width=0.75\textwidth]{Graphics/Questions/Chinese/south_dipper.png}};
		\begin{scope}[x={(image.south east)},y={(image.north west)}]
			\draw[rotate around={80:(0.26,0.44)},red,thick] (0.26,0.44) ellipse[x radius=0.15,y radius=0.1] node[above=0.21,color=black,fill=white,rounded corners=2.5pt] {South Dipper};
			\draw[rotate around={110:(0.41,0.38)},red,thick] (0.41,0.38) ellipse[x radius=0.08,y radius=0.07] node[below=0.12,color=black,fill=white,rounded corners=2.5pt,align=left] {Winnowing\\Basket};
			\draw[color=red,thick] (0.63,0.43) ellipse[x radius=0.12,y radius=0.15] node[right=0.1,color=black,fill=white,rounded corners=2.5pt] {Tail};
			\draw[color=red,thick,rotate around={70:(0.70,0.77)}] (0.70,0.77) ellipse[x radius=0.05,y radius=0.02] node[left=0.03,color=black,fill=white,rounded corners=2.5pt] {Heart};
			\draw[color=red,thick,rotate around={32:(0.78,0.87)}] (0.78,0.87) ellipse[x radius=0.02,y radius=0.14] node[right=0.04,color=black,fill=white,rounded corners=2.5pt] {Room};
		\end{scope}
	\end{tikzpicture}
	\caption{The South Dipper, Tail and vicinity}
	\label{fig10}
\end{figure}

\vspace*{-10pt}}	
\question\label{5f}
	Although the Chinese \textit{xiu} and the Western constellations look very different, some share distinct similarities in terms of their patterns. In Figure \ref{fig10}, the three \textit{xiu} (mansions) Room, Heart and Tail together form a certain IAU constellation. 
	\begin{parts}
		\part[1]\label{q5ei} Suggest the name of this IAU constellation.
		\droppoints
		\begin{solution}
			Scorpius.
		\end{solution}
		
		\part[2] Name a deep sky object that can be found in the IAU constellation identified in (\ref{q5ei}). State the \textit{xiu} that the object stated is located in. 
		\droppoints
		\begin{solution}
			\textbf{Accepted answers}: 
			\begin{itemize}[leftmargin=10pt]
				\item M6/M7/False Comet; in Tail. 
				\item M4, in Heart.
			\end{itemize}
		\end{solution}
	\end{parts}

\question[1]
	In Chinese astronomy, the \textit{xiu} Heart contains a bright reddish star known as \textit{Heart II}, symbolising the dragon’s heart. With an apparent magnitude of 1.06, this star can be easily seen with the naked eye. 
	
	Give the name of this star in modern Western astronomy.
	\droppoints
	\begin{solution}
		Antares
	\end{solution}
	
\question[1]\label{5h}
	In Chinese astronomy, the \textit{xiu}, South Dipper consists of six bright stars. Together with the \textit{xiu}, Winnowing Basket, some of the stars trace out an asterism, called the \textit{Teapot}, in Western astronomy. 
	
	Which IAU constellation contains the \textit{Teapot}?
	\droppoints
	\begin{solution}
		Sagittarius
	\end{solution}

\setcounter{question}{9}
\question
	In Western astronomy, another unrelated asterism is called the Big Dipper.
	\begin{parts}
		\part[1] Which IAU constellation contains the Big Dipper?
		\droppoints
		\begin{solution}
			Ursa Major
		\end{solution}
		
		\part[1] One can use this asterism to find Polaris. Suggest how one might do so.
		\droppoints
		\begin{solution}
			Use the front edge of the Big Dipper, draw a line extending upwards to a star of moderate brightness; this star is Polaris.
		\end{solution}

		\part[2] Briefly explain the significance of this star and its location in the sky with respect to a viewer standing on the Equator.
		\droppoints
		\begin{solution}
			Polaris lies near the North Celestial Pole and thus marks celestial true North. Thus from the equator, Polaris would lie near the horizon, directly due North of the viewer.
		\end{solution}
	\end{parts}

\fullwidth{\textbf{Answer this question on the diagrams found on pages \pageref{threestars} and \pageref{southdipper}.}
	
\vspace*{-10pt}}
\question
	In ancient Chinese mythology, two bright stars, Altair and Vega, located on either side of the Milky Way, symbolised a separated pair of lovers across a river. Together with another bright star, these three stars form an interesting asterism in Western astronomy. 
	\begin{parts}
		\part[2] State the name of the third bright star and the name of the asterism.
		\droppoints
		\begin{solution}
			Deneb, which is one of the vertices of the Summer Triangle
		\end{solution}
		
		\part[2] Identify two double stars \textit{or} deep sky objects that may be found within Figure \ref{q5c}. Indicate their approximate locations in the diagram on \textbf{page \pageref{threestars}}.
		\droppoints
		\begin{solution}
			Accept reasonable answers, but the easiest remaining objects would probably be Rigel and M41.
		\end{solution}
		
		\part[2] Identify two double stars or deep sky objects that may be found within Figure \ref{fig10}. Indicate their approximate locations in the diagram on \textbf{page \pageref{southdipper}}.
		\droppoints
		\begin{solution}
			Accept reasonable answers, but the easiest remaining objects would probably be the Lagoon Nebula (M8) as well as the objects in Scorpius that were not previously used in the question.
		\end{solution}
		
		\textbf{NOTE}: Your answers should \textit{not} include deep sky objects that you have previously listed in this question.
	\end{parts}
\end{questions}

\newpage
\begin{center}
	\textbf{\textsc{Carefully detach this page and attach it to your answer scripts.}}
\end{center}

\vfill
\begin{figure}[H]
	\centering
	\includegraphics[width=\textwidth]{Graphics/Questions/Chinese/chinese_sky.png}
	\label{chinese_sky}
\end{figure}
\vfill
\hspace{0pt}
\pagebreak

\newpage
\begin{center}
	\textbf{\textsc{Carefully detach this page and attach it to your answer scripts.}}
\end{center}

\hspace{0pt}
\vfill
\begin{figure}[H]
	\centering
	\includegraphics[width=\paperwidth,angle=90]{Graphics/Questions/Chinese/three_stars}
	\label{threestars}
\end{figure}
\vfill
\hspace{0pt}
\pagebreak


\newpage
\begin{center}
	\textbf{\textsc{Carefully detach this page and attach it to your answer scripts.}}
\end{center}

\hspace{0pt}
\vfill
\begin{figure}[H]
	\centering
	\includegraphics[width=\paperwidth,angle=90]{Graphics/Questions/Chinese/south_dipper}
	\label{southdipper}
\end{figure}
\vfill
\hspace{0pt}
\pagebreak
\end{document}
