\documentclass[a4paper,11pt]{exam}

%%%%%%%%%%%%%%%%%%%%%%%%%%%%%%%%%%%%%%%%%%%%%%%%%%%%%%%%%%%%%%%%%%%%%%%%%%%%%%%%%%%%
%
% PACKAGE IMPORTS %%%%%
%
%%%%%%%%%%%%%%%%%%%%%%%%%%%%%%%%%%%%%%%%%%%%%%%%%%%%%%%%%%%%%%%%%%%%%%%%%%%%%%%%%%%%

%%%%% SOME MISC IMPORTS TO DEFINE STUFF FIRST %%%%%
\usepackage{etoolbox}
\usepackage{pgfkeys}

%%%%% FONTS & SYMBOLS %%%%%
\usepackage[utf8]{inputenc}
\usepackage[T1]{fontenc}
\pgfkeys{
	/ac/.is family, /ac/.cd,
	font/.is choice,
	font/lm/.code={
			\usepackage{lmodern}
			\usepackage{amsfonts,amssymb}
			\providecommand{\tablining}{}
			\providecommand{\propold}{}
			% \renewcommand{\textssc}[1]{\textsc{#1}}
			% \newcommand{\boldsmallcaps}[1]{{\fontfamily{cmr}\textsc{\textbf{#1}}}}
			\usepackage{inconsolata}
		},
	font/mpro/.code={
			\usepackage{MnSymbol}
			\usepackage[
				minionint,
				lf,
				mathtabular,
				loosequotes,
				swash,
				opticals,
				footnotefigures]{MinionPro}
			\providecommand{\tablining}{\figureversion{tab}}
			\providecommand{\propold}{\figureversion{osf}}
			\newcommand{\boldsmallcaps}[1]{\textssc{\textbf{#1}}}
			\usepackage{inconsolata}
		},
	font/lmodern/.code=\pgfkeysalso{font/lm},
	font/minionpro/.code=\pgfkeysalso{font/mpro},
}
\providerobustcmd*{\setfont}[1]{\pgfqkeys{/ac}{#1}}
\usepackage{amsmath}
\usepackage{wasysym}
\usepackage{microtype}

%%%%% GEOMETRY, PAGE SETUP, SPACING, PARAGRAPHING %%%%%
\usepackage[margin=2.5cm,a4paper]{geometry}
\usepackage{titlesec}
\usepackage{multicol}
\usepackage{multirow}
\usepackage{parskip}
\usepackage{tabto}
\usepackage{pdflscape}
\usepackage{enumitem}
\usepackage{adjustbox}
\usepackage[super]{nth}

%%%%% SCIENCE FORMATTING %%%%%
\usepackage{physics}
\usepackage[
	arc-separator = \,,
	retain-explicit-plus,
	%inter-unit-product =\cdot,
	detect-weight=true,
	detect-family=true,
	range-phrase=--,
	range-units=single
]{siunitx}
\usepackage[version=4]{mhchem}
\usepackage[makeroom]{cancel}
\renewcommand{\CancelColor}{\color{red}}

%%%%% GRAPHICS, CAPTIONS, TABLES %%%%%
\usepackage[table,dvipsnames]{xcolor}
\usepackage{graphicx}
\usepackage{float}
\usepackage{tikz,tikz-3dplot}
\usepackage{pgfplots}
\usepackage{pdfpages}
\usepackage[
	justification=centering,
	labelfont={small,bf},
	font={small}
]{caption}
\usepackage{subcaption}
\usepackage{array}
\usepackage{tabularx}
\usepackage{booktabs}
\usetikzlibrary{
	calc,
	arrows,
	arrows.meta,
	positioning,
	decorations.pathreplacing,
	decorations.markings,
	decorations.text,
	calligraphy,
	pgfplots.dateplot
}
\pgfplotsset{compat=1.17}
\usepackage[outline]{contour}
\contourlength{1pt}
\newcommand*\circled[1]{
	\begin{tikzpicture}[baseline=(char.base)]
		\node[shape=circle, draw, minimum size=1.5em, inner sep=0pt, thick] (char) {#1};
	\end{tikzpicture}
}
\tikzset{
	mid arrow/.style={postaction={decorate,decoration={
							markings,
							mark=at position .15 with {\arrow[#1]{stealth}}
						}}},
	>=stealth
}

%%%%% REFERENCES AND LINKS %%%%%
\usepackage{hyperref}
\usepackage[noabbrev]{cleveref}

%%%%% MISCELLANEOUS %%%%%
\usepackage[useregional,calc]{datetime2}
\DTMlangsetup[en-GB]{ord=raise}

%%%%%%%%%%%%%%%%%%%%%%%%%%%%%%%%%%%%%%%%%%%%%%%%%%%%%%%%%%%%%%%%%%%%%%%%%%%%%%%%%%%%
%
% MATHS MACROS %%%%%
%
%%%%%%%%%%%%%%%%%%%%%%%%%%%%%%%%%%%%%%%%%%%%%%%%%%%%%%%%%%%%%%%%%%%%%%%%%%%%%%%%%%%%

\newcommand{\tomb}{\quad\blacksquare{}}

%%%%%%%%%%%%%%%%%%%%%%%%%%%%%%%%%%%%%%%%%%%%%%%%%%%%%%%%%%%%%%%%%%%%%%%%%%%%%%%%%%%%
%
% CREF AND HYPERREF SETUP %%%%%
%
%%%%%%%%%%%%%%%%%%%%%%%%%%%%%%%%%%%%%%%%%%%%%%%%%%%%%%%%%%%%%%%%%%%%%%%%%%%%%%%%%%%%

\crefdefaultlabelformat{#2\textbf{#1}#3}

\creflabelformat{equation}{#2\textbf{(#1)}#3}
\creflabelformat{figure}{#2\textbf{#1}#3}

\crefname{equation}{\textbf{equation}}{\textbf{equations}}
\Crefname{equation}{\textbf{Equation}}{\textbf{Equations}}
\crefname{figure}{\textbf{Figure}}{\textbf{Figures}}
\Crefname{figure}{\textbf{Figure}}{\textbf{Figures}}
\crefname{table}{\textbf{Table}}{\textbf{Tables}}
\Crefname{table}{\textbf{Table}}{\textbf{Tables}}
\crefname{appendix}{\textbf{Appendix}}{\textbf{Appendices}}
\Crefname{appendix}{\textbf{Appendix}}{\textbf{Appendices}}
\crefname{section}{\textbf{\S}}{\textbf{\S}}
\Crefname{section}{\textbf{\S}}{\textbf{\S}}
\crefname{algorithm}{\textbf{Algorithm}}{\textbf{Algorithms}}
\Crefname{algorithm}{\textbf{Algorithm}}{\textbf{Algorithms}}

%%%%% SPECIFIC TO EXAM CLASS%%%%%
\creflabelformat{question}{#2\textbf{#1}.#3}
\creflabelformat{partno}{(#2\textbf{#1}#3)}
\creflabelformat{subpart}{(#2\textbf{#1}#3)}

\crefname{question}{question}{questions}
\Crefname{question}{Question}{Questions}
\crefname{partno}{}{}
\Crefname{partno}{}{}
\crefname{subpart}{}{}
\Crefname{subpart}{}{}

\hypersetup{
	colorlinks   = true,            %Colours links instead of ugly boxes
	urlcolor     = NavyBlue,        %Colour for external hyperlinks
	linkcolor    = Magenta,         %Colour of internal links
	citecolor    = Aquamarine       %Colour of citations
}

%%%%%%%%%%%%%%%%%%%%%%%%%%%%%%%%%%%%%%%%%%%%%%%%%%%%%%%%%%%%%%%%%%%%%%%%%%%%%%%%%%%%
%
%%%%% EXAM CLASS SETUP %%%%%
%
%%%%%%%%%%%%%%%%%%%%%%%%%%%%%%%%%%%%%%%%%%%%%%%%%%%%%%%%%%%%%%%%%%%%%%%%%%%%%%%%%%%%

%%%%% CHOICES ON ONE PAGE %%%%%
\BeforeBeginEnvironment{choices}{\par\nopagebreak\minipage{\linewidth}}
\AfterEndEnvironment{choices}{\endminipage}

%%%%% QUESTION/CHOICE LABELS %%%%%
\renewcommand{\questionlabel}{\thequestion.\hfill}
\renewcommand{\subpartlabel}{(\thesubpart)}
\renewcommand{\choicelabel}{\circled{\thechoice}}

%%%%% POINTS FORMATTING %%%%%
\renewcommand{\questionshook}{
	\setlength{\rightpointsmargin}{1.75cm}
	\setlength{\itemsep}{30pt}
}

%%%%% QUESTION/PART/SUBPART INDENTATION %%%%%
\renewcommand{\partshook}{
	\renewcommand\makelabel[1]{\rlap{##1}\hss}
	% \setlength{\itemsep}{6pt}
}

\renewcommand{\subpartshook}{
	\renewcommand\makelabel[1]{\rlap{##1}\hss}
	% \setlength{\itemsep}{6pt}
}

\renewcommand{\choiceshook}{
	\setlength{\labelsep}{10pt}
	\settowidth{\leftmargin}{\circled{W}.\hspace{5pt}\hspace{0em}}
	\setlength{\itemsep}{10pt}
}

\renewcommand{\solutiontitle}{
	\noindent\textbf{Solution:}\par\noindent
}

%%%%% ONEPAR CHOICES SPREAD %%%%%
\patchcmd{\oneparchoices}{\penalty -50\hskip 1em plus 1em\relax}{\hfill}{}{}
\patchcmd{\oneparchoices}{\penalty -50\hskip 1em plus 1em\relax}{\hfill}{}{}

%%%%% SOLUTION ENVIRONMENT %%%%%
\SolutionEmphasis{\color{NavyBlue}}
\correctchoiceemphasis{\color{NavyBlue}\bfseries\boldmath}
\marksnotpoints{}
\pointsinrightmargin{}
\pointsdroppedatright{}
\pointformat{\bfseries\textbf[\themarginpoints]}

%%%%% REDEFINE COVER PAGINATION AS ARABIC %%%%%
\makeatletter
\renewenvironment{coverpages}{%
	\ifnum \value{numquestions}>0\relax
		\ClassError{exam}{%
			Coverpages cannot be used after questions have begun.\MessageBreak
		}{%
			All question, part, subpart, and subsubpart environments
			\MessageBreak
			must begin after the cover pages are complete.\MessageBreak
		}%
	\fi
	\@coverpagestrue
	% \pagenumbering{arabic}%
	\adj@hdht@ftht
	\thispagestyle{headandfoot}
}{%
	\clearpage
	% \setcounter{num@coverpages}{\value{page}}%
	% \addtocounter{num@coverpages}{-1}%
	% \pagenumbering{arabic}%
	% Bugfix, Version 2.307\beta, 2009/06/11:
	% We have to say \@coverpagesfalse before \adj@hdht@ftht
	% because we're still inside the group created by the
	% coverpages environment and we want to set the
	% extraheadheight and extrafootheight to the values correct
	% for the first non-cover page:
	\@coverpagesfalse
	\adj@hdht@ftht
}
\makeatother

%%%%% SHOW 'SOLUTIONS' IN TITLEPAGE IF ANSWERS PRINTED %%%%%
\providerobustcmd*{\printsolntitle}{
	\ifprintanswers{
		\vspace{20pt}
		\fbox{\fbox{\Huge{\textssc{\textbf{\textcolor{red}{solutions}}}}}}
		\vspace{20pt}
	}
	\fi{}
}
\providerobustcmd*{\envupspace}{\ifanswers{\vspace{20pt}}}

%%%%% REMOVE SPACES FROM EnvFullwidth command
\BeforeBeginEnvironment{EnvFullwidth}{\vspace{-10pt}}
\AfterEndEnvironment{EnvFullwidth}{\vspace{-35pt}}

%%%%%%%%%%%%%%%%%%%%%%%%%%%%%%%%%%%%%%%%%%%%%%%%%%%%%%%%%%%%%%%%%%%%%%%%%%%%%%%%%%%%
%
%%%%% SIUNITX SETUP %%%%%
%
%%%%%%%%%%%%%%%%%%%%%%%%%%%%%%%%%%%%%%%%%%%%%%%%%%%%%%%%%%%%%%%%%%%%%%%%%%%%%%%%%%%%

\DeclareSIUnit{\year}{y}
\DeclareSIUnit{\AU}{AU}
\DeclareSIUnit{\parsec}{pc}
\DeclareSIUnit{\lightyear}{ly}
\DeclareSIUnit{\earthmass}{\textit{M}_{\earth}}
\DeclareSIUnit{\jupitermass}{\textit{M}_{J}}
\DeclareSIUnit{\solarmass}{\textit{M}_{\astrosun}}
\DeclareSIUnit{\atm}{atm}

%%%%%%%%%%%%%%%%%%%%%%%%%%%%%%%%%%%%%%%%%%%%%%%%%%%%%%%%%%%%%%%%%%%%%%%%%%%%%%%%%%%%
%
%%%%% MISCELLANEOUS %%%%%
%
%%%%%%%%%%%%%%%%%%%%%%%%%%%%%%%%%%%%%%%%%%%%%%%%%%%%%%%%%%%%%%%%%%%%%%%%%%%%%%%%%%%%

%%%%% SET NUMBERED AND BULLETED LIST MARGIN
\setlist[itemize, 1]{left=0pt}
\setlist[enumerate, 1]{left=0pt,label=\arabic*.}

%%%%% PURPOSE FORGOTTEN %%%%%
\AtBeginEnvironment{tabularx}{
	\tablining
	\sisetup{text-rm={\tablining}}
}

\renewcommand\tabularxcolumn[1]{m{#1}}

\titlelabel{\vspace{-4cm}\thetitle\hspace{10pt}}
\setlength{\jot}{8pt}

%%%%% COPY-PASTABLE PDF %%%%%
\input{glyphtounicode}
\pdfgentounicode=1

%%%%% MICROTYPE SETUP FOR HYPHENATION %%%%%
\pretolerance=2500
\tolerance=4500
\emergencystretch=0pt
\righthyphenmin=4
\lefthyphenmin=4

%%%%% DEFAULT CENTRED FIGURES %%%%%
\AtBeginEnvironment{figure}{\centering}

%%%%% RADIO BUTTONS %%%%%
\makeatletter
\providerobustcmd*{\radiobutton}{%
	\@ifstar{\@radiobutton0}{\@radiobutton1}%
}
\providerobustcmd*{\@radiobutton}[1]{%
	\begin{tikzpicture}
		\pgfmathsetlengthmacro\radius{height("X")/2}
		\draw[radius=\radius] circle;
		\ifcase#1 \fill[radius=.6*\radius] circle;\fi
	\end{tikzpicture}
}
\makeatother

%%%%%%%%%%%%%%%%%%%%%%%%%%%%%%%%%%%%%%%%%%%%%%%%%%%%%%%%%%%%%%%%%%%%%%%%%%%%%%%%%%%%
%
% ASTROCHALLENGE SETUP AND MACROS %%%%%
%
%%%%%%%%%%%%%%%%%%%%%%%%%%%%%%%%%%%%%%%%%%%%%%%%%%%%%%%%%%%%%%%%%%%%%%%%%%%%%%%%%%%%

%%%%% ASTROCHALLENGE PAPER DATE %%%%%
\providerobustcmd*{\setacpaperdate}[1]{\DTMsavedate{theacpaperdate}{#1}}
\providerobustcmd*{\acpaperdate}{\DTMusedate{theacpaperdate}}

%%%%% ASTROCHALLENGE SMALL CAPS %%%%%
\providecommand{\astrochallenge}{\textbf{\textssc{AstroChallenge \propold\DTMfetchyear{theacpaperdate}}}}

%%%%% AUTOMATIC CATEGORY AND ROUND WITH KEY-VALUE SYSTEM %%%%%
\providebool{ismcq}
\providebool{isteam}
\providebool{isobs}
\providebool{isdaq}

\pgfkeys{
	/ac/.is family, /ac/.cd,
	category/.is choice,
	round/.is choice,
	% DEFINE ACTIONS FOR CATEGORY
	/ac/category/.cd,
	jnr/.code={\edef\category{Junior}},
	snr/.code={\edef\category{Senior}},
	junior/.code=\pgfkeysalso{category/jnr},
	senior/.code=\pgfkeysalso{category/snr},
	% DEFINE ACTIONS FOR CHOICES
	/ac/round/.cd,
	mcq/.code={\edef\round{MCQ}\booltrue{ismcq}},
	team/.code={\edef\round{Team}\booltrue{isteam}},
	obs/.code={\edef\round{Observation}\booltrue{isobs}},
	daq/.code={\edef\round{Data Analysis}\booltrue{isdaq}},
	% ALTERNATIVE NAMES
	MCQ/.code=\pgfkeysalso{round/mcq},
	Team/.code=\pgfkeysalso{round/team},
	Obs/.code=\pgfkeysalso{round/obs},
	DAQ/.code=\pgfkeysalso{round/daq},
}
\providecommand*{\setcatround}[1]{\pgfqkeys{/ac}{#1}}
\providecommand*{\catround}{\textbf{\textssc{\category{} \round{} Round}}}

%%%%% ASTROCHALLENGE TITLING %%%%%
\title{{\Huge\astrochallenge{}}}
\author{\textcopyright \ National University of Singapore Astronomical Society \\
	\textcopyright \ Nanyang Technological University Astronomical Society \\
}

%%%%% ACCESS TITLE COMMANDS %%%%%
\makeatletter
\let\newtitle\@title
\let\newauthor\@author
\makeatother

%%%%% EXAM HEADER/FOOTER SETUP %%%%%
\coverheader{\small{\astrochallenge{}}}{}{\small{\catround{}}}
\header{\small{\astrochallenge{}}}{}{\small{\catround{}}}
\headrule{}

%%%%% PAGE NUMBERING: 'Page X of total' %%%%%
\coverfooter{}{Page \thepage{} of \totalnumpages{}}{\oddeven{\textbf{[Turn over]}}}
\footer{}{Page \thepage{} of \totalnumpages{}}{
	\oddeven{
		\iflastpage{\textbf{[End of paper]}}{\textbf{[Turn over]}}
	}
}

%%%%% ASTROCHALLENGE INSTRUCTIONS: MCQ %%%%%
\providerobustcmd*{\acmcqinst}{
	\begin{enumerate}[itemsep=8pt]
		\item This paper consists of \textbf{\totalnumpages} printed pages, including this cover page.
		\item Do \textbf{NOT} turn over this page until instructed to do so.
		\item You have \textbf{2 hours} to attempt all questions in this paper. If you think there is more than one correct answer, choose the most correct answer.
		\item At the end of the paper, submit this booklet together with your answer script.
		\item Your answer script should clearly indicate your name, school, and team.
		\item It is \textit{your} responsibility to ensure that your answer script has been submitted.
	\end{enumerate}
}

%%%%% ASTROCHALLENGE INSTRUCTIONS: TEAM %%%%%
\providerobustcmd*{\acteaminst}{
	\begin{enumerate}[itemsep=8pt]
		\item This paper consists of \textbf{\totalnumpages} printed pages, including this cover page.
		\item Do \textbf{NOT} turn over this page until instructed to do so.
		\item You have \textbf{2 hours} to attempt all questions in this paper.
		\item At the end of the paper, submit this booklet together with your answer script.
		\item Your answer script should clearly indicate your name, school, and team.
		\item It is your responsibility to ensure that your answer script has been submitted.
		\item The marks for each question are given in brackets in the right margin, like such: \textbf{[2]}.
		\item The \textbf{alphabetical} parts (i) and (l) have been intentionally skipped, to avoid confusion with the Roman numbering of (i).
	\end{enumerate}
}

%%%%% ASTROCHALLENGE INSTRUCTIONS: DAQ %%%%%
\providerobustcmd*{\acdaqinst}{
	In this part of \astrochallenge{}, you will work with a moderately large (approx. \num{4000} points) data set. You will process this data set, analyse it, observe trends, and draw conclusions. \textbf{There are no right or wrong answers}; you will be marked solely on the quality of your analysis, even if your statistical methods are incorrect.\\[8pt]
	We \textbf{strongly} recommend you use industry-standard tools like \texttt{Microsoft Excel}\texttrademark, \texttt{RStudio} or various \texttt{Python} libraries to process the data.\par
}

%%%%% ASTROCHALLENGE INSTRUCTIONS: OBS %%%%%
\providerobustcmd*{\acobsinst}{
	\begin{enumerate}[itemsep=8pt]
		\item This paper consists of \textbf{\totalnumpages} printed pages, including this cover page.
		\item Do \textbf{NOT} turn over this page until instructed to do so.
		\item You have \textbf{1 hour and 30 minutes} to attempt all questions in this paper.
		\item At the end of the paper, submit this booklet together with your answer script.
		\item Your answer script should clearly indicate your name, school, and team.
		\item It is your responsibility to ensure that your answer script has been submitted.
	\end{enumerate}
}

%%%%% ASTROCHALLENGE INSTRUCTIONS %%%%%
\providerobustcmd*{\acinstructions}{
	\ifbool{ismcq}{\acmcqinst}{
		\ifbool{isteam}{\acteaminst}{
			\ifbool{isdaq}{\acdaqinst}{
				\ifbool{isobs}{\acobsinst}{}
			}
		}
	}
}

%%%%% ASTROCHALLENGE INSTRUCTION BOX %%%%%
\providerobustcmd*{\acinstbox}{
	\vspace*{20pt}
	\fbox{
		\fbox{
			\parbox{0.8\textwidth}{
				\vspace*{5pt}
				\begin{center}
					{\large{\textbf{\textssc{please read these instructions carefully.}}}}
				\end{center}
				\vspace*{10pt}

				\acinstructions{}

				\vspace*{5pt}
			}
		}
	}
}

%%%%% ASTROCHALLENGE COVER PAGE %%%%%
\providerobustcmd*{\accoverpage}{
	\begin{coverpages}
		\begin{center}
			\includegraphics[width=0.8\linewidth]{../graphics/misc/logo.jpg}

			\newtitle{}

			{\Huge{\catround{}}}

			\printsolntitle{}

			\acpaperdate{}

			\acinstbox{}

			\vspace*{20pt}

			\newauthor{}
		\end{center}
	\end{coverpages}
}


\coverfirstpageheader{\footnotesize{\textsc{\textbf{AstroChallenge \figureversion{osf}2019}}}}{}{\footnotesize{\textsc{\textbf{Senior Observation Round---Theory Set \circled{A}}}}}
\firstpageheader{\footnotesize{\textsc{\textbf{\figureversion{osf}AstroChallenge 2019}}}}{}{\footnotesize{\textsc{\textbf{Senior Observation Round---Theory Set \circled{A}}}}}
\runningheader{\footnotesize{\textsc{\textbf{AstroChallenge \figureversion{osf}2019}}}}{}{\footnotesize{\textsc{\textbf{Senior Observation Round---Theory Set \circled{A}}}}}

\begin{document}
\begin{coverpages}
	\begin{center}
		\includegraphics[width=0.8\linewidth]{Graphics/Misc/logo.jpg}
	
		\newtitle
		
		{\Huge{\textssc{\textbf{Senior Observation Round}}}}
		
		{\Huge{\textssc{\textbf{(Theory)}}}}
		
		%\printsolns
		
		{\Large{Monday, \nth{3} June 2019}}
		
		\vspace*{20pt}
		\fbox{
			\fbox{
				\parbox{0.9\textwidth}{
					\vspace*{5pt}
					\begin{center}
						{\Large{\textsc{\textbf{Please read these instructions carefully.}}}}
					\end{center}
					\vspace*{10pt}
					
					\begin{enumerate}[itemsep=8pt]
						\item This paper consists of \textbf{\totalnumpages} printed pages, including this cover page.
						\item Do \textbf{NOT} turn over this page until instructed to do so.
						\item You have \textbf{1 hour and 30 minutes} to attempt all questions in this paper. 
						\item At the end of the paper, submit this booklet together with your answer script.
						\item Your answer script should clearly indicate your name, school, and team.
						\item It is your responsibility to ensure that your answer script has been submitted.
						%\item The marks for each question are given in brackets in the right margin, like such: \textbf{[2]}.
						%\item The \textbf{alphabetical} parts (i) and (l) have been intentionally skipped, to avoid confusion with the Roman numbering of (i).
					\end{enumerate}
					\vspace*{5pt}
				}
			}
		}
	\end{center}
\pagenumbering{arabic}                     
\end{coverpages}
\newpage
\section{Cloze Passages \hfill [20\%]}
Instead of starting with eye-taxing constellation identification, the quiz-masters this year have come up with a fun activity to get you warmed up. You are given \textbf{two} cloze passages, and you are required to identify objects \circled{\textbf{A}} to \circled{\textbf{K}} for each. Each blank is worth \textbf{2 marks}, and partially-correct answers \textit{may} be awarded \textbf{1 mark}.

{\fbox{\textbf{Write your answers in the appropriate blanks on pages \pageref{cloze1} and \pageref{cloze2}.}}}

\subsection*{Passage 1} 
Constellation \circled{\textbf{A}} is one of the forty-eight original constellations listed by the classical Greek astronomer, Ptolemy. It was named after a vain queen in classical Greek mythology, and is circumpolar North of \ang{34} N. Its  $\alpha$  star is Schedar, but is often shaded by its $ \gamma $ Star in the centre of the constellation, which is quite variable. The constellation contains some of the most luminous stars known, including white and yellow hypergiants. An extremely rich section of the Milky Way runs through this constellation. 

On one side of \circled{\textbf{A}} lies constellation \circled{\textbf{B}}. This is another of the forty-eight classical constellations, and is named after a classical Greek hero, who was most famous for having killed the gorgon, Medusa. This hero rescued and eventually married the daughter of the namesake of \circled{\textbf{A}}. Incidentally, constellation \circled{\textbf{C}} was also named after this daughter, not far from \circled{\textbf{B}}.

In between \circled{\textbf{A}} and \circled{\textbf{B}} lie many famous star clusters. The most famous of these star clusters is deep sky object \circled{\textbf{F}}, also known as H and $ \chi $ (chi) of \circled{\textbf{B}}. The Heart and Soul nebulae are nearby, too.

The $ \alpha $ star of \circled{\textbf{C}}, star \circled{\textbf{D}}, is a binary. It is part of asterism \circled{\textbf{E}}, easily visible to the naked eye in moderately dark conditions. Two galaxies, easily visible to the naked eye, are nearby; the dimmer of which is galaxy \circled{\textbf{J}}.  In  between these two galaxies lies the $ \beta $ star of \circled{\textbf{C}}, star \circled{\textbf{K}}.

On the other side of \circled{\textbf{A}} lies another constellation, \circled{\textbf{G}}, and its namesake was the father of the namesake of \circled{\textbf{C}}, in Greek mythology. \circled{\textbf{G}} contains the famous Herschel’s Garnet Star, a variable red hypergiant located at the edge of the Elephant Trunk Nebula. This star is one of the largest ever known, with an estimated radius of over 1000 solar radii.

Furthermore, \circled{\textbf{G}} also contains the oldest known open cluster, the Polarissima Cluster (Caldwell 1). This cluster is because of its proximity to the North celestial pole.  Unlike other open clusters which tend drift apart within a few hundred million years, the Polarissima Cluster is believed to be 6.8 billion years old!

Extending a line from one of the branches of \circled{\textbf{B}}, one arrives at open cluster \circled{\textbf{H}}, which can be seen easily with the naked eye under moderately dark skies. Unlike the Polarissima Cluster, \circled{\textbf{H}} is estimated to last for only another 250 million years, after which it will likely disperse.

\newpage
\subsection*{Passage 2}
Constellation \circled{\textbf{A}} is the smallest IAU constellation, and is easily distinguishable even in light-polluted Singapore.  What distinguishes \circled{\textbf{A}} from other, nearby asterisms of similar shape is the presence of a fifth star, which is rather dim. Coincidentally, the brightness of the stars of this constellation, decrease in a clockwise manner.

Star \circled{\textbf{C}}, the brightest star of \circled{\textbf{A}}, is a double star. If one were to draw a line from the $ \gamma $ star of the constellation, star \circled{\textbf{E}} (another double star) to \circled{\textbf{C}} and carry on, one would arrive at the South celestial pole. Immediately next to the $ \alpha $ star of \circled{\textbf{A}} lies a famous dark nebula, \circled{\textbf{B}}.

Star \circled{\textbf{D}} is the $ \beta $ star of \circled{\textbf{A}}. is This star is well known for being close to a favourite deep sky object, cluster \circled{\textbf{F}}; the latter is readily seen with a pair of binoculars from Singapore and is extremely easy to find due to its proximity to \circled{\textbf{D}}.

Drawing a line from the $ \delta $ star of \circled{\textbf{A}} to \circled{\textbf{E}}, one arrives at \circled{\textbf{G}}, the largest globular cluster in the Milky Way. A line from \circled{\textbf{D}} to the $ \varepsilon $ star of \circled{\textbf{A}}, leads to the Pearl Cluster which is only slightly North of \circled{\textbf{H}}, a famous emission nebula. 

Extending this line further leads to \circled{\textbf{J}}, a star famous for its eruptive outburst in the 19th century. Star cluster \circled{\textbf{K}} is slightly South of \circled{\textbf{J}}. This star cluster is well known for its resemblance to its northern counterpart located in Taurus.

\newpage 
\section{Observation Plan \hfill [20\%]}
In this section, you are to create an observation plan for a particular location and time, as follows:

\begin{figure}[H]
	\centering
	\begin{tabularx}{0.8\textwidth}{@{}lX@{}}
		\toprule
		\textbf{\textsc{Location Data}} & Diaz Point, Lüderitz, Namibia \\
		\textbf{\textsc{Coordinates}} & \ang{26;38;7.7} S, \ang{15;5;17.1} E \\
		\textbf{\textsc{Time}} & Local sunset, \nth{4} July, 2019 to local sunrise, \nth{5} July, 2019. \\
		\textbf{\textsc{Equipment}} & \SI{305}{\milli\metre}/\SI{1500}{\milli\metre} classical Newtonian reflector on Dobsonian mount, manually operated. Wide range of Barlow eyepieces, focal reducers and eyepieces.\\
		\bottomrule
	\end{tabularx}
\end{figure}

For each object, you are to state its
\begin{itemize}[leftmargin=10pt]
	\item common name or astronomical code, and
	\item the type of the object.
\end{itemize}
You should arrange your objects into the order which you recommend people view them in. You are allowed to choose from all celestial objects, \textit{including} man-made satellites and space stations. Up to an additional 10\% will be awarded for filling in the Remarks/Object Comments field. You may want to fill this column with information on what users should look out for when trying to observe such an object. 

You may assume unrestricted viewing, a completely clear sky, zero light pollution, and a perfectly-collimated telescope.

\textbf{NOTE}: There may be deductions for objects given which do not exist in the night sky at that time and date. Marks may also be deducted for invisible or absurd objects like black holes and the cosmic microwave background. 

Each object listed will be awarded as follows:
\begin{figure}[H]
	\centering
	\begin{tabularx}{0.7\linewidth}{@{}Xr@{}}
		\toprule
		\textbf{Type} & \textbf{Marks} \\ \midrule
		Special stars (white dwarves, double and multiple star systems) & 3 \\
		Open clusters &  9 \\
		Globular clusters & 10 \\
		Galaxies (maximum of 9) & 15 \\
		Nebulae & 14 \\
		\bottomrule
	\end{tabularx}
\end{figure}

\textbf{Write your answers in Annex A, provided separately.}

\newpage
\section{Constellation Identification \hfill [20\%]}
In this section, you are to solve a total of \textbf{six} star maps. In each map \circled{\textbf{1}} to \circled{\textbf{6}}, you are to
\begin{enumerate}[leftmargin=12pt]
	\item Identify  and link \textbf{one} complete constellation; \hfill \textbf{[4 + 4]}
	\item name any \textbf{two} bright stars; \hfill \textbf{[3 + 3]}
	\item identify \textbf{four} deep-sky objects or double stars. \hfill \textbf{[4 $ \times $ 4]}
\end{enumerate}

The objects in points 2 and 3 do \textbf{not} have to be within the complete constellation identified in point 1, but they have to be \textbf{within} the image.

{\fbox{\textbf{Write your answers in the star maps on pages \pageref{starchart} to \pageref{lastpage}.}}}

\newpage
\section{Celestial Labyrinth \hfill [25\%]}
In this segment, you are to imagine that you are limited by the equipment below:
\begin{itemize}[leftmargin=10pt]
	\item \SI{50}{\milli\metre} finder (\ang{7} FOV, 2$ \times $ magnification)
	\item Altitude-azimuth mount
	\item N8 telescope (\SI{200}{\milli\metre} diameter, \SI{1000}{\milli\metre} focal length, inverted image)
	\item Eyepieces of focal length \SI{32}{\milli\metre}, \SI{10}{\milli\metre}, \SI{5}{\milli\metre} 
\end{itemize}
You will be given a star chart drawn by your team mates. You are to simulate navigating (using \textbf{only} the equipment above) from a given start point to a given end point, \textbf{within 5 minutes}. 

\section{Race Against Time in Stellarium \hfill [15\%]}
In this section, you are to find as many objects as possible in Stellarium, \textbf{within 5 minutes}. Objects must be easily seen with a small telescope. Each object will be granted points as follows:
\begin{figure}[H]
	\centering
	\begin{tabularx}{0.7\linewidth}{@{}Xr@{}}
		\toprule
		\textbf{Type} & \textbf{Marks} \\ \midrule
		Deep-sky object (DSO) & 4 each \\
		Constellation name and shape (maximum of 10) &  up to 3 \\
		Common asterism & 2 \\
		Bright star & 2 \\
		\bottomrule
	\end{tabularx}
\end{figure}

\newpage
\phantomsection\label{cloze1}
\begin{center}
	\fbox{\fbox{\centering
			\parbox{0.75\textwidth}{\centering \textsc{\textbf{Make sure you fill in your school, and your team number. \\[10pt] Detach pages \pageref{cloze1} to \pageref{lastpage}, and submit them to the quiz-masters.}}}}}
\end{center}
\vspace*{20pt}
\makebox[0.7\textwidth]{School:\enspace\hrulefill}{\quad}Team number:\enspace\hrulefill

\subsection*{Passage 1}
\renewcommand{\arraystretch}{3}%
\begin{tabularx}{\linewidth}{@{}lX@{}}
	\circled{\textbf{A}} & \underline{CASSIOPEIA} \\
	\circled{\textbf{B}} & \underline{PERSEUS} \\
	\circled{\textbf{C}} & \underline{ANDROMEDA} \\
	\circled{\textbf{D}} & \underline{ALPHERATZ} \\
	\circled{\textbf{E}} & \underline{GREAT SQUARE OF PEGASUS} \\
	\circled{\textbf{F}} & \underline{DOUBLE CLUSTER} \\
	\circled{\textbf{G}} & \underline{CEPHEUS} \\
	\circled{\textbf{H}} & \underline{PLEIADES} \\
	\circled{\textbf{J}} & \underline{MIRACH} \\
	\circled{\textbf{K}} & \underline{TRIANGULUM} \\
\end{tabularx}

\newpage
\label{cloze2}
\subsection*{Passage 2}
\renewcommand{\arraystretch}{3}%
\begin{tabularx}{\linewidth}{@{}lX@{}}
	\circled{\textbf{A}} & \underline{CRUX} \\
	\circled{\textbf{B}} & \underline{COALSACK} \\
	\circled{\textbf{C}} & \underline{ACRUX/$ \alpha $ CRUCIS} \\
	\circled{\textbf{D}} & \underline{MIMOSA} \\
	\circled{\textbf{E}} & \underline{GACRUX/$ \gamma $ CRUCIS} \\
	\circled{\textbf{F}} & \underline{JEWEL BOX} \\
	\circled{\textbf{G}} & \underline{OMEGA CENTAURI} \\
	\circled{\textbf{H}} & \underline{RUNNING CHICKEN/LAMBDA CENTAURI} \\
	\circled{\textbf{J}} & \underline{ETA CARINAE} \\
	\circled{\textbf{K}} & \underline{SOUTHERN PLEIADES} \\
\end{tabularx}

\newpage
\phantomsection\label{starchart}
\begin{center}
	\circled{\textbf{1}}
\end{center}
\includegraphics[width=\textwidth]{Graphics/Obs/chart1}

\hspace*{1cm}
\begin{center}
	\circled{\textbf{2}}
\end{center}
\includegraphics[width=\textwidth]{Graphics/Obs/chart2}

\filbreak
\begin{center}
	\circled{\textbf{3}}
\end{center}
\includegraphics[width=\textwidth]{Graphics/Obs/chart3}

\hspace*{1cm}
\begin{center}
	\circled{\textbf{4}}
\end{center}
\includegraphics[width=\textwidth]{Graphics/Obs/chart4}

\filbreak
\phantomsection\label{lastpage}
\begin{center}
	\circled{\textbf{5}}
\end{center}
\includegraphics[width=\textwidth]{Graphics/Obs/chart5}

\hspace*{1cm}
\begin{center}
	\circled{\textbf{6}}
	
	\includegraphics[width=0.8\textwidth]{Graphics/Obs/chart6}
\end{center}


\end{document}